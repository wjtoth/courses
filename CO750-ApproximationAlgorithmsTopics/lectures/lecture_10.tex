\documentclass[letterpaper,12pt,oneside,onecolumn]{article}
\usepackage[margin=1in, bottom=1in, top=1in]{geometry} %1 inch margins
\usepackage{amsmath, amssymb, amstext}
\usepackage{fancyhdr}
\usepackage{mathtools}
\usepackage{algorithm}
\usepackage{algpseudocode}
\usepackage{theorem}
\usepackage{tikz}
\usepackage{tkz-berge}

%Macros
\newcommand{\A}{\mathbb{A}} \newcommand{\C}{\mathbb{C}}
\newcommand{\D}{\mathbb{D}} \newcommand{\F}{\mathbb{F}}
\newcommand{\N}{\mathbb{N}} \newcommand{\R}{\mathbb{R}}
\newcommand{\T}{\mathbb{T}} \newcommand{\Z}{\mathbb{Z}}
\newcommand{\Q}{\mathbb{Q}}
 
 
\newcommand{\cA}{\mathcal{A}} \newcommand{\cB}{\mathcal{B}}
\newcommand{\cC}{\mathcal{C}} \newcommand{\cD}{\mathcal{D}}
\newcommand{\cE}{\mathcal{E}} \newcommand{\cF}{\mathcal{F}}
\newcommand{\cG}{\mathcal{G}} \newcommand{\cH}{\mathcal{H}}
\newcommand{\cI}{\mathcal{I}} \newcommand{\cJ}{\mathcal{J}}
\newcommand{\cK}{\mathcal{K}} \newcommand{\cL}{\mathcal{L}}
\newcommand{\cM}{\mathcal{M}} \newcommand{\cN}{\mathcal{N}}
\newcommand{\cO}{\mathcal{O}} \newcommand{\cP}{\mathcal{P}}
\newcommand{\cQ}{\mathcal{Q}} \newcommand{\cR}{\mathcal{R}}
\newcommand{\cS}{\mathcal{S}} \newcommand{\cT}{\mathcal{T}}
\newcommand{\cU}{\mathcal{U}} \newcommand{\cV}{\mathcal{V}}
\newcommand{\cW}{\mathcal{W}} \newcommand{\cX}{\mathcal{X}}
\newcommand{\cY}{\mathcal{Y}} \newcommand{\cZ}{\mathcal{Z}}

\newcommand\numberthis{\addtocounter{equation}{1}\tag{\theequation}}


\newenvironment{proof}{{\bf Proof:  }}{\hfill\rule{2mm}{2mm}}
\newenvironment{proofof}[1]{{\bf Proof of #1:  }}{\hfill\rule{2mm}{2mm}}
\newenvironment{proofofnobox}[1]{{\bf#1:  }}{}\newenvironment{example}{{\bf Example:  }}{\hfill\rule{2mm}{2mm}}

%\renewcommand{\thesection}{\lecnum.\arabic{section}}
%\renewcommand{\theequation}{\thesection.\arabic{equation}}
%\renewcommand{\thefigure}{\thesection.\arabic{figure}}

\newtheorem{fact}{Fact}[section]
\newtheorem{lemma}[fact]{Lemma}
\newtheorem{theorem}[fact]{Theorem}
\newtheorem{definition}[fact]{Definition}
\newtheorem{corollary}[fact]{Corollary}
\newtheorem{proposition}[fact]{Proposition}
\newtheorem{claim}[fact]{Claim}
\newtheorem{exercise}[fact]{Exercise}
\newtheorem{note}[fact]{Note}
\newtheorem{conjecture}[fact]{Conjecture}

\newcommand{\size}[1]{\ensuremath{\left|#1\right|}}
\newcommand{\ceil}[1]{\ensuremath{\left\lceil#1\right\rceil}}
\newcommand{\floor}[1]{\ensuremath{\left\lfloor#1\right\rfloor}}

%END MACROS
%Page style
\pagestyle{fancy}

\listfiles

\raggedbottom

\lhead{2017-02-07}
\rhead{William Justin Toth CO750-Approximation Algorithms Lecture 10} %CHANGE n to ASSIGNMENT NUMBER ijk TO COURSE CODE
\renewcommand{\headrulewidth}{1pt} %heading underlined
%\renewcommand{\baselinestretch}{1.2} % 1.2 line spacing for legibility (optional)

\begin{document}
\paragraph{}
We focus on the proof of the result that there exists a $1.39$ approximation algorithm for Steiner Tree.
Recall that $$\sum_t E(c(C^t)] \leq \sum_{t} E[\sum_j \frac{x_j^t}{M} c(C_j)] \leq \frac{1}{M} \sum_t E[opt^t_{k,f}] \leq \frac{1}{M} \sum_t E[opt^t_k].$$
Let $S^*$ be the optimal $k$-restricted Steiner Tree. Each time we sample $C^t$ we delete a subset of edges of $S^*$. Let $S^* = S^0 \supseteq S^1 \supseteq S^2 \supseteq \dots$ be the subgraphs obtained in this manner. We will show:
\begin{enumerate}
\item $S^t$ plus the sampled components until iteration $t$ span the terminals.
\item Each edge in $S^*$ is deleted after an (expected) number of at most $\ln (4) \cdot M$ iterations.
\end{enumerate}
$(1) + (2)$ implies the existence of a $\ln(4)$-approximation algorithm for $k$-restricted Steiner Tree.
\paragraph{Proof of $(1)$}
We begin with demonstrating $(1)$. How do we delete edges from $S^*$? By the following random process:
\begin{enumerate}
\item Convert $S^*$ to a binary tree by adding $0$-cost edges if necessary. Note that the height of the tree is at most $|R|-1$ at this stage/
\item For $u,v \in R$ let $P_{uv}$ denote the unique $uv$-path on $S^*$.
\item For each Steiner node, choose uniformly at random one of the two edges going to its children. Call the set of all such edges $\tilde{B}$.
\item Clearly $Pr[e \in \tilde{B}] = \frac{1}{2}$ for all $e \in S^*$.
\item Set $W = \{ uv \in {R\choose 2} : |P_{uv} \cap \tilde{B}| = 1\}$.
\item For each $e \in S^*$, let $W(e) = \{uv \in W: e \in P_{uv}\}$.
\item For a given component $C$ we let the set of candidate bridges be $\cB(C) = \{B \subseteq W: |B| = |R(C)| - 1, (W\backslash B) \cup C \text{ is connected} \}$.
\end{enumerate}
\paragraph{Deletion Process}
Let $C^t$ be the component sampled in iteration $t$. Delete from $W$ a random subset $\bar{B_{r_w}}(C^t) \in \cB_w (C^t)$ according to a probability distribution: $\omega_{C^t} : \cB_w (C^t) \rightarrow [0,1].$ Delete $e \in S^*$ as soon as all edges in $W(e)$ are deleted.
\paragraph{Lemma 1}
The graph $S^t \cup \bigcup_{t' = 1} ^t C^{t'}$ spans $R$.
\paragraph{Proof of Lemma 1}
Let $W^t \subseteq W$ be the set of edges of $W$ that are not deleted at iteration $t$. By the definition of $\cB_w$, $W^t \cup \bigcup_{t' = 1} ^t C^{t'}$ spans $R$. For all $e = uv \in W^t$, $P_{uv} \in W(e)$, all edges in $P_{uv}$ are not deleted. Thus $uv$ are connected in $S^t$. $\blacksquare$.
\paragraph{Proof of $(2)$}
We now consider part $(2)$. We need the following lemma.
\paragraph{Lemma $2$}
There exist probability distributions $\omega_c: \cB_w(C) \rightarrow [0,1]$ such that, in each iteration $t$, $e \in W$ gets deleted with probability at least $\frac{1}{M}$.
\paragraph{Proof of Lemma $2$}
We claim that there exists $\omega_c$ such that
\begin{equation}\label{eq:star}\sum_C \sum_{B \in \cB(C) : e\in B} x_C \cdot \omega_C (B) \geq 1\end{equation}
for all $e \in W$. If equation \ref{eq:star} holds, then $Pr[e \in \bar{B_{r_w}} (C^t)] = \sum_{C} \sum_{B\in \cB_w(C) : e \in B} \frac{x_C}{M} \cdot \omega_C(B) \geq \frac{1}{M}$ as desired. If equation \ref{eq:star} does not hold then the following linear program has no solution:
\begin{align*}
\sum_{C}\sum{B \in \cB_w(C): e\in B}x_C \cdot \omega_C(B) &\geq 1, &\forall e \in E\\
\sum_{B \in \cB(C)} \omega_C(B) &\leq 1 &\forall C \\
\omega_C(B) \geq 0.
\end{align*}
By Farkas Lemma, the following system has a solution:
\begin{align*}
y_C &\geq \sum_{e \in B} c_e x_C &\forall C, \forall B \in \cB_W(C) \\
\sum_C y_C &< \sum_{e \in W} c_e\\
(y,c) &\geq 0.
\end{align*}
Let us interpret the values of $c$ variables as edge costs. Then $y_C \geq x_C \sum_{e \in B} c_e$ for all $B$. So $y_C \geq x_C \cdot \max\{c(B) : B \in \cB_W(C)\} = x_C \cdot c(\cB_{r_w}(C))$ (bridges with $c$ as cost function). Then $\sum_C x_C\cdot \text{cost}(\cB_{r_w}(C)) \leq \sum_{C} y_C < \sum_{e\in W} c_e = \text{cost}(W)$. This contradicts bridge lemma (should be $\geq$ by lemma). $\blacksquare$.
\paragraph{}
Summarize: there exists probability distributions $\omega_C$ for all $C$ such that each edge $e \in W$ has a ``uniform" probability to be deleted in each iteration. Thus $\frac{1}{M}$. Now to more forward against $(2)$ what is the probability that an edge $e \in S^*$ is deleted? In other words, what is the expected number of iterations until $e$ is deleted?
\end{document}
