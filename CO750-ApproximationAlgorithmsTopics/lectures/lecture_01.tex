\documentclass[letterpaper,12pt,oneside,onecolumn]{article}
\usepackage[margin=1in, bottom=1in, top=1in]{geometry} %1 inch margins
\usepackage{amsmath, amssymb, amstext}
\usepackage{fancyhdr}
\usepackage{algorithm}
\usepackage{algpseudocode}
\usepackage{mathtools}
\usepackage{theorem}
\usepackage{tikz}
\usepackage{tkz-berge}

%Macros
\newcommand{\A}{\mathbb{A}} \newcommand{\C}{\mathbb{C}}
\newcommand{\D}{\mathbb{D}} \newcommand{\F}{\mathbb{F}}
\newcommand{\N}{\mathbb{N}} \newcommand{\R}{\mathbb{R}}
\newcommand{\T}{\mathbb{T}} \newcommand{\Z}{\mathbb{Z}}
\newcommand{\Q}{\mathbb{Q}}
 
 
\newcommand{\cA}{\mathcal{A}} \newcommand{\cB}{\mathcal{B}}
\newcommand{\cC}{\mathcal{C}} \newcommand{\cD}{\mathcal{D}}
\newcommand{\cE}{\mathcal{E}} \newcommand{\cF}{\mathcal{F}}
\newcommand{\cG}{\mathcal{G}} \newcommand{\cH}{\mathcal{H}}
\newcommand{\cI}{\mathcal{I}} \newcommand{\cJ}{\mathcal{J}}
\newcommand{\cK}{\mathcal{K}} \newcommand{\cL}{\mathcal{L}}
\newcommand{\cM}{\mathcal{M}} \newcommand{\cN}{\mathcal{N}}
\newcommand{\cO}{\mathcal{O}} \newcommand{\cP}{\mathcal{P}}
\newcommand{\cQ}{\mathcal{Q}} \newcommand{\cR}{\mathcal{R}}
\newcommand{\cS}{\mathcal{S}} \newcommand{\cT}{\mathcal{T}}
\newcommand{\cU}{\mathcal{U}} \newcommand{\cV}{\mathcal{V}}
\newcommand{\cW}{\mathcal{W}} \newcommand{\cX}{\mathcal{X}}
\newcommand{\cY}{\mathcal{Y}} \newcommand{\cZ}{\mathcal{Z}}

\newcommand\numberthis{\addtocounter{equation}{1}\tag{\theequation}}


\newenvironment{proof}{{\bf Proof:  }}{\hfill\rule{2mm}{2mm}}
\newenvironment{proofof}[1]{{\bf Proof of #1:  }}{\hfill\rule{2mm}{2mm}}
\newenvironment{proofofnobox}[1]{{\bf#1:  }}{}\newenvironment{example}{{\bf Example:  }}{\hfill\rule{2mm}{2mm}}

%\renewcommand{\thesection}{\lecnum.\arabic{section}}
%\renewcommand{\theequation}{\thesection.\arabic{equation}}
%\renewcommand{\thefigure}{\thesection.\arabic{figure}}

\newtheorem{fact}{Fact}[section]
\newtheorem{lemma}[fact]{Lemma}
\newtheorem{theorem}[fact]{Theorem}
\newtheorem{definition}[fact]{Definition}
\newtheorem{corollary}[fact]{Corollary}
\newtheorem{proposition}[fact]{Proposition}
\newtheorem{claim}[fact]{Claim}
\newtheorem{exercise}[fact]{Exercise}
\newtheorem{note}[fact]{Note}
\newtheorem{conjecture}[fact]{Conjecture}

\newcommand{\size}[1]{\ensuremath{\left|#1\right|}}
\newcommand{\ceil}[1]{\ensuremath{\left\lceil#1\right\rceil}}
\newcommand{\floor}[1]{\ensuremath{\left\lfloor#1\right\rfloor}}

%END MACROS
%Page style
\pagestyle{fancy}

\listfiles

\raggedbottom

\lhead{2017-01-03}
\rhead{William Justin Toth CO750-Approximation Algorithms Lecture 1} %CHANGE n to ASSIGNMENT NUMBER ijk TO COURSE CODE
\renewcommand{\headrulewidth}{1pt} %heading underlined
%\renewcommand{\baselinestretch}{1.2} % 1.2 line spacing for legibility (optional)

\begin{document}
\section{Introduction}
\paragraph{}
The central task we are interested in is solving $NP$-hard optimization problems. Finding optimal solutions to such problems in polynomial time is unlikely (unless $P = NP$).

\paragraph{}
Given that we are faced with an $NP$-hard problem there are a couple possible approaches:
\begin{enumerate}
\item Exact algorithms: Find optimal solution slowly for large input.
\item Heuristic algorithms: Find efficiently a feasible solution, but with no guarantee of optimality.
\item Approximation Algorithms: Find efficiently a feasible solution and give a certified bound on how close the solution is to optimal.
\end{enumerate}
The goal of this course is to study the third approach, and learn successful methods of designing approximation algorithms. We will draw heavily from network design for examples.

\paragraph{}
What we will see in this course:
\begin{enumerate}
\item Combinatorial techniques: Those which exploit the structure of feasible solutions.
\item Linear Programming techniques:
\begin{enumerate}
\item Primal-Dual methods: Those which exploit Duality Theory.
\item Rounding methods: Those which round fractional variables.
\end{enumerate}
\item Recent Results and Open Problems!
\end{enumerate}
\section{Vertex Cover Problem}
\paragraph{}
In an instance of {\it Vertex Cover} (VC) the input is a graph $G=(V,E)$. We may assume without loss of generality that $G$ is connected. If $G$ is not connected we may apply an algorithm which solves VC for connected graphs to each component of $G$ and union our solutions to obtain a solution for $G$. The output (solution) to an instance of VC is a subset of vertices $C \subseteq V$ of minimum cardinality satisfying $$\forall e \in E, e \cap C \neq \emptyset.$$
\paragraph{}
A possible exact algorithm would list all subsets of $V$ and return the minimum cardinality subset among those which are covers. The running time of this algorithm would be $O(2^n)$ where $n = |V|$. This is inefficient, ie. not polynomial in the size of the input graph.
\paragraph{}
An instance of a problem has a given size $m \in \Z$. The {\it running time} of the an algorithm is the number of elementary operations it performs ($+$, $-$, $\times$, $==$, $<$, $>$, $\dots$) during its execution. We measure the running time as a function of the input size $m$. An algorithm is said to be {\it efficient} if its running time is a polynomial function of $m$, ie. $O(m^k)$ for some constant $k$. For problems on graphs $n =|V|$ provides a rough estimate of the input size.
\paragraph{}
Let $\alpha \geq 1$ and let $\Pi$ be a minimization problem. We say $\cA$ is an {\it $\alpha$-approximation algorithm} for $\Pi$ provided $\cA$ is efficient on all instances of $\Pi$ and for every instance $\pi$ of $\Pi$, the solution returned by $\cA$ on $\pi$ is feasible and has objective value at most $\alpha$ times the optimal value of a solution for $\pi$. 
\paragraph{}
If instead $\Pi$ is a maximation problem then $\alpha \leq 1$ and $\cA$ returns a solution at least $\alpha$ times the optimal value.
\paragraph{}
Typical values of $\alpha$ are $2$, $O(1)$, $\log n$, or $(1 + \epsilon)$ for instance.
\paragraph{}
Returning our attention to VC, let $C^*$ be an optimal solution of a given instance of $VC$. Our goal is to find a feasible solution $C$ such that $|C| \leq \alpha |C^*|$. But we run into some difficulty since we do not know $|C^*|$. Our idea to get around this is to obtain a lower bound on $|C^*|$ and compare $|C|$ with our lower bound.
\paragraph{}
Let $M$ be a maximum (we only need maximal in full generality) matching on $G$. It is not too hard to see that $|M| \leq |C^*|$. Indeed if $|C^*| < |M|$ then there exists two edges in $M$ which are covered by the same vertex in $C^*$ (from pigeonhole principle), but this contradicts that $M$ is a matching. Note that any matching provides a lower bound on $|C^*|$, but a maximum matching yields the tightest lower bound, a desirable property.
\paragraph{}
Via the Blossom Algorithm of Edmunds a maximum cardinality (in fact maximum weight for arbitrary weights) matching can be computed efficiently.
\paragraph{Approximation Algorithm for VC}
Consider the algorithm:
\begin{enumerate}
\item Compute a maximum cardinality matching $M$
\item Output $C = \bigcup_{e \in M} e$.
\end{enumerate}
\begin{theorem}
Our Approximation Algorithm for VC is a $2$-approximation for VC.
\end{theorem}
\begin{proof}
Since maximum matching runs in polynomial time, our algorithm runs in polynomial time. To see that $C$ is feasible suppose that there exists $e \in E$ such that $e \cap C = \emptyset$. The $M \cup e$ is a matching greater cardinality than $M$. Therefore $C$ is a cover. Furthermore $$|C| = 2|M| \leq 2|C^*|$$ as desired.
\end{proof}
\paragraph{}
To see that our analysis is tight consider a graph consisting of a single edge. In this case $|C| = 2 = 2|C*|$ and hence our analysis is tight.
\section{Weighted Vertex Cover}
\paragraph{}
In an instance of {\it Weighted Vertex Cover} (WVC)the input is a connected graph $G = (V,E)$ and weights $w_v$ for all $v \in V$. The output is again a cover, but now of minimum weight, not cardinality. This problem generalizes VC whose instances can be recovered by assigned all weights the value $1$.
\paragraph{}
Our previous algorithm may fail. To see this consider a star graph with very large weight on the center vertex and very low weights on the outer vertices. The reader is encouraged to verify that the previous algorithm fails under such conditions. 
\paragraph{}
For WVC we will use linear programming (LP) to give lower bounds. Often optimization problems can be naturally formulated as integer programs (IP). Consider the following IP for WVC:
\begin{align*}
\min &\sum_{v \in V} w_v x_v \\
x_u + x_v &\geq 1 &\forall uv \in E \\
x_u& \in \{0,1\} &\forall u \in V.
\end{align*}
By relaxing the IP (change constraints $x_u \in \{0,1\}$ to $0 \leq x_u \leq 1$) we obtain an LP that can be solved efficiently. Since this LP has $O(|V|^2)$ constraints we can solve it in time polynomial in the size of our input graph. This LP gives a lower bound on $OPT$, the optimal value of our WVC instance.
\paragraph{Approximation Algorithm for WVC}
Consider the algorithm:
\begin{enumerate}
\item Solve the LP relaxation. Let $x^*$ be the optimal solution of the LP.
\item (Round up all variables of value at least $\frac{1}{2}$) Output: $C = \{v \in V: x^*_v \geq \frac{1}{2}\}$.
\end{enumerate}
\begin{theorem}
Our Approximation Algorithm for WVC is a $2$-approximation for WVC.
\end{theorem}
\begin{proof}
Algorithm is polynomial since compact linear programs can be solved in polynomial time. Now the weight of $C$ is
$$w(C) = \sum_{v \in C} w_v = 2\sum_{v \in C} w_v \frac{1}{2} \leq 2 \sum_{v \in C}w_v x^*_v \leq 2 \sum_{v \in V} w_v x^*_v \leq 2\cdot OPT.$$
It remains to verify that $C$ is feasible. Indeed for each edge $uv \in E$, either $x^*_u \geq \frac{1}{2}$ or $x^*_v \geq \frac{1}{2}$ by the constraint
$$x_u + x_v \geq 1$$
and thus either $u \in C$ or $v \in C$.
\end{proof}
\end{document}
