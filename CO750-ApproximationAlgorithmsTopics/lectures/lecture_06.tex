\documentclass[letterpaper,12pt,oneside,onecolumn]{article}
\usepackage[margin=1in, bottom=1in, top=1in]{geometry} %1 inch margins
\usepackage{amsmath, amssymb, amstext}
\usepackage{fancyhdr}
\usepackage{mathtools}
\usepackage{algorithm}
\usepackage{algpseudocode}
\usepackage{theorem}
\usepackage{tikz}
\usepackage{tkz-berge}

%Macros
\newcommand{\A}{\mathbb{A}} \newcommand{\C}{\mathbb{C}}
\newcommand{\D}{\mathbb{D}} \newcommand{\F}{\mathbb{F}}
\newcommand{\N}{\mathbb{N}} \newcommand{\R}{\mathbb{R}}
\newcommand{\T}{\mathbb{T}} \newcommand{\Z}{\mathbb{Z}}
\newcommand{\Q}{\mathbb{Q}}
 
 
\newcommand{\cA}{\mathcal{A}} \newcommand{\cB}{\mathcal{B}}
\newcommand{\cC}{\mathcal{C}} \newcommand{\cD}{\mathcal{D}}
\newcommand{\cE}{\mathcal{E}} \newcommand{\cF}{\mathcal{F}}
\newcommand{\cG}{\mathcal{G}} \newcommand{\cH}{\mathcal{H}}
\newcommand{\cI}{\mathcal{I}} \newcommand{\cJ}{\mathcal{J}}
\newcommand{\cK}{\mathcal{K}} \newcommand{\cL}{\mathcal{L}}
\newcommand{\cM}{\mathcal{M}} \newcommand{\cN}{\mathcal{N}}
\newcommand{\cO}{\mathcal{O}} \newcommand{\cP}{\mathcal{P}}
\newcommand{\cQ}{\mathcal{Q}} \newcommand{\cR}{\mathcal{R}}
\newcommand{\cS}{\mathcal{S}} \newcommand{\cT}{\mathcal{T}}
\newcommand{\cU}{\mathcal{U}} \newcommand{\cV}{\mathcal{V}}
\newcommand{\cW}{\mathcal{W}} \newcommand{\cX}{\mathcal{X}}
\newcommand{\cY}{\mathcal{Y}} \newcommand{\cZ}{\mathcal{Z}}

\newcommand\numberthis{\addtocounter{equation}{1}\tag{\theequation}}


\newenvironment{proof}{{\bf Proof:  }}{\hfill\rule{2mm}{2mm}}
\newenvironment{proofof}[1]{{\bf Proof of #1:  }}{\hfill\rule{2mm}{2mm}}
\newenvironment{proofofnobox}[1]{{\bf#1:  }}{}\newenvironment{example}{{\bf Example:  }}{\hfill\rule{2mm}{2mm}}

%\renewcommand{\thesection}{\lecnum.\arabic{section}}
%\renewcommand{\theequation}{\thesection.\arabic{equation}}
%\renewcommand{\thefigure}{\thesection.\arabic{figure}}

\newtheorem{fact}{Fact}[section]
\newtheorem{lemma}[fact]{Lemma}
\newtheorem{theorem}[fact]{Theorem}
\newtheorem{definition}[fact]{Definition}
\newtheorem{corollary}[fact]{Corollary}
\newtheorem{proposition}[fact]{Proposition}
\newtheorem{claim}[fact]{Claim}
\newtheorem{exercise}[fact]{Exercise}
\newtheorem{note}[fact]{Note}
\newtheorem{conjecture}[fact]{Conjecture}

\newcommand{\size}[1]{\ensuremath{\left|#1\right|}}
\newcommand{\ceil}[1]{\ensuremath{\left\lceil#1\right\rceil}}
\newcommand{\floor}[1]{\ensuremath{\left\lfloor#1\right\rfloor}}

%END MACROS
%Page style
\pagestyle{fancy}

\listfiles

\raggedbottom

\lhead{2017-01-24}
\rhead{William Justin Toth CO750-Approximation Algorithms Lecture 6} %CHANGE n to ASSIGNMENT NUMBER ijk TO COURSE CODE
\renewcommand{\headrulewidth}{1pt} %heading underlined
%\renewcommand{\baselinestretch}{1.2} % 1.2 line spacing for legibility (optional)

\begin{document}
\section{Generalization}
\paragraph{}
A function $f: 2^V \rightarrow \{0,1\}$ is called {\it uncrossable} if
\begin{enumerate}
\item $f(V) = 0$
\item If $f(A) = f(B) = 1$ for some $A,B \subseteq V$, then either
\begin{itemize}
\item$f(A\cup B) = f(A \cap B) = 1$
\item or $f(A\backslash B) = f(B\backslash A) = 1$.
\end{itemize}
\end{enumerate}
Observe that if $f$ is $01$-proper then $f$ is uncrossable (this is left as an exercise).
\begin{theorem}
The Primal-Dual algorithm of the previous lecture is a $2$-approximation algorithm for uncrossable functions provided that you have a polynomial time oracle to find minimum violated sets.
\end{theorem}
\begin{proof}
Omitted.
\end{proof}
\section{General Proper Functions}
\paragraph{}
We now consider a proper function $f : 2^V \rightarrow \N$. Consider the following IP for proper function $f$:
\begin{align*}
\min &\sum_e c_e x_e \\
\text{s.t.} \sum_{e\in \delta(S)} x_e &\geq f(S) &\forall S \subseteq V \\
1 \geq &x_e \geq 0 &\forall e \in E \\
x \in \Z^E.
\end{align*}
Our goal is to design a primal-dual approximation algorithm for solving this IP with approximation factor $O(\log f_{max})$ where $f_{max}$ is $\max_S\{f(S)\}$. Note that there exists a $2$-approximation algorithm for this IP using iterative rounding (to be seen in later lectures). It is an open problem to design a $2$-approximation algorithm for this IP using primal-dual methods.
\paragraph{}
Consider the following Primal-Dual Algorithm:
\begin{enumerate}
\item $F_0 = \emptyset$
\item For phase $p = 1, \dots, f_{max}$
	\begin{enumerate}
	\item Set $h_p(S) = 1$ if $f(S) - |\delta_{F_{p-1}(S)}| = f_{max} - p + 1$, and set $h_p(S) = 0$ otherwise.
	\item Set $E_p = E - F_{p-1}$
	\item Apply $2$-approximation algorithm using primal-dual for $(V,E_p)$ with function $h_p$, and let $A$ be the edges returned by the algorithm.
	\item Set $F_p = F_{p-1} \cup A$.
	\end{enumerate}
\item Return $F_{f_{max}}$.
\end{enumerate}
\begin{lemma}
For all iterations $p$, the function $h_p$ is uncrossable.
\end{lemma}
\begin{proof}
(Exercise).
\end{proof}
\begin{theorem}
The primal-dual algorithm given above is an $O(\log f_{max})$-approximation algorithm provided that you have a $2$-approximation primal-dual algorithm for $h_p$.
\end{theorem}
\begin{proof}
Let $y^p$ be the dual solution constructed in each iteration $p$. Set $z_e^p = 0$ for all $e \in E_p$. Set $z_e^p = \sum_{S: e\in \delta(S)} y_S^p$ otherwise. Then $(y^p, z^p)$ is dual feasible for the original problem. We compute the bound on its optimal value $D^*$:
\begin{align*}
D^* &\geq \sum_S f(S) y_S^p - \sum_{e\in E} z_e^p\\
& = \sum_S f(S) y_S^p - \sum_{e\not\in E_p}\sum_{S : e\in\delta(S)}y_S^p\\
&= \sum_S f(S) y_S^p - \sum_S |\delta_{F_{p-1}}(S)| y_S \\
&= \sum_S (f(S) - |\delta_{F_{p-1}}(S)|)y_S^p \\
&= (f_{max} - p + 1)\sum_S y_S^p.
\end{align*}
This implies that
$$\sum_{e \in F_{max}^p} c_e \leq 2 \sum_{p=1}^{f_{max}} y_S^p \leq 2 \sum_{p=1}^{f_{max}} \frac{D^*}{f_{max} - p + 1} = O(\log f_{max}) D^* .$$
\end{proof}
\end{document}
