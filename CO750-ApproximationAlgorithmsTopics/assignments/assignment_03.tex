\documentclass[letterpaper,12pt,oneside,onecolumn]{article}
\usepackage[margin=1in, bottom=1in, top=1in]{geometry} %1 inch margins
\usepackage{amsmath, amssymb, amstext}
\usepackage{fancyhdr}
\usepackage{mathtools}
\usepackage{algorithm}
\usepackage{algpseudocode}
\usepackage{theorem}
\usepackage{tikz}
\usepackage{tkz-berge}

%Macros
\newcommand{\A}{\mathbb{A}} \newcommand{\C}{\mathbb{C}}
\newcommand{\D}{\mathbb{D}} \newcommand{\F}{\mathbb{F}}
\newcommand{\N}{\mathbb{N}} \newcommand{\R}{\mathbb{R}}
\newcommand{\T}{\mathbb{T}} \newcommand{\Z}{\mathbb{Z}}
\newcommand{\Q}{\mathbb{Q}}
 
 
\newcommand{\cA}{\mathcal{A}} \newcommand{\cB}{\mathcal{B}}
\newcommand{\cC}{\mathcal{C}} \newcommand{\cD}{\mathcal{D}}
\newcommand{\cE}{\mathcal{E}} \newcommand{\cF}{\mathcal{F}}
\newcommand{\cG}{\mathcal{G}} \newcommand{\cH}{\mathcal{H}}
\newcommand{\cI}{\mathcal{I}} \newcommand{\cJ}{\mathcal{J}}
\newcommand{\cK}{\mathcal{K}} \newcommand{\cL}{\mathcal{L}}
\newcommand{\cM}{\mathcal{M}} \newcommand{\cN}{\mathcal{N}}
\newcommand{\cO}{\mathcal{O}} \newcommand{\cP}{\mathcal{P}}
\newcommand{\cQ}{\mathcal{Q}} \newcommand{\cR}{\mathcal{R}}
\newcommand{\cS}{\mathcal{S}} \newcommand{\cT}{\mathcal{T}}
\newcommand{\cU}{\mathcal{U}} \newcommand{\cV}{\mathcal{V}}
\newcommand{\cW}{\mathcal{W}} \newcommand{\cX}{\mathcal{X}}
\newcommand{\cY}{\mathcal{Y}} \newcommand{\cZ}{\mathcal{Z}}

\newcommand\numberthis{\addtocounter{equation}{1}\tag{\theequation}}


\newenvironment{proof}{{\bf Proof:  }}{\hfill\rule{2mm}{2mm}}
\newenvironment{proofof}[1]{{\bf Proof of #1:  }}{\hfill\rule{2mm}{2mm}}
\newenvironment{proofofnobox}[1]{{\bf#1:  }}{}\newenvironment{example}{{\bf Example:  }}{\hfill\rule{2mm}{2mm}}

%\renewcommand{\thesection}{\lecnum.\arabic{section}}
%\renewcommand{\theequation}{\thesection.\arabic{equation}}
%\renewcommand{\thefigure}{\thesection.\arabic{figure}}

\newtheorem{fact}{Fact}[section]
\newtheorem{lemma}[fact]{Lemma}
\newtheorem{theorem}[fact]{Theorem}
\newtheorem{definition}[fact]{Definition}
\newtheorem{corollary}[fact]{Corollary}
\newtheorem{proposition}[fact]{Proposition}
\newtheorem{claim}[fact]{Claim}
\newtheorem{exercise}[fact]{Exercise}
\newtheorem{note}[fact]{Note}
\newtheorem{conjecture}[fact]{Conjecture}

\newcommand{\size}[1]{\ensuremath{\left|#1\right|}}
\newcommand{\ceil}[1]{\ensuremath{\left\lceil#1\right\rceil}}
\newcommand{\floor}[1]{\ensuremath{\left\lfloor#1\right\rfloor}}

%END MACROS
%Page style
\pagestyle{fancy}

\listfiles

\raggedbottom

\lhead{2017-04-07}
\rhead{W. Justin Toth CO750-Approximation Algorithms Assignment 3} %CHANGE n to ASSIGNMENT NUMBER ijk TO COURSE CODE
\renewcommand{\headrulewidth}{1pt} %heading underlined
%\renewcommand{\baselinestretch}{1.2} % 1.2 line spacing for legibility (optional)

\begin{document}

%Question 1
\section{}
\paragraph{}
Let $G=(V,E)$ be a metric graph with costs $c: E \rightarrow \R_+$. Let $T \subseteq V$ be a subset of nodes. Let $\equiv_2$ denote equivalence modulo $2$ (that is $a \equiv_2 b$ if and only if $\exists k \in \Z$ such that $a = b + 2k$). Suppose that $|T| \equiv_2 0$.
\paragraph{}
The minimum cost connected $T$-Join problem asks us to find a minimum cost multiset $F$ of edges such that the set of nodes with odd degree in the multigraph $H=(V,F)$ is exactly $T$ and $H$ is connected.
\paragraph{}
Let $\cA$ denote the following algorithm for solving minimum cost connected $T$-Join on $G$:
\begin{enumerate}
\item Compute a minimum cost spanning tree on $G$. Call it $S$.
\item Let $A \subseteq T$ be the set of nodes in $T$ of even degree in $S$.
\item Let $B \subseteq V\backslash T$ be the set of nodes not in $T$ of odd degree in $S$.
\item Let $T' = A \cup B$. Compute a minimum cost $T'$-join in $G$. Call it $J$.
\item Return $S \cup J$.
\end{enumerate}
\begin{theorem}
The algorithm $\cA$ is a $\frac{5}{3}$-approximation algorithm for the minium cost connected $T$-join problem.
\end{theorem}
\begin{proof}
Standard algorithms for minimum cost spanning tree and $T$-join run in polynomial time. Hence $\cA$ is a polynomial time algorithm. The multigraph $(V, S\cup J)$ is connected since it contains a spanning tree $S$. Further by our choice of $A$ and $B$, the nodes in $T$ are precisely the nodes of odd degree in $(V, S \cup J)$. With respect to correctness of $\cA$, the only question which remains is that $|T'| \equiv_2 0$ so that a $T'$-join can be computed. 
\paragraph{}
Let any $C \subseteq V$ let $C^o := \{ v \in C: d_S(v) \equiv_2 1 \}$ where $d_S$ denotes $|\delta(v) \cap S|$. Then by our choice of $A$ and $B$: 
$$|T| = |A| + |T^o| = |A| + |V^o\backslash B| = |A| + |V^o| - |B|.$$
Since $|T| \equiv_2 0$ and $|V^o| \equiv_2 0$ (it is the set of odd degree vertices in $(V,S)$ so its cardinality is even):
$$|A| \equiv_2 |B|,$$
Thus $$|T'| \equiv_2 |A| + |B| \equiv_2 2|A| \equiv_2 0.$$
So we have verified that $|T'|$ is even. Therefore the algorithm $\cA$ terminates in polynomial time returning a feasible solution to minimum cost connected $T$-Join.
\paragraph{}
We now verify the approximation factor. Let $J^*$ denote an optimal minimum cost connected $T$-join. Since $(V,J^*)$ is connected, $J^*$ contains a spanning tree. Therefore 
$$c(S) \leq c(J^*).$$
Now since $(V,S)$ is connected it contains a $T'$-Join. Let $J_1$ be such a $T'$-Join. Also since $(V,J^*)$ is connected it contains a $T'$-Join. Let $J_2$ be such a $T'$-Join. We claim that $J_3 := S\backslash J_1 \cup J^* \backslash J_2$ is a $T'$-join. To verify this consider a vertex $v \in T'$. We will study the parity of $d(v)$ with respect various sets mentioned above. Since $J_1$ and $J_2$ are $T'$-Joins we have
$$d_{J_1}(v) \equiv_2 d_{J_2}(v) \equiv_2 0.$$
Thus
$$d_{J_3}(v) \equiv_2 d_S(v) + d_{J^*}(v).$$
Since $J^*$ is a $T$-Join we have
$$d_{J^*}(v) \equiv_2 \begin{cases}
1, &\text{if } v \in V\backslash T \\
0, &\text{if } v \in T.
\end{cases}$$
By our choice of $T'$, $v$ is either in $A\subseteq T$ or $B\subseteq V\backslash T$. So the case breakdown for $d_{J^*}(v)$ is more precisely:
 $$d_{J^*}(v) \equiv_2 \begin{cases}
1, &\text{if } v \in B \\
0, &\text{if } v \in A.
\end{cases}$$
By our choice $A$ and $B$ we have
$$d_{S}(v) \equiv_2 \begin{cases}
1, &\text{if } v \in B \\
0, &\text{if } v \in A.
\end{cases}$$
Now observe that in either case ($v \in A$ or $v\in B$)
$$d_{J_3}(v) = d_S(v) + d_{J^*}(v) = 0.$$
Thus $J_3$ is a $T'$-Join as claimed.
\paragraph{}
Since $J_1$, $J_2$, and $J_3$ are $T'$-Joins, and $J$ is a minimum cost $T'$-Join,
$$3c(J) \leq c(J_1) + c(J_2) + c(J_3).$$
By our choices of $J_1$, $J_2$, and $J_3$ their union of edges is a subset of $S \cup J^*$. So we have
$$c(J_1) + c(J_2) + c(J_3) \leq c(S) + c(J^*) \leq 2c(J^*).$$
Combining inequalities we see that
$$3c(J) \leq 2c(J^*).$$
Therefore 
$$c(S) + c(J) \leq c(J^*) + \frac{2}{3}c(J^*) = \frac{5}{3}c(J^*)$$
and so the approximation factor holds as desired.
\end{proof}
%Question 2
\section{}
\paragraph{}
Let $M$ be a set of machines and let $J$ be a set of jobs. For each $j \in J$ let $M_j \subseteq M$ denote the subset of machines $j$ can be processed on. Further for each $j \in J$, let $p_j \in \N$ denote the processing time of job $j$ on any machine in $M_j$. We consider the scheduling problem, which we will denote $(SP)$, that asks us to find a schedule minimizing the makespan: the maximum completion time of any machine provided we process all jobs. We will formalize this objective momentarily.
\paragraph{}
The order we process jobs on a machine is irrelevant in this problem, and we cannot split jobs in pieces to give to different machines. So a solution to $(SP)$ is an assignment $a: J \rightarrow M$. Let $A$ be the set of all feasible assignments. That is $A = \{ a : \forall j \in J, a(j) \in M_j\}.$ Then $(SP)$ can be stated as
$$\min_{a \in A} \max_{m \in M} \sum_{j : a(j) = m} p_j.$$
\paragraph{}
Given an instance of $(SP)$ and a tentative makespan value $s \in \N$ we can construct an instance of Unsplittable Flow, which we'll denote $(UF)$, which we can use to decide if $(SP)$ has an assignment yielding a makespan of value $s$. The vertex set will consist of a source $r$, a node $v_m$ for every $m \in M$, and a terminal node $t_j$ for every $j\in J$. The demands for each $t_j$ are the processing times $p_j$. The arc set consists of an arc $(r,v_m)$ for every $m \in M$. The capacity of each $(r,v_m)$ will be the makespan $s$. The arc set will also contain an arc $(v_m, t_j)$ for each $j \in J$ and $m \in M_j$. The capacity of each $(v_m, t_j)$ will be $p_j$. We will let $c$ denote the capacity function. Let $D$ be the resulting directed graph.
\begin{lemma}\label{lemma:sp-uf}
An instance of $(SP)$ has an assignment of makespan at most $s$ if and only if the corresponding instance of $(UF)$ has a solution. Moreover the translation between the two problems can be done in polynomial time.
\end{lemma}
\begin{proof}
\paragraph{$(\implies)$} Let $a$ be the assignment which yields a makespan of value at most $s$. For each $j \in J$ define $P_j = \{ (r,a(j)),(a(j),t_j)\}$. That is, $P_j$ is the path in $D$ (the directed graph representing $(UF)$) upon which we will route a flow of $p_j$. Each arc $(a(j), t_j)$ for $j \in J$ receives total flow $p_j$, hence respecting its capacity. Each arc $(r,m)$ for $m \in M$ receives $\sum_{j : a(j) = m}p_j$ total flow. Since $\sum_{j : a(j) = m} p_j \leq s$ the capacity of $(r,m)$ is respected. Thus we have a $(UF)$ solution.
\paragraph{$(\impliedby)$}
Let $\{P_j : j \in J\}$ be the set of paths given in the solution of $(UF)$ corresponding an instance of $(SP)$ and value $s$. We will construct an assignment $a$ for $(SP)$ that has makespan $s$. For each $j \in J$ set $a(j) = m$ if $(m,j) \in P_j$. Such an $m$ must exist, and is unique, by our construction of $D$. Since arcs incoming to $j$ are precisely those in $M_j$ this assignment is feasible. Now we consider the makespan of this assignment. Since the capacities on each arc $(r,m)$ for all $m \in M$ is respected by flow we have for all $m \in M$:
$$\sum_{j: a(j) = m} p_j = \sum_{j: (m,j) \in P_j} p_j = f(r,m) \leq c(r,m) = s.$$ 
Thus $a$ has makespan at most $s$. Clearly the translation given between the solutions of $(UF)$ and $(SP)$ can be done in polynomial time.
\end{proof}
\paragraph{}
In our algorithm we will also use the following critical lemma from class.
\begin{lemma}\label{lemma:ff-uf}
Given an instance of $(UF)$ with demands $d_1, \dots, d_k$, if $f$ is a fractional flow realizing the demands then there exists an unsplittable flow $\bar{f}$ realizing the demands such that $\bar{f}(e) \leq f(e) + d_{max}$ for all $e \in E$, where $d_{max} = \max_{i\in [k]} d_i$. Moreover this unsplittable flow can be found in polynomial time.
\end{lemma}
\paragraph{}
We give an approximation algorithm for $(SP)$. The algorithm performs a binary search to find the optimal makespan value. Let $\cA$ be the algorithm for $(SP)$ that follows:
\begin{enumerate}
\item Let $\ell = 0$ and let $u = \sum_{j \in J} p_j$.
\item While $\ell \neq u$ do:
	\begin{enumerate}
	\item Construct the corresponding $(UF)$ instance with value $s=\floor{\frac{\ell + u}{2}}$.
	\item Use a standard max flow algorithm to find a fractional flow $f$ on the $(UF)$ instance.
	\item If $f$ satisfies the demands for all $t_j$ then set $\ell = s$
	\item Otherwise set $u = s$.
	\end{enumerate}
\item Using lemma \ref{lemma:ff-uf} compute $\bar{f}$ from $f$.
\item Using lemma \ref{lemma:sp-uf} compute an assignment $a$ from $\bar{f}$ (with makespan at most $2s$). Return $a$.
\end{enumerate}
\begin{theorem}
The algorithm $\cA$ is a $2$-approximation algorithm for $(SP)$.
\end{theorem}
\begin{proof}
From lemma \ref{lemma:sp-uf} we know that $\cA$ will return a feasible solution. Each inner iteration of the While loop runs in polynomial time since we have polynomial time algorithms for max flow. Further we know from lemma \ref{lemma:ff-uf} that step $3$ can be done in polynomial time, and from lemma \ref{lemma:sp-uf} that step $4$ can be done in polynomial time. It remains to verify that the binary search procedure runs in polynomial time. For our choice of $\ell$ and $u$ the while loop will run in $O(\log(\sum_{j \in J} p_j))$ iterations. The question is whether this is polynomial in the number of bits used to represent our input.
\paragraph{}
We know each $p_j$ uses approximately $\log(p_j)$ bits to be represented in memory. Thus $\sum_{j \in J} \log(p_j)$ is a polynomial in the input size. But we may assume for the purpose of counting input size that each $p_j \geq 2$. We may do so because each $p_j$ uses at least one bit of memory to represent itself, yet $\log(1) = 0$. Now, if each $p_j \geq 2$ then $\Pi_{j\in J} p_j \geq \sum_{j \in J} p_j$. Thus $\sum_{j \in J} \log(p_j) \geq \log(\sum_{j \in J} p_j)$, and hence the binary search runs in polynomial time.
\paragraph{}
We have verified that $\cA$ runs in polynomial time returning a feasible solution. It remains to verify the approximation factor of $2$. Let $s^*$ be the makespan value of an optimal solution to $(SP)$. The instance of $(UF)$ constructed in the last iteration of the binary search has the maximal $s$ such that a fractional flow is present on $(UF)$ satisfying all the demands. Since the existence of an unsplittable flow implies the existence of a fractional flow satisfying the demands, by lemma \ref{lemma:sp-uf} using $s^*$, $$s \leq s^*.$$
\paragraph{}
By lemma \ref{lemma:ff-uf}, $\cA$ returns a solution of makespan $2s$. This follows since the maximum capacity of an arc in our instance of $(UF)$ is $s$. Hence we return a solution of value
$$2s \leq 2s^*.$$
Therefore the approximation factor holds.
\end{proof}
%Question 3
\section{}
\paragraph{}
In an instance of $(WTAP)$ we are given a tree $G=(V,E)$ and a set $L \subseteq V\times V$, called ``links". We are also given a cost function $c: L \rightarrow \R_+$. The goal is to find a minimum cost subset $\bar{L} \subseteq L$ such that $(V, E \cup \bar{L})$ is $2$-edge-connected.
%Part (a) of Q3
\subsection{a}
We are given the following algorithm for $(WTAP)$, as usual we will denote it $\cA$:
\begin{enumerate}
\item Select an arbitrary $r \in V$ as the root. Orient $E$ towards $r$, and set the $w(a) = 0$ for all the resulting arcs $a$.
\item For each $uv \in L$ do:
	\begin{enumerate}
	\item Let $p$ be the lowest-common-ancestor of $u$ and $v$ on $G$. Add to $G$ the arcs $(p,u)$ and $(p,v)$ with $w(p,u) = w(p,v) = c(uv)$. Note that if $p=u$ or $p=v$ we only add one arc). We say the arcs $(p,u)$ and $(p,v)$ ``correspond" to the link $uv$.
	\end{enumerate}
\item Compute a minimum weight $out$-arborescence $R$ rooted at $r$ in the resulting digraph from steps $1$ and $2$.
\item Output the set $\bar{L} \subseteq L$ such that $\ell \in \bar{L}$ iff there exists an arc in $R$ that corresponds to $\ell$.
\end{enumerate}

\begin{lemma}\label{lemma:3a-poly}
The algorithm $\cA$ returns a feasible solution for $(WTAP)$ on $G$ with links $L$ in polynomial time.
\end{lemma}
\begin{proof}
First we check that the algorithm terminates in polynomial time. Step $1$ can be done in $O(|E|)$ time. Step $2$ uses $O(|L|)$ iterations, and each iteration can use a depth-first-search to find the lowest-common ancestor in polynomial time. Hence Step $2$ runs in polynomial time. We have seen an exact polynomial time algorithm in class for computing a minimum weight arborescence. Hence Step $3$ runs in polynomial time. Step $4$ only necessitates iterating over links of $L$ and arcs of $R$ and thus runs in polynomial time. Hence $\cA$ terminates in polynomial time.
\paragraph{}
Consider the solution $\bar{L}$ returned by the algorithm. Let $up \in E$ where $p$ is the parent of $u$ with respect to the orientation in Step $1$. Let $G_u$ be the connected component of $G-up$ which contains $u$ and let $G_p$ be the connected component of $G-up$ which contains $p$. To verify that $(V, E \cup \bar{L})$ is $2$-edge-connected we will show that there exists $\ell \in \bar{L}$ with one endpoint in $V(G_u)$ and one in $V(G_p)$.
\paragraph{}
Consider the directed path $P$ from $r$ to $u$ in the arborescence $R$ computed in Step $3$. Let $g$ be the last ancestor of $u$ on $P$ before reaching $u$. Then $(g,c) \in P$ where $c$ is either a child of $u$ or $c=u$. By our construction, since $(g,c)$ travels from a vertex closer to the root to a vertex farther from the root, $(g,c)$ corresponds to some link $\ell \in \bar{L}$. The endpoint $c$ of $\ell$ lies $G_u$. The other endpoint, say $v$, has their lowest common ancestor with $c$ being $g$. Thus $v \not \in V(G_u)$ and therefore $v \in V(G_p)$. Therefore $\ell$ is a link connected $G_u$ to $G_p$, and thus $(V, E \cup \bar{L})$ is $2$-edge-connected.
\end{proof}
\begin{theorem}
The algorithm $\cA$ is $2$-approximation algorithm for $(WTAP)$.
\end{theorem}
\begin{proof}
By lemma \ref{lemma:3a-poly} it remains to verify the approximation factor. Let $L^* \subseteq L$ be an optimal solution to $(WTAP)$ on $G$. Let $D$ denote the directed graph which results from the construction in Steps $1$ and $2$ of $\cA$. We will construct from $L^*$ an arc set $R^*$ of cost
$$c(R^*)\leq 2c(L^*)$$
which contains at $out$-arborescence rooted at $r$ on $D$.
Then, since $R$ is a minimum weight $out$-arborescence on $G$ with $c(\bar{L}) = c(R)$ this will suffice to prove the approximation factor as $c(R) \leq c(R^*) \leq 2c(L^*)$.
\paragraph{}
For each $\ell \in L^*$ add the arcs which correspond to $\ell$ to $R^*$. Now we have $c(L^*) = 2c(R^*)$. It remains to fix connectivity in $R^*$ so that there is a directed path from $r$ to each vertex. We will fix it by adding $0$ cost arcs. For each $u \in V\backslash \{r\}$ do the following. Let $p$ be the parent of $u$ with respect to the orientation of step $1$.  Let $G_u$ be the connected component of $G-up$ which contains $u$ and let $G_p$ be the connected component of $G-up$ which contains $p$. Since $L^*$ is feasible for $(WTAP)$ there exists some $ab \in L^*$ such that $a \in V(G_u)$ and $b \in V(G_p)$. Now $a$ is a descendant of $u$. Thus there exists a $0$-cost directed path from $a$ to $u$. Add this path to $R^*$. Note this does not change $c(R^*)$.
\paragraph{}
We claim that after doing this for each vertex, the connectivity requirement is satisfied. Let $u \in V$. We proceed by induction on the distance from $u$ to the root. In the base case $r=u$ and so there trivially is a path from $r$ to $u$ in $R^*$. Now suppose $u$ is at least at distance $1$ from $r$, and for all ancestors $u$ there a directed path from $r$ to $u$'s ancestor using arcs of $R^*$. Let $p$, $a$, and $b$ be as in the previous paragraph with respect to $u$. Since $a$ is in $G_u$ and $b$ is in $G_p$ the lowest common ancestor of $a$ and $b$ is some vertex $g$ that is an ancestor of $u$. By induction there is a directed path $P_{rg}$ from $r$ to $g$ using arcs of $R^*$. By construction the arc $(g,a)$ is in $R^*$. So $P_{rg}$, $(g,a)$, and the $0$-cost path from $a$ to $u$ added to $R^*$ in the previous paragraph forms a directed path from $r$ to $u$ in $R^*$. Hence the claim holds. 
\paragraph{}
So $R^*$ contains a directed path from $r$ to every vertex of $D$. Therefore $R^*$ contains an $out$-arborescence, and as argued in the first paragraph this suffices to prove the approximation factor.
\end{proof}
%Part b of Q3
\subsection{b}
\paragraph{}
Now we change $(WTAP)$ so that the requirement is $(V, E\cup \bar{L})$ is $2$-vertex-connected. We may assume we are not in the degenerate case where we have a graph consisting of only two vertices connected by an edge. Such a tree is $2$-vertex-connected by default. We will modify the algorithm $\cA$ as follows, the big changes are that we take a leaf to be the root, and we ``shorten" the downward arcs, and we add ``cross" arcs:
\begin{enumerate}
\item Select an arbitrary leaf $r \in V$ as the root. Orient $E$ towards $r$, and set the $w(a) = 0$ for all the resulting arcs $a$.
\item For each $uv \in L$ do:
	\begin{enumerate}
	\item Let $p$ be the lowest-common-ancestor of $u$ and $v$ on $G$. Add to $G$ the arcs $(p,u)$ and $(p,v)$ with $w(p,u) = w(p,v) = c(uv)$. Note that if $p=u$ or $p=v$ we only add one arc). We say the arcs $(p,u)$ and $(p,v)$ ``correspond" to the link $uv$.
	\item If neither $u$ nor $v$ are each others' ancestor then  add to $G$ the arcs $(u,v)$ and $(v,u)$. Set $w(u,v) = w(v,u) = c(uv)$. We also say the arcs $(u,v)$ and $(v,u)$ ``correspond to the link $uv$. Note that at most four arcs are generated between this step and the previous step for each link.
	\item For each downward arc ($p$ is an ancestor of $u$) $(p,u)$ do: if $p$ is not a leaf let $c_p$ be the child of $p$  that succeeds $p$ on the path to $u$ (otherwise let $c_p = p$). Replace the arc $(p,u)$ with the arc $(c_p, u)$ with the same weight, and same corresponding link as $(p,u)$.
	\end{enumerate}
\item Compute a minimum weight $out$-arborescence $R$ rooted at $r$ in the resulting digraph from steps $1$ and $2$.
\item Output the set $\bar{L} \subseteq L$ such that $\ell \in \bar{L}$ iff there exists an arc in $R$ that corresponds to $\ell$.
\end{enumerate}
\begin{lemma}\label{lemma:3b-poly}
The algorithm $\cA$ returns a feasible solution for $(WTAP)$ on $G$ with links $L$ in polynomial time.
\end{lemma}
\begin{proof}
As argued in lemma \ref{lemma:3a-poly} the algorithm $\cA$ runs in polynomial time, it is clear that the additional processing runs in polynomial time. To verify feasibility suppose for a contradiction that the graph $H := (V, E \cup \bar{L})$ has a vertex $p$ whose deletion disconnects the graph. Let $g$ be the parent of $p$. The vertex $g$ exists since we choose $r$ to be a leaf and hence $p \neq r$. Let $u_1, \dots, u_k$ be the children of $p$. For each $v \in V$, let $C(v,p)$ denote the connected component of $H-p$ containing vertex $v$. Let $A := \{u \in \{u_1, \dots, u_k\} : u \in C(g,p)\}$ be the set of children of $p$ which are connected to $g$ after removing $p$. We claim that $A \neq \emptyset$.
\paragraph{}
Clearly $r \in C(g,p)$, as are all ancestors of $p$. Now since $R$ is an $out$-arborescence there exists a directed path from $r$ to $\bigcup_{i=1}^k u_i$. The arc such a path uses to enter $\bigcup_{i=1}^k u_i$ is a downward arc, from either $p$ or  some ancestor of $p$, $a$, to a descendant of $p$, $d$. Let $u$ be a child of $p$ such that $d \in C(u,p)$. If the arc is of the form $(a,d)$ then the corresponding link  in $\bar{L}$ connects a vertex of $C(g,p)$ to some descendant of a child $u$. Hence $C(g,p)$ is connected to $C(u)$. That is $u \in A$ and hence $A \neq \emptyset$. In the case the arc is of the form $(p,d)$, then by step $2(c)$, the arc $(p,d)$ arose from a downward arc $(g,d)$. Thus there is a link in $\bar{L}$ connecting a vertex of $C(g,p)$ to $d$. Thus $C(g)$ is connected to $C(u)$. Therefore our claim that $A$ is not empty holds.
\paragraph{}
 Let $\bar{A} = \{u_1, \dots, u_k\} \backslash A$. By our contradiction assumption, $\bar{A}\neq \emptyset$ as otherwise removing $p$ does not disconnect $H$. Now, again since $R$ is an $out$-arborescence there exists a directed path from $r$ to $\bigcup_{u \in \bar{A}} C(u)$. The directed arc which enters $\bigcup_{u \in \bar{A}} C(u)$ is either a downward arc or a cross arc arising from step $2(b)$. In the case it is a downward arc, by the same argument as the previous paragraph the corresponding link in $\bar{L}$ connects some $C(v, p)$ for $v \in \bar{A}$ to some $C(s,p)$ for $s \in A \cup \{g\}$. But this contradicts the definition of $\bar{A}$. In the case of a cross arc, the head is necessarily in $C(g,p)$ and the tail in some $C(v,p)$ for $v \in \bar{A}$. Thus, as in the previous case, the corresponding link in $\bar{L}$ connects some $C(v, p)$ for $v \in \bar{A}$ to some $C(s,p)$ for $s \in A \cup \{g\}$. In either case a contradiction is reached since the corresponding components to vertices in $\bar{A}$ cannot be connected to $C(g,p)$ by definition. Therefore no vertex exists whose deletion disconnects the graph.
\end{proof}
\begin{theorem}
The algorithm $\cA$ is $2$-approximation algorithm for $(WTAP)$.
\end{theorem}
\begin{proof}
By lemma \ref{lemma:3b-poly} it remains to verify the approximation factor. Let $L^* \subseteq L$ be an optimal solution to $(WTAP)$ on $G$. Let $D$ denote the directed graph which results from the construction in Steps $1$ and $2$ of $\cA$. We will construct from $L^*$ an arc set $A^*$ of cost
$$c(A^*)\leq 4c(L^*)$$
which contains at $out$-arborescence $R^*$ rooted at $r$ on $D$ with cost $c(R^*) \leq 2c(L^*)$.
Then, since $R$ is a minimum weight $out$-arborescence on $G$ with $c(\bar{L}) = c(R)$ this will suffice to prove the approximation factor as $c(R) \leq c(R^*) \leq 2c(L^*)$.
\paragraph{}
For each $\ell \in L^*$ add the arcs which correspond to $\ell$ (with respect to Step $2$) to $A^*$. At most four arcs are added so we have $c(A^*) \leq 4c(L^*)$. Suppose we also add to $A^*$ the $0$-cost arcs formed in Step $1$ of $\cA$. We want to show that every vertex is reachable from $r$ by a directed path in $A^*$. Then we can compute an arborescence on $A^*$ and use that to bound the optimal solution.
\paragraph{}
We claim that for every $r' \in V$, $r'$ can reach, by a directed path using arcs of $A^*$, every vertex in the subtree rooted at $r'$. We proceed by induction on the number of vertices of the subtree rooted at $r'$. In the base case the subtree is of size $1$, and hence $r'$ is a leaf and the claim is immediate. Suppose the claim holds for all subtrees with less vertices that the subtree rooted at $r'$. Let $d$ be a descendant of $r'$. We have two cases: either $r' = r$ or not. In the case $r'\neq r$, the deletion of $r'$ disconnects $G$. Let $u_1, \dots, u_k$ be the children of $r'$. Since $r'$ is not the root $r$, $r'$ has some parent $p$. Let $C_{u_1}, \dots, C_{u_k}, C_p$ denote the respective connected components of $G$ after deleting $r'$. Suppose without loss of generality that $d \in C_{u_1}$. Since $(V, E \cup L^*)$ is $2$-vertex-connected, if we think of $C_{u_1}, \dots, C_{u_k}, C_p$ as vertices, there is a subset of links $T \subseteq L^*$ which forms a spanning tree on $C_{u_1}, \dots, C_{u_k}, C_p$. There exists a path from $C_{u_1}$ to $C_g$  using links of $T$. We may assume without loss generality that this path is of the form $C_{u_1}, C_{u_2}, \dots, C_{u_k}, C_p$ since we may simply ignore connected components which do not join this path (decreasing $k$) and relabelling the children.  
\paragraph{}
Now, for $i \in [k-1]$ the link between $C_{u_{i}}$ and $C_{u_{i+1}}$ induces a directed arc in $A^*$ from $C_{u_{i+1}}$ to $C_{u_i}$ by Step $2(b)$. Call this arc $t_i$, The link between $C_{u_k}$ and $C_p$ induces a directed arc from an ancestor of $u_k$ to a vertex of $C_{u_k}$ by Step $2(a)$ and $2(c)$. Call this arc $t_k$. Now we describe the directed path to $d$. From $r'$ take the up-arcs of $0$-cost to the head of $t_k$ (note the head of $t_k$ may be $r'$, in which case take an empty path). Then follow $t_k$ to a vertex of $C_{u_k}$. We use the following process to traverse $C_k$: Use the $0$-cost arcs to reach $u_k$. Use the inductive hypothesis to reach the head of $t_{k-1}$, and follow $t_{k-1}$ to a vertex of $C_{k-1}$. Repeat this process for each $C_i$ from $C_{k-1}$ to $C_1$. After doing so the directed path has reached the tail of $t_1$. Again follow the $0$-cost arcs up to $u_1$, and use induction to find a directed path from $u_1$ to $d$. All together this process yields a directed path from $r'$ to $d$.
\paragraph{}
In the induction step the case where $r' =r$ is simpler. Let $u$ be the child of $r'$. Then $d$ is a descendant of $u$ or equal to $u$. In either case, since $L^*$ is feasible there exists some link from $r$ to a descendant of $u$. This link induces an arc from $r$ to a descendant $d'$ of $u$ in $A^*$. Since $r$ is a leaf, the head of this arc is not-moved from $r$ in step $2(c)$. To reach $d$ from $r$, simply traverse the arc from $r$ to $d'$ then traverse the $0$-cost path from $d'$ up to $u$ and use the inductive hypothesis to find a directed path from $u$ to $d$. The resulting concatenated directed path connects $r$ to $d$. Hence by induction, the root $r$ is connected to every vertex of $D$ by a directed path using only arcs of $A^*$.
\paragraph{}
Therefore we can form a arborescence on $D$, say $R^*$ which is a subset of $A^*$. Choose the minimum cost such $R^*$. We claim that $R^*$ uses at most two arcs corresponding to any given link $uv \in L^*$. Suppose for a contradiction $R^*$ uses at least three arcs corresponding to $uv$. If $R^*$ uses both $(u,v)$ and $(v,u)$ then those two arcs form a directed cycle in $R^*$, contradicting that $R^*$ is acyclic. Hence $R^*$ uses one of $(u,v)$ or $(v,u)$, as well as the two downward arcs formed. Say without loss of generality $R^*$ uses $(u,v)$. Let $p$ be the lowest common ancestor of $u$ and $v$. Let $p_u$ and $p_v$ be the respective children of $p$ on the path to $u$ and $v$ respectively. So $R^*$ uses the arcs $(p_u,u)$ and $(p_v,v)$ as well as $(u,v)$ since $R^*$ uses at least three corresponding arcs. But then $R^*$ has two directed paths from $r$ to $v$. The first from $r$ to $p_u$, and then along $(p_u,u)$ and $(u,v)$ to reach $v$. The second along the directed path from $r$ to $p_v$ then along $(p_v, v)$ to reach $v$. So arc $(u,v)$ is not needed for connectivity in $R^*$, contradicting the optimality of $R^*$.
\paragraph{}
Therefore we have formed an arborescence $R^*$ such that $c(R^*) \leq 2c(L^*)$. Since $R$ is a minimum cost arborescence and $c(\bar{L}) = c(R)$ we have
$$c(\bar{L}) = c(R) \leq c(R^*) \leq 2c(L^*).$$
So the approximation factor holds as desired.
\end{proof}
%Question 4
\section{}
\paragraph{}
We are given a bipartite graph $G=(A\cup B, E)$ where $A$ denotes the set of ``men" and $B$ denotes the set of ``women". Each vertex gives a preference order over their neighbouring vertices. The preference orders are strict for men, but may contain ties for women. We let $>_u$ denote the preference order for vertex $u$. That is, we say $w >_u v $ if $u$ prefers neighbour $w$ to neighbour $v$. We will use $\geq_u$, $=_u$, $<_u$, and $\leq_u$ in a similar manner. Let $N(u)$ denote the neighbourhood of vertex $u$. Let the first choice of vertex $u$ with be denoted $f(u)$. That is $f(u) \geq_u w$ for all $w \in N(u)$. Let $(SM)$ denote the problem of finding a maximum cardinality stable matching in $G$.
\paragraph{}
We assume that any vertex prefers to be matched to unmatched. Given a matching $M$, we let $M(u) := v$ iff $uv \in M$. If $u$ is unmatched in $M$ we let $M(u)$ denote some empty element which indicates that $u$ is unmatched. Let $V(M):= \{ u \in V(G): \exists v \in V(G), uv \in M\}$ be the set of matched vertices. We are given the following algorithm, which we will denote $\cA$:
\begin{enumerate}
\item Let $M = \emptyset$. Let $G' = G$. No $m \in A$ is a bachelor to start.
\item While there exists an unmatched $m \in A$ with some women on his list do:
	\begin{enumerate}
	\item $m$ proposes to $w = f(m)$ with respect to $G'$. Let $m' = M(w)$.
	\item If $m >_w m'$ then set $M = M\backslash \{m'w\} \cup \{mw\}$. Set $G' = G' - m'w$ (remove $w$ from $m'$'s list).
	\item If $m =_w m'$ and $m$ is a bachelor then set $M = M\backslash \{m'w\} \cup \{mw\}$. Set $G' = G'-m'w$.
	\item Else (both previous conditions fail) $w$ rejects $m$ and set $G' = G' -mw$ (remove $w$ from $m$'s list). 
	\item If $N(m) = \emptyset$ for the first time, label $m$ a bachelor and add all edges in $\delta(m)$ with respect to $G$ back to $G'$.
	\end{enumerate}
\item Return $M$.
\end{enumerate}
\begin{lemma}\label{lemma:4poly}
The algorithm $\cA$ returns a stable matching in polynomial time.
\end{lemma}
\begin{proof}
The algorithm iterates over each man's list at most twice, and each iteration takes polynomial time, so $\cA$ runs in polynomial time. So it remains to verify that $\cA$ returns a stable matching. It is easy to see that $M$ is a matching since each vertex has at most one partner at any time in the operation of $\cA$. Suppose for a contradiction there exists some blocking pair $ab$ with $a \in A$ and $b \in B$. Since $b >_a M(a)$, whenever $a$ proposes to $b$, $a$ is rejected. Thus during the iterations $a$ proposes to $b$, $M(b) \geq_b a$. Since $b$ never releases a man for one she likes less, the quality of $b$'s partner from her perspective is non-decreasing each iteration. So since $M(b) \geq_b a$ when $a$ proposed to $b$ for the last time, this also holds at the end of $\cA$. But this contradicts $a>_b M(a)$ which comes from $ab$ being a blocking pair. Hence the matching $M$ returned is stable. 
\end{proof}
\begin{theorem}
The algorithm $\cA$ is a $\frac{3}{2}$-approximation algorithm for $(SM)$.
\end{theorem}
\begin{proof}
By lemma \ref{lemma:4poly} it only remains to show the approximation factor holds. Let $M^*$ be an optimal stable matching solution to $(SM)$. Let $H = (V, M \Delta M^*)$. By standard matching theory, $H$ is a composition of paths and cycles. If we can show there are no $M$-exposing paths of lengths at most $3$ then the approximation ratio holds. This follows since we would have $$\frac{|M|}{|M^*} \leq \frac{|P \cap M|}{|P\cap M^*|} \leq \frac{2}{3},$$
where $P$ is the shortest $M$-augmenting path in $H$. The second inequality follows since $M$-augmenting paths are of odd length in $H$, and so $|P| \geq 5$.
\paragraph{}
To complete the proof we show there are no $M$-augmenting paths of length at most $3$. An $M$-augmenting path of length $1$ would be a blocking pair against $M$, so by stability such a path does not exist. So suppose for a contradiction that there is an $M$-augmenting path of length $3$. Say the path is $a_1b_1, b_1a_2,a_2b_2$ where $a_1b_1,a_2b_2 \in M^*$ and $b_1a_2 \in M$ (with $a_1,a_2 \in A$ and $b_1,b_2 \in B$). Since $M^*$ is stable $a_1 \geq_{b_1} a_2$ or $b_2 >_{a_2} b_1$.
\paragraph{}
If $b_2 >_{a_2} b_1$ then $a_2b_2$ blocks $M$, since $b_2$ is unmatched in $M$ and prefers to be matched. Similarly if $a_1 >_{b_1} a_2$ then $a_1b_1$ blocks $M$. So $a_1 =_{b_1} a_2$, and $b_1 >_{a_2} b_2$. We claim that $a_2$ proposes to $b_1$ at most once during $\cA$. Once a woman becomes matched in $\cA$ she never becomes unmatched. Thus since $b_2$ is unmatched at the end of $\cA$ and $b_2$ is acceptable to $a_2$, $a_2$ never ran through his whole proposal list during $\cA$. Therefore $a_2$ is accepted by $b_1$ in their first proposal round. Since we know women never free partners for someone worse during $\cA$ whoever $b_1$ is paired with at any stage of the algorithm she likes at most as much as $a_1$ and $a_2$. Since $b_1$ would reject $a_2$ otherwise, this implies that $b_1$ is matched to $a_2$ during the first stage of the algorithm, and in particular during $a_1$'s first proposal. Since $a_1$ is unmatched at the end of $\cA$, we know $a_1$ proposed to $b_1$ twice, and she rejected him twice. But during the second round $a_1$ became a bachelor such that $a_1 =_{b_1} a_2$, and so $\cA$ would match $a_1b_1$ during $a_1$'s second proposal to $b_1$. But this is a contradiction. Therefore $M$-augmenting paths of length at most three do not exist.
\end{proof}
\end{document}
