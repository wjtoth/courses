\documentclass[11pt]{amsart}
\usepackage{amsmath,amsthm,amsfonts,amssymb}
\usepackage{fullpage}
\usepackage[usenames]{xcolor}

\theoremstyle{plain}
\newtheorem{theorem}{Theorem}
\newtheorem{lemma}{Lemma}
\newtheorem{corollary}{Corollary}
\newtheorem{proposition}{Proposition}
\theoremstyle{definition}
\newtheorem{definition}{Definition}
\newtheorem{exercise}{Exercise}
\newtheorem{remark}{Remark}
\newtheorem{problem}{Problem}
\newtheorem{challenge}{Challenge Problem}
\newtheorem{open}{Open Problem}
\theoremstyle{plain}

\newcommand{\argmin}{\mathrm{argmin}}
\newcommand{\calX}{\mathcal{X}}
\newcommand{\calY}{\mathcal{Y}}
\newcommand{\calZ}{\mathcal{Z}}
\newcommand{\cost}{\mathrm{cost}}
\newcommand{\dHam}{\mathrm{d}_{\mathrm{Ham}}}
\newcommand{\disc}{\mathrm{disc}}
\newcommand{\discu}{\mathrm{disc_u}}
\newcommand{\Disj}{\textsc{Disj}}
\newcommand{\Eq}{\textsc{Eq}}
\newcommand{\GHD}{\textsc{GHD}}
\newcommand{\GIP}{\textsc{GIP}}
\newcommand{\GT}{\textsc{GT}}
\newcommand{\HD}{\textsc{HD}}
\newcommand{\ICext}{\mathrm{IC}^{\mathrm{ext}}}
\newcommand{\ICint}{\mathrm{IC}^{\mathrm{int}}}
\newcommand{\icostext}{\mathrm{icost}^{\mathrm{ext}}}
\newcommand{\icostint}{\mathrm{icost}^{\mathrm{int}}}
\newcommand{\IP}{\textsc{IP}}
\newcommand{\Index}{\textsc{Index}}
\newcommand{\Maj}{\textsc{Maj}}
\newcommand{\MIP}{\textsc{MIP}}
\newcommand{\R}{\mathbb{R}}
\newcommand{\rank}{\mathrm{rank}}
\newcommand{\Rpriv}{R^{\mathrm{priv}}}
\newcommand{\SumIndex}{\textsc{SumIndex}}

\newcommand{\exercises}{\bigskip \noindent\rule{8cm}{0.4pt} \medskip}
\newcommand{\replacethistext}[1]{\textcolor{red}{#1}}




\title{Communication complexity: Chapter 7 \\ Multiparty communication}
\author{Eric Blais and (YOUR NAME HERE)}


\begin{document}

\maketitle

In this chapter, we move beyond the two-party setting of communication complexity and consider the communication complexity of functions that need to be computed by 3 or more parties.

There are two broad classes of models for multiparty communication complexity. In both of them, the input to the function is divided among the $k$ parties. In the \emph{number in hand} model, each party sees their own input and none of the other players' inputs. This model is a natural extension of our two-party communication model, and it is apparent that it only gets harder when there are more players involved. As a result, all the lower bounds that we have established in previous chapters apply to (natural variants of the functions we have seen for) this setting as well.

In the \emph{number on the forehead} model, by contrast, every player sees \emph{all} the inputs except their own. In this model, adding new players now make the function easier to compute, not harder. The lower bounds we have developed no longer apply, and we need to develop new tools to understand which functions remain hard to compute in this setting.


\newpage 
\section{Number-on-the-forehead model}

In the \emph{$k$-party number on the forehead} (NOF) model of communication complexity, $k$ players $P_1,\ldots,P_k$ aim to compute a function $f : \calX^k \to \calY$ for some finite sets $\calX$ and $\calY$. On input $x_1,\ldots,x_k$, each player $P_i$ sees \emph{all} the inputs \emph{except} $x_i$. A \emph{protocol} determines which player sends the next bit of communication; the bit sent by player $P_i$ depends on the inputs $x_1,\ldots,x_{i-1},x_{i+1},\ldots,x_k$ as well as the preivous messages. All messages communicated by a player are seen by \emph{all} other players. This is known as the \emph{blackboard} setting of communication.

The \emph{cost} of a protocol is the maximum number of bits communicated before the protocol halts, where the maximum is taken over all possible inputs. A protocol \emph{computes} the function $f$ if all the players know $f(x)$ at the end of the protocol. As in the two-player setting, we can represent a protocol as a tree and define the function that it computes as the one determined by the value at each leaf.

The \emph{deterministic communication complexity} of a function $f : \calX^k \to \calY$, which will again be denoted by $D(f)$, is the minimum cost of a $k$-party number-on-the-forehead communication protocol that computes $f$ in the blackboard model.

\begin{theorem}
Almost all functions $f : (\{0,1\}^{n})^k \to \{0,1\}$ have communication complexity $D(f) = \Omega(n)$.
\end{theorem}

\begin{proof}
The number of possible functions $f$ is counted by the number of binary strings of length $|(\{0,1\}^n)^k|$. Hence the number of possible functions $f$ is 
$$2^{2^{kn}}.$$
For a protocol tree of height $h$, there are at most $2^{h+1}$ nodes in the protocol tree. Each node belongs to one of $k$ players, and is associated with a function whose output is binary and whose input domain is the space of all other players inputs. There are 
$$2^{2^{(k-1)n}}$$ 
such functions. Thus in total there are at most
$$(k\cdot 2^{2^{(k-1)n}})^{2^{h+1}} = k^{2^{h+1}}2^{2^{(k-1)n + h + 1}}$$
possible protocol using $h$ bits of communication. Since we require protocols to be deterministic, the ratio of functions computable with $h$ bits of the total possible functions is 
$$ 2^{2^{h+1}\log(k) - 2^{kn}}2^{2^{(k-1)n +h + 1 -kn}}.$$
If $h = o(n)$ the limit of this ratio as $n$ approaches infinity is $0$. Thus the theorem holds.
\end{proof}

\exercises

\begin{exercise}
Show that every function $f : (\{0,1\}^{n})^k \to \{0,1\}$ has multiparty communication complexity $D(f) \le n+1$.
\end{exercise}

\begin{open}
Identify any explicit function $f : (\{0,1\}^{n})^k \to \{0,1\}$ with $k \ge \log n$ that has communication complexity $\omega(1)$.
\end{open}


\newpage 
\section{Equality XI}

Some functions can be much easier in the multiparty communication complexity model. Consider for instance the extension of the equality function for $k$ players $\Eq_k : (\{0,1\}^n)^k \to \{0,1\}$ defined by
\[
\Eq^k(x_1,\ldots,x_k) = \begin{cases}
1 & \mbox{if } x_1 = x_2 = \cdots = x_k \\
0 & \mbox{otherwise.}
\end{cases}
\]
\begin{theorem}
For any $k \ge 3$, $D(\Eq^k) \le 2$.
\end{theorem}

\begin{proof}
Consider the protocol where player $1$ outputs $1$ if $x_2 = x_3 = \dots x_k$ and outputs $0$ otherwise. The protocol terminates with output $0$ in the latter case. In the former, player $2$ outputs $1$ if $x_1 = x_3$ and outputs $0$ otherwise, and these outputs correspond to the decision of the protocol in their respective cases.
\end{proof}



\newpage 
\section{Majority Inner Product}

Other functions also have very efficient protocols in the multiparty communication complexity setting. 

For three bits $a,b,c$, define the \emph{majority} function $\Maj : \{0,1\}^3 \to \{0,1\}$ to be $\Maj(a,b,c) = 1$ if $a+b+c \ge 2$ and $0$ otherwise. The \emph{Majority inner product} function $\MIP : (\{0,1\}^n)^3 \to \{0,1\}$ is defined by
\[
\MIP(x,y,z) = \bigoplus_{i \in [n]} \Maj(x_i,y_i,z_i).
\]
\begin{theorem}
$D(\MIP) \le 3$.
\end{theorem}

\begin{proof}
We use the following identity which can be verified by case analysis
$$\Maj(x_i,y_i,z_i) = x_iy_i + y_iz_i + x_iz_i \mod 2.$$
Using this identity we see that
$$\MIP(x,y,z) = \bigoplus_{i\in[n]}\Maj(x_i,y_i,z_i) = \left(\bigoplus_{i\in [n]} x_iy_i\right)\oplus\left(\bigoplus_{i\in [n]} y_iz_i\right)\oplus\left(\bigoplus_{i\in [n]} x_iz_i\right).$$
On the right hand side, the first term can be computed by player $3$ and output, the second by player $1$, and the third by player $2$. With all $3$ bits anyone can then compute $\MIP(x,y,z)$.
\end{proof}


\newpage 
\section{Generalized Inner Product}

Other functions become easier, but non-trivial, when the number of players is greater than 2 but less than $\log n$.

The \emph{generalized inner product} function $\GIP^k : (\{0,1\}^n)^k \to \{0,1\}$ is defined by
\[
\GIP^k(x^{(1)},\ldots,x^{(k)}) = \bigoplus_{i \in [n]} x^{(1)}_i \wedge x^{(2)}_i \wedge \cdots \wedge x^{(k)}_i.
\]
When we view the input to the function as a $k \times n$ matrix, the function takes the value $1$ if and only if the number of all-one columns of this matrix is odd. 

\begin{theorem}
$D(\GIP^k) = O(\frac{nk}{2^k})$.
\end{theorem}

\begin{proof}
Let $M$ be the corresponding $k\times n$ matrix to the input, where each player cannot see a particular row. Our protocol is as follows.

We first partition the columns into blocks into floor of $\frac{n}{2^{k-1}-1}$ blocks. We will give a $k$ cost protocol for each block, and summing the parities of the blocks mod $2$ will yield the parity of the number of all-ones columns in $M$, within the desired communication cost bound.

Consider block $B$ of $M$. By the pigeonhole principle, since $B$ has fewer than $2^{k-1}$ columns, there is a $k-1$-bit string $s$ which does not appear in $B$ delete row $1$. Player $1$ communicates $s$ at a cost of $O(\log(n))$. By permuting the rows, we may assume that $0s$, which does not appear in the columns of $B$, is of the form $0^\ell 1^{k-\ell}$ for some $\ell$.

Let $c_i$ denote the number of columns of $B$ of the form $0^i1^i$. Observe that player $i$ can compute $c_{i-1} + c_i$, but not $c_{i-1}$ and $c_i$ since they do not know the bit in the $i$th position. Our protocol proceeds as follows: for each player $i$ from $1$ to $\ell-1$: player $i$ outputs the parity of $c_{i-1} + c_i$. Then player $\ell$ outputs the parity of  $c_{\ell - 1}$ (since we knows $c_\ell = 0$). By back-substitution, any player can now compute the parity of $c_0$, as desired. Since $\ell \leq k$, this protocol uses at most $k$ bits.
\end{proof}



\newpage 
\section{Cylinder intersections}

Rectangles have a natural generalization in the number-on-the-forehead model in the form of \emph{cylinder intersections}.

\begin{definition}
A \emph{cylinder} in $\calX^k$ \emph{in the $i$th direction} is a set $S \subseteq \calX^k$ such that whether an element $(x^{(1)},\ldots,x^{(k)}) \in S$ or not is independent of the value of $x^{(i)}$.
\end{definition}

\begin{definition}
The set $S \subseteq \calX^k$ is a \emph{cylinder intersection} if $S = \cap_{i \in [n]} S_i$ where each $S_i$ is a cylinder in the $i$th direction.
\end{definition}

Cylinder intersections can be used to bound the multiparty communication complexity of functions via the following fundamental lemma.

\begin{lemma}
Every multiparty protocol that computes $f$ and has cost $c$ partitions the domain of $f$ into at most $2^c$ monochromatic cylinder intersections.
\end{lemma}

\begin{proof}
Let $\pi$ be a protocol which computes $f$ at cost $c$. Let $\ell$ be a leaf of $\pi$. Let $S(\ell)$ be the set of inputs in $\calX^k$ for which $\pi$ outputs $\ell$. Clearly $\{S(\ell)\}_\ell$ is a partition of the domain of $f$, since $\pi$ is deterministic.

Since there are $2^c$ such $S(\ell)$, it only remains to verify that $S(\ell)$ is a cylindrical intersection. We define
$$S_i(\ell):=\{x \in \calX^k: \exists y^i s.t. x^1\dots x^{i-1}y^ix^{i+1}\dots x^k \in S(\ell)\}.$$
Since containment in $S_i(\ell)$ is independent of $x^i$, $S_i(\ell)$ is a cylinder. Thus we need to show
$$S(\ell) = \bigcap_{i\in[k]} S_i(\ell).$$
On the one hand for $x\in S(\ell)$, we have $x\in S_i(\ell)$ immediately (by taking $y^i = x^i$) for all $i$. 

On the other hand, let $x \in \bigcap_{i\in [k]} S_i(\ell).$ Fix $i\in[k]$. We have for some $y^i$ that
$$x^1\dots x^{i-1}y^ix^{i+1}\dots x^k \in S(\ell).$$
We also have for some $y^1$ that
$$y^1x^2\dots x^i\dots x^k\in S(\ell).$$
But then in the first case, player $i$ can switch their input to $x^i$ without the protocol tree reaching a different leaf. So $x \in S(\ell)$ as desired.
\end{proof}



\newpage 
\section{Discrepancy}

Having generalized the notion of rectangles to the multiparty setting, we could hope that all the lower bound techniques also generalize. That unfortunately does not appear to be the case. The only technique that generalizes to this setting is discrepancy.

\begin{definition}
For any distribution $\mu$ on $\calX^k$, 
the \emph{$\mu$-discrepancy} of $f : \calX^k \to \{0,1\}$ is
\[
\disc_\mu(f) = \max_{S} 
\left| \mu\big( S \cap f^{-1}(0) \big) - \mu\big( S \cap f^{-1}(1)\big) \right|
\]
where the maximum is over all cylinder intersections $S \subseteq \calX^k$.
\end{definition}


\begin{lemma}
For every function $f : \calX^k \to \{0,1\}$ and distribution $\mu$ on $\calX^k$,
\[
D(f) = \Omega\left( \log \frac{1}{\disc_\mu(f)}\right).
\]
\end{lemma}

\begin{proof}
Let $\pi$ be a protocol computing $f$ at cost $D(f)$. Let $\mu$ be a distribution over $\calX^k$. Since $\pi$ is deterministic,
$$1 = \Pr_{x\sim \mu}(\pi(x) = f(x)).$$
Let $t = 2^{D(f)}$ and consider a partition of the domain of $f$ into $t$ cylindrical intersections $S_1, \dots, S_t$.

Then we have
\begin{align*}
1 &=\Pr(\pi(x) = f(x))\\
&=\sum_{i=1}^t\Pr(\pi(x) = f(x)\text{ and } x\in S_i) \\
&= \sum_{i=1}^t\Pr(\pi(x) = f(x)\text{ and } x\in S_i) - \Pr(\pi(x) \neq f(x)\text{ and } x\in S_i)\\
&\leq \sum_{i=1}^t|\mu(S\cap f^{-1}(0)) - \mu(S\cap f^{-1}(1))|\\
&= t\cdot \disc{_\mu}(f).
\end{align*}
Rearraning we obtain the desired inequality.
\end{proof}

\exercises

\begin{exercise}
We can also consider \emph{randomized} multiparty communication protocols. Show that for every function $f$, every distribution $\mu$, and every $\epsilon \le \frac12$, we also have
\[
R_\epsilon(f) \ge \log \frac{1-2\epsilon}{\disc_\mu(f)}.
\]
\end{exercise}


\newpage 
\section{Generalized inner product II}

We can use the discrepancy bound to give a lower bound on the communication complexity of the generalized inner product function.

\begin{theorem}
$D(\GIP^k) = \Omega(\frac{n}{4^k})$.
\end{theorem}

\begin{proof}
\replacethistext{Enter your proof here.}
\end{proof}



\newpage 
\section{Multiparty Simultaneous Message Passing}

In the \emph{simultaneous message passing} (SMP) model of multiparty communication complexity, we again consider the number-on-the-forehead model with blackboard communication with one change: all the players now write their messages on the blackboard simultaneously. Or, equivalently, all the bits communicated by the protocol depend on the other players inputs but \emph{not} on the messages previously sent.

The minimum cost of an SMP multiparty communication protocol that computes a function $f : \calX^k \to \{0,1\}$ is denoted by $D^\parallel(f)$.

A particularly interesting function to study in the multiparty SMP model is the $k$-party generalization of the $\Index$ function where $\Index^k : \{0,1\}^{n^{k-1}} \times [n]^{k-1} \to \{0,1\}$ where player $P_1$'s input is a $(k-1)$-dimensional array of bits, the $k-1$ other players all receive an index in $[n]$, and
\[
\Index^k(A,i_2,\ldots,i_k) = A[i_2,\ldots,i_k].
\]

\begin{theorem}
$D^\parallel(\Index^k) = \Omega(n/k)$.
\end{theorem}

\begin{proof}
Let $\pi$ be a multiparty SMP protocol for $\Index^k$ with communication cost $c$. Suppose for a contradiction that $c < \frac{n}{k}$. Consider the following communication problem: Alice receives $A \in \{0,1\}^{n^{k-1}}$ and Bob receives $i_2,\dots, i_k$ and they need to compute $A[i_2,\dots, i_k]$. We will present a \emph{one-way} deterministic protocol for this problem using $\pi$ with cost less than $n^{k-1}$. Using the same argument as we did in Chapter $6$ for the one-way deterministic communication cost of $\Index$, we can show that this problem was one-way deterministic communication complexity at least $n^{{k-1}})$. Hence we will have reached a contradiction.

It remains to give the promised protocol. For each player $j \in \{2, \dots, k\}$ Alice enumerates the possible input $k-2$-tuples $t$ that $j$ would see the other players in $\{2,\dots, k\}$ holding, and for each such possiblity Alice sends to Bob the message $j$ would send in $\pi$ seeing $A$ and $t$. For each player this total communication cost is $cn^{k-2}$. Thus Alice sends strictly less than
$$ckn^{n-2} < n^{k-1}$$
bits it total.
Using the bits Alice sent, Bob can determine the messages that all players would sent in $\pi$ under input $A,i_2,\dots, i_k$, and can output the referees decision, computing $\Index^k$.
\end{proof}

\exercises

\begin{exercise}
Show that $D(\Index^k) = \Theta(\log n)$.
\end{exercise}


\newpage 
\section{Sum Index}

Another interesting variant of the index function is the 3-party function $\SumIndex : \{0,1\}^n \times [n] \times [n]$ defined by
\[
\SumIndex(A,i,j) = A[ i \oplus j]
\]
with $i \oplus j$ being the bitwise OR operation. (It is useful to consider only the case where $n$ is a power of $2$ for simplicity.)

\begin{theorem}
$D^\parallel(\SumIndex) = \Omega(\sqrt{n})$.
\end{theorem}

\begin{proof}
Suppose for a contradiction there exists a simultaneous message passing protocol $\pi$ for $\SumIndex$ using $c = o(\sqrt{n})$ bits. We may assume that players $2$ and $3$ announce their $j$ and $i$ respectively in their messages at a cost of $\log(n)$ bits. This is without loss of generality since $c + \log(n) = o(\sqrt{n})$. Now since the referee learns from players $2$ and $3$ everything that player $1$ sees, player $1$ does not need to send anything in $\pi$. So we assume player $1$'s message is empty.

\end{proof}

\exercises

\begin{exercise}
Prove that $D^\parallel(\SumIndex) = o(n)$.
\end{exercise}

\begin{open}
The best bounds on the SMP complexity of the $\SumIndex$ function are
\[
\Omega(\sqrt{n}) \le D^\parallel(\SumIndex) \le O(n^{0.73}).
\]
Can you improve either the upper or the lower bound?
\end{open}

\end{document}