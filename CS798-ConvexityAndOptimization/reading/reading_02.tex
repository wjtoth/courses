\documentclass[letterpaper,12pt,oneside,onecolumn]{article}
\usepackage[margin=1in, bottom=1in, top=1in]{geometry} %1 inch margins
\usepackage{amsmath, amssymb, amstext}
\usepackage{fancyhdr}
\usepackage{mathtools}
\usepackage{algorithm}
\usepackage{algpseudocode}
\usepackage{theorem}
\usepackage{tikz}
\usepackage{tkz-berge}

%Macros
\newcommand{\A}{\mathbb{A}} \newcommand{\C}{\mathbb{C}}
\newcommand{\D}{\mathbb{D}} \newcommand{\F}{\mathbb{F}}
\newcommand{\N}{\mathbb{N}} \newcommand{\R}{\mathbb{R}}
\newcommand{\T}{\mathbb{T}} \newcommand{\Z}{\mathbb{Z}}
\newcommand{\Q}{\mathbb{Q}}
 
 
\newcommand{\cA}{\mathcal{A}} \newcommand{\cB}{\mathcal{B}}
\newcommand{\cC}{\mathcal{C}} \newcommand{\cD}{\mathcal{D}}
\newcommand{\cE}{\mathcal{E}} \newcommand{\cF}{\mathcal{F}}
\newcommand{\cG}{\mathcal{G}} \newcommand{\cH}{\mathcal{H}}
\newcommand{\cI}{\mathcal{I}} \newcommand{\cJ}{\mathcal{J}}
\newcommand{\cK}{\mathcal{K}} \newcommand{\cL}{\mathcal{L}}
\newcommand{\cM}{\mathcal{M}} \newcommand{\cN}{\mathcal{N}}
\newcommand{\cO}{\mathcal{O}} \newcommand{\cP}{\mathcal{P}}
\newcommand{\cQ}{\mathcal{Q}} \newcommand{\cR}{\mathcal{R}}
\newcommand{\cS}{\mathcal{S}} \newcommand{\cT}{\mathcal{T}}
\newcommand{\cU}{\mathcal{U}} \newcommand{\cV}{\mathcal{V}}
\newcommand{\cW}{\mathcal{W}} \newcommand{\cX}{\mathcal{X}}
\newcommand{\cY}{\mathcal{Y}} \newcommand{\cZ}{\mathcal{Z}}

\newcommand\numberthis{\addtocounter{equation}{1}\tag{\theequation}}


\newenvironment{proof}{{\bf Proof:  }}{\hfill\rule{2mm}{2mm}}
\newenvironment{proofof}[1]{{\bf Proof of #1:  }}{\hfill\rule{2mm}{2mm}}
\newenvironment{proofofnobox}[1]{{\bf#1:  }}{}\newenvironment{example}{{\bf Example:  }}{\hfill\rule{2mm}{2mm}}

%\renewcommand{\thesection}{\lecnum.\arabic{section}}
%\renewcommand{\theequation}{\thesection.\arabic{equation}}
%\renewcommand{\thefigure}{\thesection.\arabic{figure}}

\newtheorem{fact}{Fact}[section]
\newtheorem{lemma}[fact]{Lemma}
\newtheorem{theorem}[fact]{Theorem}
\newtheorem{definition}[fact]{Definition}
\newtheorem{corollary}[fact]{Corollary}
\newtheorem{proposition}[fact]{Proposition}
\newtheorem{claim}[fact]{Claim}
\newtheorem{exercise}[fact]{Exercise}
\newtheorem{note}[fact]{Note}
\newtheorem{conjecture}[fact]{Conjecture}

\newcommand{\size}[1]{\ensuremath{\left|#1\right|}}
\newcommand{\ceil}[1]{\ensuremath{\left\lceil#1\right\rceil}}
\newcommand{\floor}[1]{\ensuremath{\left\lfloor#1\right\rfloor}}

%END MACROS
%Page style
\pagestyle{fancy}

\listfiles

\raggedbottom

\lhead{2017-01-27}
\rhead{William Justin Toth CS798-Convexity and Optimization Reading 2} %CHANGE n to ASSIGNMENT NUMBER ijk TO COURSE CODE
\renewcommand{\headrulewidth}{1pt} %heading underlined
%\renewcommand{\baselinestretch}{1.2} % 1.2 line spacing for legibility (optional)

\begin{document}
\paragraph{}
We cover Sections 5.8 and 5.9 of Convex Optimization by Boyd and Vandenberghe.
\section*{5.8 - Theorems of  Alternatives}
\subsection*{Weak Alternatives via the dual}
\paragraph{}
We study the use of Lagrangian Duality Theory to the problem of determining feasibility for a set of inequality and equalities:
$$f_i(x) \leq 0,\ \ i=1,\dots,m \quad h_i(x) = 0, \ \ i=1,\dots,p $$
We assume the domain $\cD = \cap_{i=1}^m \textbf{dom} f_i \cap \cap_{i=1}^p \textbf{dom}h_i$ is nonempty. We can think of this as a standard minimization problem with objective function $0$ whose optimal value if $0$ if the system is feasible and $\infty$ otherwise.
\paragraph{}
We may associate with our inequality system a dual function
$$g(\lambda, \nu) = \inf_{x \in \cD} (\sum_{i=1}^m \lambda_i f_i(x) + \sum_{i=1}^p \nu_i h_i(x))$$
The associated dual problem to our associated primal problem is to maximize $g(\lambda, \nu)$ subject to $\lambda \geq 0$. Since $f_0 = 0$ this dual as optimal value
$$ d^* = \begin{cases}
\infty, &\lambda \geq 0 g(\lambda,\nu) > 0 \text{ is feasible} \\
0, &\text{otherwise}.
\end{cases}
$$
Hence at most one of our primal and dual systems is feasible since the dual lower bounds the primal. The two systems are called $\textit{weak alternatives}$ since they satisfy this property. Notice the primal system need not be convex, but the dual system always is.
\subsection*{Strong Alternatives}
\paragraph{}
When the original inequality system is convex, and some type of constraint qualification holds, we can obtain that the systems are $\textit{strong alternatives}$ which means that exactly one of them always holds.
\paragraph{}
Consider the system
$$f_i(x) \leq 0,\ \ i=1,\dots,m \quad Ax=b $$
and its alternative
$$\lambda \geq 0 \quad g(\lambda, \nu) > 0$$
Provided there exists an $x \in \textbf{relint} \cD$ with $Ax = b$ and optimal value $p^*$ (the minimum slack scalar optimization problem) that , then the two systems above are strong alternatives. With these assumptions we have strong duality $p^*=d^*$. Now suppose that our primal system is infeasible. Then $p^* > 0$. Then the dual optimal point $(\lambda^*, \nu^*)$ satisfies the alternative system.
\section*{5.9 - Generalized Inequalities}
\paragraph{}
The theory above associated with Lagrangian Duality can be extended to problems with generalized inequality constraints:
\begin{align*}
\min &f_0(x) \\
\text{s.t.} f_i(x) &\preccurlyeq_{K_i} 0, &i=1,\dots,m \\
h_i(x) & = 0, &i=1,\dots,p.
\end{align*}
where $K_i$ are proper cones (that is pointed closed convex cones). We need not assume the problem is convex, and as before we assume $\cD \neq \emptyset$. The order associated with cone $K_i$, denoted $\preccurlyeq_{K_i}$ is given as $A \preccurlyeq_{K_i} B$ if and only if $B-A \in K_i$.
\end{document}
