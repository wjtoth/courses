\documentclass[letterpaper,12pt,oneside,onecolumn]{article}
\usepackage[margin=1in, bottom=1in, top=1in]{geometry} %1 inch margins
\usepackage{amsmath, amssymb, amstext}
\usepackage{fancyhdr}
\usepackage{mathtools}
\usepackage{algorithm}
\usepackage{algpseudocode}
\usepackage{theorem}
\usepackage{tikz}
\usepackage{tkz-berge}
\usepackage[braket, qm]{qcircuit}
\usepackage{hyperref}

%Macros
\newcommand{\A}{\mathbb{A}} \newcommand{\C}{\mathbb{C}}
\newcommand{\D}{\mathbb{D}} \newcommand{\F}{\mathbb{F}}
\newcommand{\N}{\mathbb{N}} \newcommand{\R}{\mathbb{R}}
\newcommand{\T}{\mathbb{T}} \newcommand{\Z}{\mathbb{Z}}
\newcommand{\Q}{\mathbb{Q}}
 
 
\newcommand{\cA}{\mathcal{A}} \newcommand{\cB}{\mathcal{B}}
\newcommand{\cC}{\mathcal{C}} \newcommand{\cD}{\mathcal{D}}
\newcommand{\cE}{\mathcal{E}} \newcommand{\cF}{\mathcal{F}}
\newcommand{\cG}{\mathcal{G}} \newcommand{\cH}{\mathcal{H}}
\newcommand{\cI}{\mathcal{I}} \newcommand{\cJ}{\mathcal{J}}
\newcommand{\cK}{\mathcal{K}} \newcommand{\cL}{\mathcal{L}}
\newcommand{\cM}{\mathcal{M}} \newcommand{\cN}{\mathcal{N}}
\newcommand{\cO}{\mathcal{O}} \newcommand{\cP}{\mathcal{P}}
\newcommand{\cQ}{\mathcal{Q}} \newcommand{\cR}{\mathcal{R}}
\newcommand{\cS}{\mathcal{S}} \newcommand{\cT}{\mathcal{T}}
\newcommand{\cU}{\mathcal{U}} \newcommand{\cV}{\mathcal{V}}
\newcommand{\cW}{\mathcal{W}} \newcommand{\cX}{\mathcal{X}}
\newcommand{\cY}{\mathcal{Y}} \newcommand{\cZ}{\mathcal{Z}}

\newcommand\numberthis{\addtocounter{equation}{1}\tag{\theequation}}


\newenvironment{proof}{{\bf Proof:  }}{\hfill\rule{2mm}{2mm}}
\newenvironment{proofof}[1]{{\bf Proof of #1:  }}{\hfill\rule{2mm}{2mm}}
\newenvironment{proofofnobox}[1]{{\bf#1:  }}{}\newenvironment{example}{{\bf Example:  }}{\hfill\rule{2mm}{2mm}}

%\renewcommand{\thesection}{\lecnum.\arabic{section}}
%\renewcommand{\theequation}{\thesection.\arabic{equation}}
%\renewcommand{\thefigure}{\thesection.\arabic{figure}}

\newtheorem{fact}{Fact}[section]
\newtheorem{lemma}[fact]{Lemma}
\newtheorem{theorem}[fact]{Theorem}
\newtheorem{definition}[fact]{Definition}
\newtheorem{corollary}[fact]{Corollary}
\newtheorem{proposition}[fact]{Proposition}
\newtheorem{claim}[fact]{Claim}
\newtheorem{exercise}[fact]{Exercise}
\newtheorem{note}[fact]{Note}
\newtheorem{conjecture}[fact]{Conjecture}

\newcommand{\size}[1]{\ensuremath{\left|#1\right|}}
\newcommand{\ceil}[1]{\ensuremath{\left\lceil#1\right\rceil}}
\newcommand{\floor}[1]{\ensuremath{\left\lfloor#1\right\rfloor}}

\DeclarePairedDelimiter\abs{\lvert}{\rvert}%
\DeclarePairedDelimiter\norm{\lVert}{\rVert}%
%END MACROS
%Page style
\pagestyle{fancy}

\listfiles

\raggedbottom

\lhead{CO 749 - \today}
\rhead{W. Justin Toth - A1} %CHANGE n to ASSIGNMENT NUMBER ijk TO COURSE CODE
\renewcommand{\headrulewidth}{1pt} %heading underlined
%\renewcommand{\baselinestretch}{1.2} % 1.2 line spacing for legibility (optional)

\begin{document}
\section{}
\paragraph{}
Let $G$ be a graph of minimum degree at least $3$ satisfying all of the following:
\begin{itemize}
	\item Every triangle contains at most one vertex of degree three.
	\item If two vertices of degree three are adjacent then they are in triangles.
	\item No two distinct triangles share a common edge.
\end{itemize}
Suppose for a contradiction that $\abs{E(G)} < 5\abs{V(G)}/3$.
\begin{claim}\label{1.1}
	For every $v \in V(G)$, if $d(v) = 3$ then $v$ has at most one neighbour of degree $3$.
\end{claim}
\begin{proof}
	Assume for a contradiction that there exist $u,v, w \in V(G)$ satisfying
	$$d(u) = d(v) = d(w) = 3$$
	and $uv, vw \in E(G)$, i.e. $u$ and $w$ are adjacent to $v$. But then, using the second property of $G$, $v$ is in a triangle. Denote this triangle by $C$. Since every triangle contains at most one vertex of degree three, $u$ and $w$ are not contained in $C$. But then since $d(v) = 3$, $v$ has at most one neighbour in $C$, a contradiction to $C$ being a $K_3$.
\end{proof}
\paragraph{}
We proceed by a discharging argument. For each vertex $v \in V(G)$ set the initial charge as
$$ch_0(v) = d(v) - 4$$
and for each edge $e \in E(G)$ set the initial charge
$$ch_0(e) = 1/3.$$
Then our total charge is, using handshaking,
\begin{align*}
\sum_{v\in V(G)} ch_0(v) + \sum_{e \in E(G)} ch_0(e) &= \sum_{v \in V(G)} d(v) - 4|V(G)| + \frac{1}{3} |E(G)|\\ &=  \frac{7}{3}|E(G)| - 4|V(G)\\ &< \frac{35}{9}|V(G)| - \frac{36}{9}|V(G)|\\ &< 0
\end{align*}
with the first inequality following from our assumption that $|E(G)| < 5|V(G)|/3$.
\paragraph{}
Redistribute the charge, to obtain $ch_1$, according to the following Rules:
\begin{enumerate}
	\item If edge $uv$ contains precisely one vertex $u$ such that $d(u) = 3$ then $uv$ sends $1/3$ charge to $u$.
	\item If edge $uv$ is such that $d(u) \geq 4$ and $d(v) \geq 4$ and $u,v$ are in a triangle with vertex $w$ then $uv$ sends $1/3$ charge to $w$.
\end{enumerate}
\paragraph{}
Since vertices sends no charge, clearly every vertex $v \in V(G)$ with $d(v) \geq 4$ has $$ch_1(v) = ch_0(v) \geq 0.$$ Now we will show that if $v$ is a vertex with $d(v) = 3$ then $ch_1(v) \geq 0$. From Claim \ref{1.1} $v$ is adjacent with at most one vertex of degree $3$. If $v$ is adjacent with no degree three vertices then each edge incident with $v$ sends $\frac{1}{3}$ charge to $v$ by Rule $1$ and hence
$$ch_1(v) = ch_0(v) + 3\cdot\frac{1}{3} = 0.$$
On the other hand, if $v$ is adjacent with one degree three vertex then $v$ receives $\frac{2}{3}$ charge from incident edges by Rule $1$. The remaining needed $\frac{1}{3}$ is obtained from Rule $2$ as follows. Since $v$ is adjacent with a degree three vertices, $v$ lies in a triangle, and since every triangle contains at most one degree three vertex the edge containing the other two vertices of this triangle sends $\frac{1}{3}$ charge to $v$. Therefore $ch_1(v) \geq 0$.
\paragraph{}
It remains to verify that each edge $uv$ satisfies $ch_1(uv) \geq 0$. Edge $uv$ can satisfy at most one of Rule $1$ and Rule $2$. If $uv$ satisfies Rule $1$ it sends $\frac{1}{3}$ charge and so $ch_1(uv) = 0$. Instead if $uv$ satisfies Rule $2$ it can only do so for one triangle, since no two distinct triangles share a common edge. So $uv$ only sends $\frac{1}{3}$ charge by Rule $2$. Therefore $ch_1(uv) \geq 0$.
\paragraph{}
Therefore we have a contradiction as our total charge is negative but $ch_1$ of each object is non-negative.$\blacksquare$

\newpage
\section{}
\paragraph{}
Assume Lemma $2$ is false. Let $G$ be a minimum counterexample. That is $G$ is a smallest planar graph satisfying:
\begin{itemize}
	\item The girth of $G$ is at least seven,
	\item The minimum degree of $G$ is at least two, and
	\item If $v$ is a degree $2$ vertex of $G$ with neighbours $u$ and $w$ then the degrees of $u$ and $w$ are each at least $4$.
\end{itemize}  
\paragraph{}
Since the minimum degree is at least $2$, $G$ is not acyclic. So $G$ contains cycles, and each cycle is of length at least $7$. Hence every bounded face of $G$ is bounded by a cycle of length at least $7$ and the unbounded face is incident with at least $7$ edges.
\paragraph{}
We proceed by a discharging argument. For each vertex $v \in V(G)$ assign initial charge to $v$:
$$ch_0(v) = 2d(v) - 6.$$
For each face $f$ assign initial charge to $f$:
$$ch_0(f) = |f| - 6.$$
Then by our Lemma from class the sum of the initial charges is $-12$. By the girth and minimum degree of $G$, the charge of every face is at least $1$ and the charge of every vertex is non-negative except for degree $2$ vertices.
\paragraph{}
Discharge, establishing final charges $ch_1$ according to the following Rules:
\begin{enumerate}
	\item Each face $f$ sends $1$ charge to each incident vertex of degree $2$.
	\item Each vertex of degree at least $4$ sends $\frac{1}{2}$ charge to each incident face.
\end{enumerate}
\paragraph{}
If a vertex has degree at least $4$ then their initial charge is at least half their degree, so their final charge is non-negative after sending $\frac{1}{2}$ to each incident face. Degree $3$ vertices neither send nor receive charge and so their final charge is $0$.
\paragraph{}
If $v$ is vertex of degree $2$ then $v$ is incident with two faces, and so receives $2$ charge. Thus the final charge of $v$ is
$$ch_1(v) = ch_0(v) + 2 = -2 + 2 = 0.$$
Hence we have shown that the final charge of each vertex is non-negative. It remains to show the final charge of each face is non-negative to achieve the desired contradiction.
\paragraph{}
Let $f$ be a face of $G$. Let $q$ be the number of degree two vertices incident with $f$. Since the neighbours of any degree two vertex cannot be of degree two we observe that
$$q \leq \floor{\frac{|f|}{2}}.$$
Let $p$ be the number of vertices incident with $f$ of degree at least $4$. Since each degree two vertex has both neighbours of degree four, we have
$$p \geq q.$$
We have that $f$ loses $q$ charge by Rule $1$ and gains $\frac{p}{2}$ charge by Rule $2.$ Thus the final charge of $f$ is
$$ch_1(f) = ch_0(f) -q + p \geq |f| -6 -\frac{q}{2} \geq \frac{3}{4}|f| - 6.$$
So the final charge of $f$ is non-negative when $|f| \geq 8$.
\paragraph{}
Therefore we may assume that $|f| = 7$. In this case we may strengthen our bound on $p$ by observing that
$p \geq q+1$
by a parity argument. Indeed if $p = q$ then $f$ is a cycle of vertices alternating degree $2$ and degree $4$, but the number of vertices on $f$ is odd, so this cannot happen. Now we tighten our estimate of the final charge:
$$ch_1(f) \geq |f| - 6 -\frac{q}{2} + 1 \geq \frac{3}{4}\cdot 7 - 6 + 1 \geq 0.$$
Thus we have shown every face has non-negative final charge, a contradiction. $\blacksquare$

\newpage
\section{}
\end{document}
