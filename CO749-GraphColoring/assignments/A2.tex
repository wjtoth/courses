\documentclass[letterpaper,12pt,oneside,onecolumn]{article}
\usepackage[margin=1in, bottom=1in, top=1in]{geometry} %1 inch margins
\usepackage{amsmath, amssymb, amstext}
\usepackage{fancyhdr}
\usepackage{mathtools}
\usepackage{algorithm}
\usepackage{algpseudocode}
\usepackage{theorem}
\usepackage{tikz}
\usepackage{tkz-berge}
\usepackage[braket, qm]{qcircuit}
\usepackage{hyperref}

%Macros
\newcommand{\A}{\mathbb{A}} \newcommand{\C}{\mathbb{C}}
\newcommand{\D}{\mathbb{D}} \newcommand{\F}{\mathbb{F}}
\newcommand{\N}{\mathbb{N}} \newcommand{\R}{\mathbb{R}}
\newcommand{\T}{\mathbb{T}} \newcommand{\Z}{\mathbb{Z}}
\newcommand{\Q}{\mathbb{Q}}
 
 
\newcommand{\cA}{\mathcal{A}} \newcommand{\cB}{\mathcal{B}}
\newcommand{\cC}{\mathcal{C}} \newcommand{\cD}{\mathcal{D}}
\newcommand{\cE}{\mathcal{E}} \newcommand{\cF}{\mathcal{F}}
\newcommand{\cG}{\mathcal{G}} \newcommand{\cH}{\mathcal{H}}
\newcommand{\cI}{\mathcal{I}} \newcommand{\cJ}{\mathcal{J}}
\newcommand{\cK}{\mathcal{K}} \newcommand{\cL}{\mathcal{L}}
\newcommand{\cM}{\mathcal{M}} \newcommand{\cN}{\mathcal{N}}
\newcommand{\cO}{\mathcal{O}} \newcommand{\cP}{\mathcal{P}}
\newcommand{\cQ}{\mathcal{Q}} \newcommand{\cR}{\mathcal{R}}
\newcommand{\cS}{\mathcal{S}} \newcommand{\cT}{\mathcal{T}}
\newcommand{\cU}{\mathcal{U}} \newcommand{\cV}{\mathcal{V}}
\newcommand{\cW}{\mathcal{W}} \newcommand{\cX}{\mathcal{X}}
\newcommand{\cY}{\mathcal{Y}} \newcommand{\cZ}{\mathcal{Z}}

\newcommand\numberthis{\addtocounter{equation}{1}\tag{\theequation}}


\newenvironment{proof}{{\bf Proof:  }}{\hfill\rule{2mm}{2mm}}
\newenvironment{proofof}[1]{{\bf Proof of #1:  }}{\hfill\rule{2mm}{2mm}}
\newenvironment{proofofnobox}[1]{{\bf#1:  }}{}\newenvironment{example}{{\bf Example:  }}{\hfill\rule{2mm}{2mm}}

%\renewcommand{\thesection}{\lecnum.\arabic{section}}
%\renewcommand{\theequation}{\thesection.\arabic{equation}}
%\renewcommand{\thefigure}{\thesection.\arabic{figure}}

\newtheorem{fact}{Fact}[section]
\newtheorem{lemma}[fact]{Lemma}
\newtheorem{theorem}[fact]{Theorem}
\newtheorem{definition}[fact]{Definition}
\newtheorem{corollary}[fact]{Corollary}
\newtheorem{proposition}[fact]{Proposition}
\newtheorem{claim}[fact]{Claim}
\newtheorem{exercise}[fact]{Exercise}
\newtheorem{note}[fact]{Note}
\newtheorem{conjecture}[fact]{Conjecture}

\newcommand{\size}[1]{\ensuremath{\left|#1\right|}}
\newcommand{\ceil}[1]{\ensuremath{\left\lceil#1\right\rceil}}
\newcommand{\floor}[1]{\ensuremath{\left\lfloor#1\right\rfloor}}

\DeclarePairedDelimiter\abs{\lvert}{\rvert}%
\DeclarePairedDelimiter\norm{\lVert}{\rVert}%
%END MACROS
%Page style
\pagestyle{fancy}

\listfiles

\raggedbottom

\lhead{\today}
\rhead{W. Justin Toth - A2} %CHANGE n to ASSIGNMENT NUMBER ijk TO COURSE CODE
\renewcommand{\headrulewidth}{1pt} %heading underlined
%\renewcommand{\baselinestretch}{1.2} % 1.2 line spacing for legibility (optional)

\begin{document}
\section{}
\paragraph{}
By Voigt's Theorem there exists a planar graph $H$ and a list assignment $L$ with $|L(u)| \geq 4$ for all $u \in V(H)$, such that $H$ is not $L$-colorable. We construct a graph $G$ as follows. Take $5$ distinct copies of $H$: $H_1, \dots, H_5$ where each vertex $u_i \in V(H_i)$ corresponds to vertex $u \in V(H)$, and take an additional vertex $v$. Add edges $vu_i$ for all $u_i \in V(H_i)$ for all $i \in \{1,\dots, k\}$. i.e. $v$ is adjacent to every vertex in every copy of $H$.
\paragraph{}
To see that $G$ is an apex graph observe that $G-v$ consists of five disjoint copies of a planar graph, and hence $G-v$ is planar.
\paragraph{}
We now construct a $5$-list-assignment $L'$ for which $G$ does not have an $L'$-coloring, completing the proof that $G$ is not $5$-list-colorable. Let $c_1, \dots, c_5$ be $5$ disjoint colors from $L$. That is for all $u \in V(H)$, $$L(u) \cap \{c_1, \dots, c_5\} = \emptyset.$$
Let $L'(v) = \{c_1, \dots, c_5\}$. For each $i \in \{1, \dots, 5\}$ and for each $u_i \in V(H_i)$ let $L'(u_i) = L(u) \cup \{c_i\}$. Since $|L(u)| \geq 4$, $|L'(u_i)|\geq 5$. Thus we have a $5$-list-assignment $L'$ for $G$.
\paragraph{}
Suppose for a constradiction $G$ has an $L'$-coloring $\phi$. Say $\phi(v) = c_i$. Then $\phi(u_i) \neq c_i$ for all $u_i \in V(H_i)$. Hence $\phi(u_i) \in L(u_i)$ for all $u_i \in V(H_i)$. That is, $\phi$ restricted to $H_i$ is an $L$-coloring of $H-_i$. But this contradicts that $H_i$ is not $L$-colorable. Therefore $G$ does not have an $L'$-coloring and hence is not $5$-list-colorable. $\blacksquare$
\end{document}
