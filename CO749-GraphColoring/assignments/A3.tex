\documentclass[letterpaper,12pt,oneside,onecolumn]{article}
\usepackage[margin=1in, bottom=1in, top=1in]{geometry} %1 inch margins
\usepackage{amsmath, amssymb, amstext}
\usepackage{fancyhdr}
\usepackage{mathtools}
\usepackage{algorithm}
\usepackage{algpseudocode}
\usepackage{theorem}
\usepackage{tikz}
\usepackage{tkz-berge}
\usepackage[braket, qm]{qcircuit}
\usepackage{hyperref}

%Macros
\newcommand{\A}{\mathbb{A}} \newcommand{\C}{\mathbb{C}}
\newcommand{\D}{\mathbb{D}} \newcommand{\F}{\mathbb{F}}
\newcommand{\N}{\mathbb{N}} \newcommand{\R}{\mathbb{R}}
\newcommand{\T}{\mathbb{T}} \newcommand{\Z}{\mathbb{Z}}
\newcommand{\Q}{\mathbb{Q}}
 
 
\newcommand{\cA}{\mathcal{A}} \newcommand{\cB}{\mathcal{B}}
\newcommand{\cC}{\mathcal{C}} \newcommand{\cD}{\mathcal{D}}
\newcommand{\cE}{\mathcal{E}} \newcommand{\cF}{\mathcal{F}}
\newcommand{\cG}{\mathcal{G}} \newcommand{\cH}{\mathcal{H}}
\newcommand{\cI}{\mathcal{I}} \newcommand{\cJ}{\mathcal{J}}
\newcommand{\cK}{\mathcal{K}} \newcommand{\cL}{\mathcal{L}}
\newcommand{\cM}{\mathcal{M}} \newcommand{\cN}{\mathcal{N}}
\newcommand{\cO}{\mathcal{O}} \newcommand{\cP}{\mathcal{P}}
\newcommand{\cQ}{\mathcal{Q}} \newcommand{\cR}{\mathcal{R}}
\newcommand{\cS}{\mathcal{S}} \newcommand{\cT}{\mathcal{T}}
\newcommand{\cU}{\mathcal{U}} \newcommand{\cV}{\mathcal{V}}
\newcommand{\cW}{\mathcal{W}} \newcommand{\cX}{\mathcal{X}}
\newcommand{\cY}{\mathcal{Y}} \newcommand{\cZ}{\mathcal{Z}}

\newcommand\numberthis{\addtocounter{equation}{1}\tag{\theequation}}


\newenvironment{proof}{{\bf Proof:  }}{\hfill\rule{2mm}{2mm}}
\newenvironment{proofof}[1]{{\bf Proof of #1:  }}{\hfill\rule{2mm}{2mm}}
\newenvironment{proofofnobox}[1]{{\bf#1:  }}{}\newenvironment{example}{{\bf Example:  }}{\hfill\rule{2mm}{2mm}}

%\renewcommand{\thesection}{\lecnum.\arabic{section}}
%\renewcommand{\theequation}{\thesection.\arabic{equation}}
%\renewcommand{\thefigure}{\thesection.\arabic{figure}}

\newtheorem{fact}{Fact}[section]
\newtheorem{lemma}[fact]{Lemma}
\newtheorem{theorem}[fact]{Theorem}
\newtheorem{definition}[fact]{Definition}
\newtheorem{corollary}[fact]{Corollary}
\newtheorem{proposition}[fact]{Proposition}
\newtheorem{claim}[fact]{Claim}
\newtheorem{exercise}[fact]{Exercise}
\newtheorem{note}[fact]{Note}
\newtheorem{conjecture}[fact]{Conjecture}

\newcommand{\size}[1]{\ensuremath{\left|#1\right|}}
\newcommand{\ceil}[1]{\ensuremath{\left\lceil#1\right\rceil}}
\newcommand{\floor}[1]{\ensuremath{\left\lfloor#1\right\rfloor}}

\DeclarePairedDelimiter\abs{\lvert}{\rvert}%
\DeclarePairedDelimiter\norm{\lVert}{\rVert}%
%END MACROS
%Page style
\pagestyle{fancy}

\listfiles

\raggedbottom

\lhead{\today}
\rhead{W. Justin Toth - A3} %CHANGE n to ASSIGNMENT NUMBER ijk TO COURSE CODE
\renewcommand{\headrulewidth}{1pt} %heading underlined
%\renewcommand{\baselinestretch}{1.2} % 1.2 line spacing for legibility (optional)

\begin{document}
\section{}
A $k$-critical graph $G$ is $m$-\emph{splittable} (with $2\le m \le k-1$) if for every vertex $z\in V(G)$ and every split of $z$ into vertices $z_1,\ldots, z_m$ of positive degree , the resulting graph has a $(k-1)$-coloring such that $z_1, \ldots, z_m$ have distinct colors. We say such $G$ is \emph{splittable} if $G$ is $m$-splittable for every $m$ such that $2\le m \le k-1$.
\begin{lemma}
	An Ore-composition of two splittable $k$-critical graphs is splittable $k$-critical.
\end{lemma}
\begin{proof}
	Let $G_1$ and $G_2$ be two splittable $k$-critical graphs. Let $G$ be the Ore-composition of $G_1$ and $G_2$ obtained by deleting edge $xy \in E(G_1)$ and splitting vertex $z \in V(G_2)$ into $z_1$ and $z_2$ of positive degree and identifying $x$ with $z_1$ and $y$ with $z_2$. We denote the identified vertex for $x$ with $z_1$ as $z_x$, and for $y$ with $z_2$ as $z_y$ in $G$.
	\paragraph{}
	We first show that $G$ is $k$-critical. Suppose for a contradiction $G$ has a $(k-1)$-coloring $\phi$. If $\phi(z_x) \neq \phi(z_y)$ then $\phi$ restricted to $G_1$ (taking in the obvious manner $\phi(x) = \phi(z_x)$ and $\phi(y) = \phi(z_y)$) is a $(k-1)$-coloring of $G_1$. If $\phi(z_x) = \phi(z_y)$ then $\phi$ restricted to $G_2$, and takiing $\phi(z) = \phi(z_x) = \phi(z_y)$ yields a $(k-1)$-coloring of $G_2$. In either case we contradicted that $G_1$ and $G_2$ do not have a $(k-1)$-coloring. Therefore $G$ is not $(k-1)$-colorable.
	\paragraph{}
	To finish verifying that $G$ is $k$-critical we need to show that every strict subgraph of $G$ is $(k-1)$-colorable. Let $H \subseteq G$ be a strict subgraph of $G$. We proceed by case-distinction on $H$.
		\subparagraph{} Case $1$: $z_x$ is not contained in $H$ and $z_y$ is not contained in $H$. Then $H$ is the disjoint union of a strict subgraph of $G_1$ and a strict subgraph of $G_2$. Since $G_1$ and $G_2$ are $k$-critical we can obtain a $(k-1)$-coloring of their respective strict subgraphs in $H$ and combine them to obtain a $(k-1)$-coloring of $H$.
		\subparagraph{} Case $2:$ Exactly one of $z_x, z_y$ is in $H$. Without loss of generality say $z_x \in V(H)$ and $z_y \not\in V(H)$. Then treating $z_x$ as $x$ in $G_1$ and $z$ in $G_2$ we see that $H$ is the union of two strict subgraphs $H_1 \subseteq G_1$ and $H_2 \subseteq G_2$, overlapping only at $z_x$. We see $H_2$ is a strict subgraph since the degree of $z_1$ and $z_2$ are both positive. Again we invoke criticality of $G_1$ and $G_2$ to obtain $(k-1)$-colorings of $H_1$ and $H_2$, which we join into a $(k-1)$-coloring of $H$. Note we permute the coloring of $H_1$ so the two colorings agree at $z_x$ if necessary.
		\subparagraph{} Case $3:$. Both $z_x$ and $z_y$ are in $H$. Let $G_2'$ denote the graph $G_2$ with $z$ split into $z_1$ and $z_2$. Then $H$ is the union of two subgraphs $H_1 \subseteq G_1$ and $H_2 \subseteq G_2'$ overlapping only at $\{z_x,z_y\}$. Since $H$ is a strict subgraph of $G$ one of $H_1$ or $H_2$ is a strict subgraph of $G_1-xy$ or $G_2'$ respectively.
		\subparagraph{} First suppose $H_1$ is a strict subgraph of $G_1-xy$. Then invoking the criticality of $G_1$, we obtain that $H_1 + xy$ has a $(k-1)$-coloring $\phi_1$. Then $\phi_1$ is a $(k-1)$-coloring of $H_1$ where $\phi_1(x)\neq \phi_1(y)$. Since $G_2$ is $2$-splittable, there exists a $(k-1)$-coloring $\phi_2$ of $G_2'$. Note that $\phi_2(z_1) \neq \phi_2(z_2)$ for otherwise setting $\phi_2(z) = \phi_2(z_1) = \phi_2(z_2)$ yields a $(k-1)$-coloring of $G_2$. Hence we may permute the colors of $\phi_2$ so $\phi_2(z_1) = \phi_1(x)$ and $\phi_2(z_2) = \phi_1(y)$. Then the coloring obtained by combining $\phi_1$ and $\phi_2$ is a $(k-1)$-coloring of $H$.
		\subparagraph{}
		So we may assume $H_1$ is not a strict subgraph of $G_1-xy$, i.e. $H_1 = G_1-xy$. Then $H_2$ is a strict subgraph of $G_2'$. By identifying $z_x$ and $z_y$ as $z$ in $H_2$ we obtain a strict subgraph of $G_2$. By criticality of $G_2$, this subgraph has a $(k-1)$-coloring $\phi_2$. By setting $\phi_2(z_x) = \phi_2(z_y) = \phi_2(z)$ we obtain a $(k-1)$-coloring of $H_2$ wherein $z_x$ and $z_y$ receive the same color. Invoking criticality of $G_1$, we obtain a $(k-1)$-coloring $\phi_1$ of $H_1$ ($=G_1-xy$). Observe that $\phi_1(x) = \phi_1(y)$, for otherwise $\phi_1$ is a $(k-1)$-coloring of $G_1$. But then we may permute the colors of $\phi_2$ so $\phi_2(z_1) = \phi_1(x)$ and $\phi_2(z_2) = \phi_1(y)$. Again the coloring obtained by combining $\phi_1$ and $\phi_2$ is a $(k-1)$-coloring of $H$, as desired.
	\paragraph{Splittability}
	Thus we have shown that $G$ is $k$-critical. It remains to verify that $G$ is splittable. First consider $v \in V(G) \backslash\{z_x,z_y\}$. Consider $G'$, where we split $v$ into $v_1, \dots, v_m$ each with positive degree, $2 \leq m \leq k-1$. If $v \in V(G_1)$ then apply the corresponding split $v_1, \dots, v_m$ to $G_1$, and use that $G_1$ is splittable to obtain a $(k-1)$-coloring $\phi_1$. Observe $\phi_1(x) \neq \phi_1(y)$ since $xy \in E(G_1)$. Since $G_2$ is $2$-splittable, $G - G_1$ has a $(k-1)$-coloring $\phi_2$. Since $G_2$ is not $(k-1)$-colorable, $\phi_2(z_x) \neq \phi_2(z_y)$. Again we may permute colors of $\phi_2$ so $\phi_1$ and $\phi_2$ agree on $z_x$ and $z_y$, obtaining a $(k-1)$-coloring of $G'$.
	\paragraph{}
	If $v \in V(G_2)$ then apply the corresponding split $v_1, \dots, v_m$ to $G_2$ and use splittability to obtain a $(k-1)$-coloring $\phi_2$. Setting $\phi_2(z_x) = \phi_2(z_y) = \phi_2(z)$ we obtain a $(k-1)$-coloring of $G-G_1$. By criticality, $G_1-xy$ has a $(k-1)$-coloring $\phi_1$. Since $G_1$ is not $(k-1)$-colorable, $\phi_1(x) = \phi_1(y)$. Hence we may permute colors of $\phi_2$ so $\phi_1$ and $\phi_2$ agree on $z_x$ and $z_y$, obtaining a $(k-1)$-coloring of $G'$.
	\paragraph{}
	It remains to consider splitting one $z_x$ or $z_y$ in $G$. Without loss of generality let $G'$ be the graph obtained by splitting $z_x$ into $z_x^1, \dots, z_x^m$, each with positive degree, $2 \leq m \leq k-1$. Split $x$ in $G_1$ into $x^1, \dots, x^m$ with neighbourhoods $N(x^i) = N(z_x^i)\cap V(G_1)$ yielding graph $G_1'$. Obtain a $(k-1)$-coloring $\phi_1$ of $G_1'$ by splittability of $G$.
	\subparagraph{}Case $1:$ $\phi_1(x^i) = \phi_1(y)$ for some $i \in \{1,\dots, m\}$. Split $z$ in $G_2$ into $z^1,\dots, z^m$ with neighbourhoods $N(z^j) = N(z_x^j) \cap V(G_2)$ for $j \neq i$ and $$N(z^i) = (N(z_x^i) \cup N(z_y)) \cap V(G_2)$$
	obtaining graph $G_2'$. Color $G_2'$ by splittability of $G_2$ obtaining $(k-1)$-coloring $\phi_2$. Setting $\phi_2(z_y) = \phi_2(z^i)$ we obtain a $(k-1)$-coloring of $(G-G_1) \cup \{z_x^1, \dots, z_x^m, z_y\}$. Since $\phi_1$ and $\phi_2$ give all the split vertices different colors, and give $z_y$ the same color $z_x^i$ we may permute $\phi_1$ and $\phi_2$ to agree on $\{z_x^1, \dots, z_x^m, z_y\}$ obtaining a $(k-1)$-coloring of $G'$.
	\subparagraph{}Case $2:$ $\phi_1(x^i) \neq \phi_2(y)$ for all $i \in \{1,\dots, m\}$. Notice in this case $m < k-1$ by the pidgeonhole principle since $\phi_1$ assigns different colors to each split vertex. Split $z$ in $G_2$ into $z^1, \dots, z^m, z^{m+1}$ with neighbours $N(z^i) = N(z_x^i) \cap V(G_2)$ for $i \leq m$ and set $N(z^{m+1}) = N(z_y) \cap V(G_2)$. Applying splitability we obtain a $(k-1)$-coloring $\phi_2$. Now both $\phi_1$ and $\phi_2$ give $\{z_x^1, \dots, z_x^m, z_y\}$ different colors, so we may permute their colors to agree on $\{z_x^1, \dots, z_x^m, z_y\}$, and join the colorings to obtain a coloring of $G'$, as desired.
	\end{proof}
\begin{corollary}
	If $G$ is a $k$-Ore graph then $G$ is splittable $k$-critical.
\end{corollary}
\begin{proof}
	Suppose not. Let $G$ be a minimum counterexample. Then $G$ cannot be the Ore-composition of two $k$-Ore graphs $G_1$, $G_2$, for otherwise $G_1$ and $G_2$ are splittable $k$-critical by minimality and hence $G$ is splittable $k$-critical by the previous Lemma.
	\paragraph{}
	Thus $G$ is isomorphic to $K_k$. It remains to show $K_k$ is splittable. Suppose we want to split vertex $v \in V(G)$ into $v_1, \dots, v_m$ with $2\leq m \leq k-1$ such that each $v_i$ has positive degree, obtaining graph $G'$. We can trivially color $G - v$ with $k-1$ colors, obtaining coloring $\phi$. Now we can extend $\phi$ to $G'$ since each $v_i$ has degree at most $k-2$ and hence $|\phi(N(v_i))| \leq k-2$ and there is a free color available to assign $v_i$. This contradicts that
\end{proof}

\newpage
\section{}
\subsection{a}
\begin{lemma}
	If $G$ is the Ore-composition of two graphs $G_1$ and $G_2$ then $\alpha(G) \leq \alpha(G_1) + \alpha(G_2)$.
\end{lemma}
\begin{proof}
Let $G$ be the Ore-composition of graphs $G_1$ and $G_2$ obtained by deleting edge $xy \in E(G_1)$ and splitting vertex $z \in V(G_2)$ into $z_1$ and $z_2$ of positive degree and identifying $x$ with $z_1$ and $y$ with $z_2$. We denote the identified vertex for $x$ with $z_1$ as $z_x$, and for $y$ with $z_2$ as $z_y$ in $G$. Let $S \subseteq V(G)$ be an independent set in $G$ such that $|S| = \alpha (G)$. 
\paragraph{}
We claim that we can always partition $S$ into $S_1$ and $S_2$ where $S_1$ is an independent set in $G_1$ and $S_2$ is an independent set in $G_2$. We proceed by case distinction:
\subparagraph{} Case $1:$ $\{z_x,z_y\} \cap S = \emptyset$. Then trivially $S_1 = S \cap V(G_1)$ and $S_2 = S\cap V(G_2) = S \backslash S_1$.
\subparagraph{} Case $2:$ $|\{z_x,z_y\} \cap S| \leq 1$. Without loss of generality we may assume $x \in S$ and  $y \not\in S$. Then treating $z_x$ as $x$ we can partition as before: $S_1 = (S \cap V(G_1)) \cup \{x\}$ and $S_2 = S\cap V(G_12)$.
\subparagraph{} Case $3:$ $\{z_x, z_y\} \subseteq S$. Then for all $v \in S\cap V(G_2)$, $v$ is not adjacent to $z$, for otherwise $v$ would be adjacent to one of $z_x, z_y$ contradicting that $S$ is an independent $S$. So treating $z_x$ as $x$ and $z_y$ as $z$, we can partition $S$ into $$S_1 = ((S\backslash\{z_x,z_y\})\cap V(G_1)) \cup \{x\}$$ and $$S_2 = ((S\backslash\{z_x,z_y\})\cap V(G_2)) \cup \{z\}.$$
Now using our claim we see that
$$\alpha(G) = |S| = |S_1| + |S_2| \leq \alpha(G_1) + \alpha(G_2)$$
as desired.
\end{proof}

\subsection{b}
\begin{lemma}
	If $G$ is $k$-Ore then $\alpha(G) = \frac{v(G) - 1}{k-1}$.
\end{lemma}
\begin{proof}
	We proceed by induction on $v(G)$. The smallest $k$-Ore graph is $K_k$, for which
	$$\alpha(K_k) = 1 = \frac{k-1}{k-1} = \frac{v(G) -1}{k-1}.$$
	So the base case for induction is satisfied.
	\paragraph{}
	So we consider a $k$-Ore graph $G$ that is the Ore-composition of $G_1$ and $G_2$ and suppose for induction that
	$$\alpha(G_1) = \frac{v(G_1) - 1}{k-1} \quad\text{ and }\quad \alpha(G_2) = \frac{v(G_2) -1}{k-1}.$$
	We observe from the previous Lemma, and induction, that
	$$\alpha(G) \leq \alpha(G_1) + \alpha(G_2) = \frac{v(G_1) +v(G_2)- 2}{k-1} = \frac{v(G) + 1 - 2}{k-1} = \frac{v(G) - 1}{k-1}.$$
	Thus $\alpha(G) \leq \frac{v(G) - 1}{k-1}$. To show $\alpha(G) \geq \frac{v(G) - 1}{k-1}$ it suffices to construct an independent set in $G$ of size $\frac{v(G) - 1}{k-1}$.
	\paragraph{}
	For $i \in \{1, \dots, 2\}$ let $S_i$ be in an independent set in $G_i$ satisfying $$|S_i| = \alpha(G_i) = \frac{v(G_i) - 1}{k-1}.$$ We use the same notation as in the above problems for $x,y, z, z_x, z_y$. Since $xy \in E(G_1)$, $|\{x,y\} \cap S_1| \leq 1$. Without loss of generality suppose $y \not\in S_1$. Treating $x$ as $z_x$ and $z$ as $z_y$ we observe that
	$$S := S_1 \cup S_2$$
	is an independent set of $G$ of size $|S_1| + |S_2|$. But then we see that
	$$|S| = |S_1| + |S_2| =  \frac{v(G_1) +v(G_2)- 2}{k-1} = \frac{v(G) - 1}{k-1}.$$
	Thus $\alpha(G) \geq \frac{v(G) - 1}{k-1}$, and therefore by induction the Lemma is proven.
\end{proof}

\subsection{c}
\paragraph{}
From our work in class it suffices to show $c\alpha(R') \leq c\alpha(R) + c\alpha(W) - c\alpha(K_{|X|}) + c$ for $c \geq 0$. Since we know the core is always non-empty, $\alpha(K_{|X|}) = 1$ and so the problem reduces to showing
$$\alpha(R') \leq \alpha(R) + \alpha(W).$$
Let $S$ be an independent set of $G[R']$ such that $|S| = \alpha(R')$. Since $R'= V(W) \cup R - \{v_1, \dots, v_{k-1}\}$, where $\{v_1, \dots, v_{k-1}\}$ are the set of color-identified vertices, as in the usual definition. So we have
$$|S| \leq |S\cap (V(W) - \{v_1, \dots, v_k\}| + |S\cap R|$$
On the one hand $S\cap (V(W) - \{v_1, \dots, v_k\})$ is an independent set in $W$ and $S \cap R$ is an indepenent set in $G[R]$, so then
$$ \alpha(R') = |S| \leq |S\cap (V(W) - \{v_1, \dots, v_k\}| + |S\cap R| \leq \alpha(W) + \alpha(R)$$
as desired. $\blacksquare$
\newpage
\section{}
\subsection{a}
\begin{lemma}
If $G$ is the Ore-composition of $G_1$ and $G_2$ then
$$T_k(G) \geq T_k(G_1-e) + T_k(G_2 - z) \geq T_k(G_1) + T_k(G_2) - 2$$
where $e = xy$ is the replaced edge of $G_1$ and $z$ is the split vertex of $G_2$.
\end{lemma}
\begin{proof}
Let $H_1 \subseteq G_1- e$ be the subgraph of $G_1 - e$ which achieves the optimum in 
$$T_k(G_1-e) := \max_{H \subseteq G_1 - e} T_k(H).$$
Similarly let $H_2 \subseteq G_2 - z$ be the subgraph of $G_2 - z$ which achieves the optimum in $T_k(G_2 - z)$. Then $H_1 \cup H_2$ is a disjoint union of cliques in $G$. So we have
$$T_k(G_1-e) + T_k(G_2 - z) = T_k(H_1) + T_k(H_2) = T_k(H_1 \cup H_2) \leq T_k(G).$$
So the first inequality of the lemma is proven.
\paragraph{}
Now let $H_1' \subseteq G_1$ be the subgraph of $G_1$ which achieves the optimum in $T_k(G_1)$, and let $H_2' \subseteq G_2$ be the subgraph of $G_2$ which achieves the optimum in $T_k(G_2)$. If $|e \cap V(H'_1)| \leq 1$ then $H'_1$ is a subgraph of $G_1 - e$ and so $T_k(G_1) \leq T_k(G_1-e)$. Otherwise, i.e. $ e\in E(H'_1)$ then $H'_1 - x$ is a subgraph of $G_1 - e$, and $T_k(H'_1-x) = T_k(H'_1) - 1$. Thus 
$$T_k(G_1 - e) \geq T_k(H'_1-x) \geq T_k(H'_1) - 1 =T_k(G_1) - 1.$$
Similarly $H'_2 - z$ is a subgraph of $G_2 - v$ and $T_k(H'_2-z) \geq T_k(H'_2) - 1$. So 
$$T_k(G_2 - z) \geq T_k(H'_2 - z) \geq T_k(H'_2) - 1 = T_k(G_2) - 1.$$
Summing our inequalities above we obtain
$$T_k(G_1 - e) + T_k(G_2 - z) \geq T_k(G_1) + T_k(G_2) - 2$$
as desired.
\end{proof}
\subsection{b}
\begin{lemma}
If $G$ is $k$-Ore and $G \neq K_k$ then $T_k(G) \geq 2+ \frac{v(G) - 1}{k-1}$.
\end{lemma}
\begin{proof}
Suppose not and let $G$ be a minimum counterexample. Since $G$ is $k$-Ore and $G\neq K_k$ there exist $k$-Ore graphs $G_1$ and $G_2$ such that $G$ is the Ore-composition of $G_1$ and $G_2$. We proceed by case distinction on if $G_1, G_2$ are $K_k$ or not. If neither $G_1$ nor $G_2$ are $K_k$ then by minimality of $G$,
$$T_k(G_i) \geq 2 + \frac{v(G_i) - 1}{k-1}$$
for $i =1,2$. But then combining with the inequality
$$T_k(G) \geq T_k(G_1) + T_k(G_2) - 2$$
from the previous Lemma, we see that
$$T_k(G) \geq 2 + \frac{v(G_1) - 1}{k-1} + 2 + \frac{v(G_2) - 1}{k-1} -2 \geq 2 + \frac{v(G)-1}{k-1}$$
as desired, using that $v(G_1) + v(G_2) -1 = v(G)$.
\paragraph{}
If $G_1 = K_k$ and $G_2 \neq K_k$ then using the first inequality from the previous Lemma we have
$$T_k(G) \geq T_k(G_1-e) + T_k(G_2-z) \geq 2 + T_k(G_2-z)$$
Since $T_k(G_2-z) \geq T_k(G_2) - 1$, 
$$T_k(G) \geq 1 + T_k(G_2) \geq \frac{k-1}{k-1} + \frac{v(G_2)-1}{k-1} + 2= \frac{v(G_1) + v(G_2) - 2}{k-1} +2 = \frac{v(G) - 1}{k-1} +2$$
The proof for $G_2 = K_k$ and $G_1 \neq K_k$ follows similarly.
\paragraph{}
If $G_1 = K_k$ and $G_2 = K_k$ then 
$$T_k(G) \geq T_k(G_1-e) + T_k(G_2 - z) = 2  + 2 = \frac{2(k-1)}{k-1} + 2 = \frac{v(G_1) + v(G_2) -2}{k-1} + 2 = \frac{v(G) -1}{k-1} + 2$$
as desired.
\end{proof}
\subsection{c}
\paragraph{}
It suffices to show
$$T_k(R) + T_k(W) - T_k(K_{|X|}) \leq T_k(R') + |X|.$$
Let $H_R \subseteq G[R]$ be a disjoint union of cliques attaining the maximum in $T_k(R)$. Let $H_W \subseteq W$ be a disjoint union of cliques attaining the maximum in $T_k(W)$. If a vertex in $X$ is deleted from a clique in $H_W$ then the value of $T_k$ for the resulting graph will decrease by at most one. Hence $T_k(H_W - X) \geq T_k(H_W) - |X|$. But then $H_R \cup (H_W - X)$ is a disjoint union of cliques in $G[R']$. So we have
\begin{align*}
T_k(R) + T_k(W) - T_k(K_{|X|}) &\leq T_k(H_R) + T_k(H_W)\\ &\leq T_k(H_R) + T_k(H_W - X) - |X|\\ &\leq T_k(H_R \cup (H_W-X)) - |X|\\ &\leq T_k(R') - |X|\end{align*}
as desired. $\blacksquare$
\newpage
\section{}
\paragraph{}
Let $k \geq 4$. Let $p(G) = (k-1)v(G) - 2e(G)$. Then $p(G) \leq 0$ if $G$ is $k$-critical since the minimum degree of $G$ is $k-1$. Let $G$ be a counterexample to the potential version of Brooks' Theorem. That is, $G$ is $k$-critical, $G\neq K_k$, and $p(G) = 0$. Since $p(G) = 0$, we have that $G$ is $(k-1)$-regular.
\begin{claim}
If $R \subset V(G)$ then $p(R) \geq k-1$.
\end{claim}
\begin{proof}
Suppose not and let $R$ be a counterexample with maximum number of vertices. Then $R$ has an extension $R'$ with extender $W$ and core $X$. By the potential-extension lemma,
$$p(R') \leq p(R) + p(W) - p(K_{|X|})$$
Now $$p(R) =  (k-1)|R| - 2e(R) \geq (k-|R|)|R|$$
which is at least $k-1$ when $1 \leq |R| \leq k-1$. Hence $|R| \geq k$, and since $G$ is not equal to $K_k$ and $G$ is $(k-1)$-regular, $G[R]$ is not a clique of size at least $k$. Therefore $W$ is a strict subgraph of $G$, and so by minimality of $G$,
$$p(W) \leq -1.$$
By the same reasoning as above for $R$, $$p(K_{|X|}) \geq k-1.$$
Thus we have
$$p(R') \leq p(R) + p(W) - p(K_{|X|}) \leq (k-2)  - 1 -(k-1) \leq - 2$$
By maximality of $R$, $p(R') \geq k-1$ contradicting the above.
\end{proof}
\begin{claim}
If $z$ is a vertex with two non-adjacent neighbours $x,y$ then there exists a $(k-1)$-clique $H$ in $G\backslash \{x,y,z\}$ such that $N(x) \cap V(H)$, $N(y) \cap V(H)$ partition $V(H)$.
\end{claim}
\begin{proof}
Let $G'$ be the graph obtained from $G$ by identifying vertices $x$ and $y$ (deleting parallel edges if necessary) as vertex $v_{xy}$ and deleting vertex $z$. If $G'$ has a $(k-1)$-coloring $\phi'$ then $\phi'$ can be extending to a $(k-1)$-coloring of $G$ by assigning $\phi'(x ) = \phi'(y) = \phi'(v_{xy})$, then assigning $z$ a color not present on the neighbours of $z$. Such a color exists since $G$ is $(k-1)$-regular and two neighbours of $z$, $x$ and $y$, receive the same color under $\phi'$.
\paragraph{}
Thus $G'$ is not $(k-1)$-colorable. So $G'$ has a $k$-critical subgraph $K'$. Then $v_{xy} \in V(K')$ for otherwise $K'$ is a $k$-critical strict subgraph of $G$, contradicting that $G$ is $k$-critical. By minimality of $G$ we have
$$p(K') \leq 0$$
with equality if and only if $K'$ is $K_k$. Let $K$ be the graph $K' - v_{xy} + \{x,y\}$. Then $K$ is a strict subgraph of $G$, since $z \not\in V(K)$. So by the previous claim
$$k-1 \leq p(G[V(K')]) \leq p(K')$$
Now we observe that
$$p(K) = p(K') + (k-1) - 2|E(v_{xy}, N(x) \cap N(y) \cap V(K'))|$$
Since $K$ has more vertex that $K'$, and parallel edges were deleted in obtaining $G'$. Combining inequalities we see that
$$k-1 \leq p(K) \leq 0 + (k-1) - 2|E(v_{xy}, N(x) \cap N(y) \cap V(K'))|$$
and hence 
$$|E(v_{xy}, N(x) \cap N(y) \cap V(K'))| \leq 0.$$
Since $|E(v_{xy}, N(x) \cap N(y) \cap V(K'))|$ is non-negative this implies $|E(v_{xy}, N(x) \cap N(y) \cap V(K'))| = 0$ and thus
$N(x)\cap N(y) \cap V(K') = \emptyset$. Furthermore the displayed inequalities are tight, implying that $K'$ is isomorphic to $K_k$. Letting $H = K'-v_{xy}$ we obtain the desired clique of the claim.
\end{proof}
\paragraph{}
Since $G$ is not $K_k$, there exists a vertex $z$ with tow non-adjacent neighbbours $x,y$. Invoking the previous Claim we obtain a clique $H$ in $G\backslash\{x,y,z\}$ such that $N(x) \cap V(H)$, $N(y) \cap V(H)$ partition $V(H)$. Now let $G'$ be the graph obtained from $G$ by adding edge $xy$ and deleting $H$. If $G'$ has a $(k-1)$-coloring $\phi'$ then we claim $\phi'$ can be extended to a $(k-1)$-coloring of $G$. Since $G$ is $(k-1)$-regular and the neighbours of $x$ and $y$ partition $V(H)$, every vertex of $H$ is adjacent with every other vertex of $H$, and precisely one of $x$ or $y$. Let $v_x \in N(x) \cap V(H)$ and let $v_y \in N(y) \cap V(H)$. Assign $\phi'(v_x) = \phi'(y)$ and $\phi'(v_y) = \phi'(x)$. Assign the remaining $k-3$ colors (those not equal to $\phi'(x)$ or $\phi'(y)$) to the $k-3$ vertices of $H - \{v_x, v_y\}$ so each receives a unique color. This yields a $(k-1)$-coloring of $G$, hence $G'$ is not $(k-1)$-colorable.
\paragraph{}
Since $G'$ is not $(k-1)$-colorable, $G'$ contains a $k$-critical subgraph $W$. Then $xy \in E(W)$, for otherwise $W$ is a $k$-critical subgraph of $G$, contradicting that $G$ is $k$-critical. Now, using Claim $4.1$, we observe that
$$p(W) = p(G[V(W)]) - 2 \geq k-1 - 2 = k-3 \geq 1 $$
with the last inequality following since $k \geq 4$. But on the other hand $p(W) \leq 0$ since $W$ is $k$-critical. Then we have $1 \leq 0$, a contradiction. $\blacksquare$

\end{document}
