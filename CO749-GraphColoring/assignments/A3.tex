\documentclass[letterpaper,12pt,oneside,onecolumn]{article}
\usepackage[margin=1in, bottom=1in, top=1in]{geometry} %1 inch margins
\usepackage{amsmath, amssymb, amstext}
\usepackage{fancyhdr}
\usepackage{mathtools}
\usepackage{algorithm}
\usepackage{algpseudocode}
\usepackage{theorem}
\usepackage{tikz}
\usepackage{tkz-berge}
\usepackage[braket, qm]{qcircuit}
\usepackage{hyperref}

%Macros
\newcommand{\A}{\mathbb{A}} \newcommand{\C}{\mathbb{C}}
\newcommand{\D}{\mathbb{D}} \newcommand{\F}{\mathbb{F}}
\newcommand{\N}{\mathbb{N}} \newcommand{\R}{\mathbb{R}}
\newcommand{\T}{\mathbb{T}} \newcommand{\Z}{\mathbb{Z}}
\newcommand{\Q}{\mathbb{Q}}
 
 
\newcommand{\cA}{\mathcal{A}} \newcommand{\cB}{\mathcal{B}}
\newcommand{\cC}{\mathcal{C}} \newcommand{\cD}{\mathcal{D}}
\newcommand{\cE}{\mathcal{E}} \newcommand{\cF}{\mathcal{F}}
\newcommand{\cG}{\mathcal{G}} \newcommand{\cH}{\mathcal{H}}
\newcommand{\cI}{\mathcal{I}} \newcommand{\cJ}{\mathcal{J}}
\newcommand{\cK}{\mathcal{K}} \newcommand{\cL}{\mathcal{L}}
\newcommand{\cM}{\mathcal{M}} \newcommand{\cN}{\mathcal{N}}
\newcommand{\cO}{\mathcal{O}} \newcommand{\cP}{\mathcal{P}}
\newcommand{\cQ}{\mathcal{Q}} \newcommand{\cR}{\mathcal{R}}
\newcommand{\cS}{\mathcal{S}} \newcommand{\cT}{\mathcal{T}}
\newcommand{\cU}{\mathcal{U}} \newcommand{\cV}{\mathcal{V}}
\newcommand{\cW}{\mathcal{W}} \newcommand{\cX}{\mathcal{X}}
\newcommand{\cY}{\mathcal{Y}} \newcommand{\cZ}{\mathcal{Z}}

\newcommand\numberthis{\addtocounter{equation}{1}\tag{\theequation}}


\newenvironment{proof}{{\bf Proof:  }}{\hfill\rule{2mm}{2mm}}
\newenvironment{proofof}[1]{{\bf Proof of #1:  }}{\hfill\rule{2mm}{2mm}}
\newenvironment{proofofnobox}[1]{{\bf#1:  }}{}\newenvironment{example}{{\bf Example:  }}{\hfill\rule{2mm}{2mm}}

%\renewcommand{\thesection}{\lecnum.\arabic{section}}
%\renewcommand{\theequation}{\thesection.\arabic{equation}}
%\renewcommand{\thefigure}{\thesection.\arabic{figure}}

\newtheorem{fact}{Fact}[section]
\newtheorem{lemma}[fact]{Lemma}
\newtheorem{theorem}[fact]{Theorem}
\newtheorem{definition}[fact]{Definition}
\newtheorem{corollary}[fact]{Corollary}
\newtheorem{proposition}[fact]{Proposition}
\newtheorem{claim}[fact]{Claim}
\newtheorem{exercise}[fact]{Exercise}
\newtheorem{note}[fact]{Note}
\newtheorem{conjecture}[fact]{Conjecture}

\newcommand{\size}[1]{\ensuremath{\left|#1\right|}}
\newcommand{\ceil}[1]{\ensuremath{\left\lceil#1\right\rceil}}
\newcommand{\floor}[1]{\ensuremath{\left\lfloor#1\right\rfloor}}

\DeclarePairedDelimiter\abs{\lvert}{\rvert}%
\DeclarePairedDelimiter\norm{\lVert}{\rVert}%
%END MACROS
%Page style
\pagestyle{fancy}

\listfiles

\raggedbottom

\lhead{\today}
\rhead{W. Justin Toth - A3} %CHANGE n to ASSIGNMENT NUMBER ijk TO COURSE CODE
\renewcommand{\headrulewidth}{1pt} %heading underlined
%\renewcommand{\baselinestretch}{1.2} % 1.2 line spacing for legibility (optional)

\begin{document}
\section{}
A $k$-critical graph $G$ is $m$-\emph{splittable} (with $2\le m \le k-1$) if for every vertex $z\in V(G)$ and every split of $z$ into vertices $z_1,\ldots, z_m$ of positive degree , the resulting graph has a $(k-1)$-coloring such that $z_1, \ldots, z_m$ have distinct colors. We say such $G$ is \emph{splittable} if $G$ is $m$-splittable for every $m$ such that $2\le m \le k-1$.
\begin{lemma}
	An Ore-composition of two splittable $k$-critical graphs is splittable $k$-critical.
\end{lemma}
\begin{proof}
	Let $G_1$ and $G_2$ be two splittable $k$-critical graphs. Let $G$ be the Ore-composition of $G_1$ and $G_2$ obtained by deleting edge $xy \in E(G_1)$ and splitting vertex $z \in V(G_2)$ into $z_1$ and $z_2$ of positive degree and identifying $x$ with $z_1$ and $y$ with $z_2$. We denote the identified vertex for $x$ with $z_1$ as $z_x$, and for $y$ with $z_2$ as $z_y$ in $G$.
	\paragraph{}
	We first show that $G$ is $k$-critical. Suppose for a contradiction $G$ has a $(k-1)$-coloring $\phi$. If $\phi(z_x) \neq \phi(z_y)$ then $\phi$ restricted to $G_1$ (taking in the obvious manner $\phi(x) = \phi(z_x)$ and $\phi(y) = \phi(z_y)$) is a $(k-1)$-coloring of $G_1$. If $\phi(z_x) = \phi(z_y)$ then $\phi$ restricted to $G_2$, and takiing $\phi(z) = \phi(z_x) = \phi(z_y)$ yields a $(k-1)$-coloring of $G_2$. In either case we contradicted that $G_1$ and $G_2$ do not have a $(k-1)$-coloring. Therefore $G$ is not $(k-1)$-colorable.
	\paragraph{}
	To finish verifying that $G$ is $k$-critical we need to show that every strict subgraph of $G$ is $(k-1)$-colorable. Let $H \subseteq G$ be a strict subgraph of $G$. We proceed by case-distinction on $H$.
		\subparagraph{} Case $1$: $z_x$ is not contained in $H$ and $z_y$ is not contained in $H$. Then $H$ is the disjoint union of a strict subgraph of $G_1$ and a strict subgraph of $G_2$. Since $G_1$ and $G_2$ are $k$-critical we can obtain a $(k-1)$-coloring of their respective strict subgraphs in $H$ and combine them to obtain a $(k-1)$-coloring of $H$.
		\subparagraph{} Case $2:$ Exactly one of $z_x, z_y$ is in $H$. Without loss of generality say $z_x \in V(H)$ and $z_y \not\in V(H)$. Then treating $z_x$ as $x$ in $G_1$ and $z$ in $G_2$ we see that $H$ is the union of two strict subgraphs $H_1 \subseteq G_1$ and $H_2 \subseteq G_2$, overlapping only at $z_x$. We see $H_2$ is a strict subgraph since the degree of $z_1$ and $z_2$ are both positive. Again we invoke criticality of $G_1$ and $G_2$ to obtain $(k-1)$-colorings of $H_1$ and $H_2$, which we join into a $(k-1)$-coloring of $H$. Note we permute the coloring of $H_1$ so the two colorings agree at $z_x$ if necessary.
		\subparagraph{} Case $3:$. Both $z_x$ and $z_y$ are in $H$. Let $G_2'$ denote the graph $G_2$ with $z$ split into $z_1$ and $z_2$. Then $H$ is the union of two subgraphs $H_1 \subseteq G_1$ and $H_2 \subseteq G_2'$ overlapping only at $\{z_x,z_y\}$. Since $H$ is a strict subgraph of $G$ one of $H_1$ or $H_2$ is a strict subgraph of $G_1-xy$ or $G_2'$ respectively.
		\subparagraph{} First suppose $H_1$ is a strict subgraph of $G_1-xy$. Then invoking the criticality of $G_1$, we obtain that $H_1 + xy$ has a $(k-1)$-coloring $\phi_1$. Then $\phi_1$ is a $(k-1)$-coloring of $H_1$ where $\phi_1(x)\neq \phi_1(y)$. Since $G_2$ is $2$-splittable, there exists a $(k-1)$-coloring $\phi_2$ of $G_2'$. Note that $\phi_2(z_1) \neq \phi_2(z_2)$ for otherwise setting $\phi_2(z) = \phi_2(z_1) = \phi_2(z_2)$ yields a $(k-1)$-coloring of $G_2$. Hence we may permute the colors of $\phi_2$ so $\phi_2(z_1) = \phi_1(x)$ and $\phi_2(z_2) = \phi_1(y)$. Then the coloring obtained by combining $\phi_1$ and $\phi_2$ is a $(k-1)$-coloring of $H$.
		\subparagraph{}
		So we may assume $H_1$ is not a strict subgraph of $G_1-xy$, i.e. $H_1 = G_1-xy$. Then $H_2$ is a strict subgraph of $G_2'$. By identifying $z_x$ and $z_y$ as $z$ in $H_2$ we obtain a strict subgraph of $G_2$. By criticality of $G_2$, this subgraph has a $(k-1)$-coloring $\phi_2$. By setting $\phi_2(z_x) = \phi_2(z_y) = \phi_2(z)$ we obtain a $(k-1)$-coloring of $H_2$ wherein $z_x$ and $z_y$ receive the same color. Invoking criticality of $G_1$, we obtain a $(k-1)$-coloring $\phi_1$ of $H_1$ ($=G_1-xy$). Observe that $\phi_1(x) = \phi_1(y)$, for otherwise $\phi_1$ is a $(k-1)$-coloring of $G_1$. But then we may permute the colors of $\phi_2$ so $\phi_2(z_1) = \phi_1(x)$ and $\phi_2(z_2) = \phi_1(y)$. Again the coloring obtained by combining $\phi_1$ and $\phi_2$ is a $(k-1)$-coloring of $H$, as desired.
	\paragraph{Splittability}
	Thus we have shown that $G$ is $k$-critical. It remains to verify that $G$ is splittable.
\end{proof}
\end{document}
