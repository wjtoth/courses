\documentclass{article}

\usepackage{apacite}
\usepackage[colorlinks = true,
linkcolor = red,
urlcolor  = blue,
citecolor = green,
filecolor = cyan,
anchorcolor = blue]{hyperref}
\usepackage{setspace}
\usepackage[margin=1in, bottom=1in, top=1in]{geometry}
\title{Teaching Dossier}
\author{W. Justin Toth}
\begin{document}
\maketitle
\newpage
\tableofcontents
\newpage
\section{Statement of Teaching Philosophy}
\paragraph{}
Mathematics is a rich, deep tapestry of fascinating interrelated concepts about shapes and numbers. Many passionate teachers of the subject, myself included, share this sentiment. As an instructor, my goal is to transmit this perspective to my students. I want to stoke their passion, ignite their curiosity, and instill lifelong confidence in their understanding. Towards these lofty goals I have adopted methods of teaching which encourages students to take greater ownership over the material and their own learning process.

The heart of the student-teacher relationship is the lecture, and I believe in transforming my lectures into an active dialogue between student and teacher to encourage greater engagement by students. During my lectures I alternate between explaining concepts and asking questions with a frequency that evenly divides time between me speaking and students speaking. For instance, when presenting a definition I immediately follow up asking students for an example which satisfies the definition, or during a proof I may begin by writing what we are trying to show and ask students what technique we should proceed with. I find this to be one of the most powerful techniques for teaching. It pushes students to actively engage with the material and also allows them to quickly perform self-diagnosis to see if they are absorbing the lecture content. It even helps me better understand my audience. I can quickly determine which students are easily understanding, which are struggling, which ones are not feeling confident about speaking, and which ones are giving their attention. This feedback allows me to adapt during the current and future lectures by changing pace, or stopping to clarify some important background concept I notice students are missing. Within the first couple weeks of any given term this method leads to significant changes in many students. Often in the first lecture students are hesitant to speak, and will only speak when called upon. As they get used to my style of teaching they become more confident in responding, and become more likely to volunteer ideas or ask questions without being prompted.

I believe doing is necessary for learning, and central to the act of doing mathematics is problem solving. I see independent problem solving as a skill which can be taught and I use problem solving as an opportunity to foster student creativity and increase their comfort in engaging with the unknown. To teach problem solving I work proto-typical problems into my lectures. When working these problems, whether I’m presenting the next step or asking the students for it, the thinking which led to the step is front and center. I present problem solving in a four step approach as outlined by the famous mathematician George Polya in his book “How to Solve It”: understand the problem, devise a plan, carry out the plan, and look back. This framework gives students structure and I find it severely cuts back on generic questions about not knowing what to do.  I see students reframing their questions more specifically in terms of where in the problem solving steps they are stuck, which in turn leads to a deeper understanding on their part, and makes it much easier to diagnose their challenges and provide meaningful feedback. As my students grow in confidence, I start using problems to provoke their curiousity. My lectures and assignments are peppered with optional exercises that highlight a subtle corner-case, interesting novelty, or surprising connection with material outside the course. This is an opportunity to broaden student horizons without overwhelming those who are already bogged down in coursework. Students who engage with this material will often come to office hours to discuss what they discovered, which leads to fascinating side discussions that enrich the course and capture student interest.

Ultimately student confidence hinges on their ability to recall material and identify where it is useful. Similarly, a major motivator for student passion is seeing how interconnected the material they are learning is with other material they have learned or will learn. Thus strengthening both passion and confidence can be done through putting material in context and forming a dense web of connections between different concepts. The basic organization of my courses, lectures, notes, and assignments supports this goal. It starts with the course syllabus where I outline the learning objectives for the course alongside a timeline of when these objectives will be achieved. At the start of a lesson I outline learning outcomes for that session, and review them again at the end of the lesson. Lectures themselves start with a review of previous material needed for the lecture, and end by describing what we learned that session and how we will need it in future sessions. Before exams I set aside lecture time to review the skills I expect students to have attained, and will be testing on that particular exam. Altogether this creates a pattern of reinforcement for the important concepts of the course, and supports students in drawing connections between the lectures over time. I dedicate special revision passes on my lecture notes and assignments to identifying connections with not only past and future lectures, but also with other courses the students may have taken in the past or will take in the future. This exercise is immensely valuable, not only for retention of material, but for motivation. I've seen tremendous improvements in student passion for the material when they come to see how the topics are situated within the bigger picture of what they are learning in their program.

I view teaching as a craft, and I am constantly honing my tools and methods to perform better in this craft. I consider myself incredibly lucky to have the opportunity to share my knowledge, understanding, and ultimately passion with my students. I strive to make the most of this opportunity every time I step in front of a class or open my door for office hours.

\section{Teaching Experience}

\subsection{Courses Taught}
\emph{CO $327$: Deterministic OR Models}

 Role: Instructor
 
 Term(s): Spring $2019$

This course is an applications-focused introduction to the field of operations research for non-specialist students, meaning those not enrolled in an Honours Mathematics program, at the University of Waterloo. Its core topics are linear and integer programming models, duality theory, sensitivity analysis, and cutting planes and branch and bound methods for solving integer programs. The instructor also has some discretion in additional topics to cover from operations research after covering the aforementioned areas. I was the sole instructor for this course in Spring $2019$ and was fully responsible for all areas of the course including the syllabus, course materials, office hours, exams, and $80$ minute lectures which occurred twice per week. 

Course notes from previous offerings were available to me, but were considerably out of date. I decided to rewrite the course notes from scratch. I also created new assignments and exams for the course to align with the updated course notes I had written. For the additional topics, traditionally dynamic programming or stochastic optimization is taught. In seeking feedback from my class, I found that they were not interested in these topics and I decided to teach them about stable matching: a Nobel Prize winning method of assigning goods in two-sided markets which had never been taught in the course before but was extremely relevant and the students were enthusiastic about learning.

\subsection{Teaching Assistantships}
\paragraph{Graduate TA}Over the course of my graduate studies at the University of Waterloo I have had the priviledge of being a teaching assistant for many different courses. The standard duties of a TA in our department, Combinatorics and Optimization, involve grading, proctoring exams, and holding office hours to support students seeking help. Below I have described the courses to which I was assigned a teaching assistantship, as well as any additional duties I was responsible for in each course.
\bigskip

\noindent$\cdot$\emph{CO $456$: Introduction to Game Theory}

Term(s): Fall $2019$, Fall $2018$, Fall $2017$, Fall $2016$

This fourth year course offers students a mathematically rigorous introduction to the field of game theory. It covers combinatorial games, strategic games, the existence and computation of Nash Equilibria, fixed point theorems, mechanism design, and cooperative game theory. In addition to the standard TA duties, I was a guest lecturer for this course in $2019$ teaching a series of three lectures on Sperner's Lemma, Brouwer's Fixed Point Theorem, and Nash's Theorem. I was also a guest lecturer in $2018$, giving a lecture on correlated equilibria. 

In the $2017$ and $2016$ offerings of the course there was a programming project component where the students would work in groups, using their knowledge of game theory to code AI competitors for a simulated tournament. During these offerings I was responsible for programming the tournament simulation, and making sure the competitions which were held throughout the term ran smoothly.

\noindent$\cdot$\emph{CO $353$: Computational Discrete Optimization}

Term(s): Winter $2019$, Winter $2018$, Winter $2017$, Winter $2016$

CO $353$ covers the algorithmic aspects of discrete optimization. This includes problem formulations, greedy algorithms, local-search heuristics, approximation algorithms, linear programming duality, cutting planes, branch-and-bound, column generation, dynamic programming, problem reductions and NP-hardness.

\noindent$\cdot$\emph{CO $351$: Network Flow Theory}

Term(s): Spring $2018$

Network Flow Theory covers the fundamental properties and algorithms for flows on directed graphs. It covers shortest path problems, the max-flow min-cut theorem and applications, minimum cost flow problems, network simplex and primal-dual algorithms.

\noindent$\cdot$\emph{CO $250$: Introduction to Optimization}

Term(s): Spring $2017$

This second year course is a typically an undergraduate's first exposure to optimization. The central focus of this course is to teach students linear programming theory and methods for solving linear programs.

\paragraph{}
Early in my graduate career I was assigned teaching assistant positions for two faculty of mathematics courses given to first year students at the University of Wateroo. In addition to the standard TA duties, for theses courses I was responsible for running a weekly tutorial where I would prepare problems for the students to practice in class, demonstrate key techniques, and provide reviews of critical course material.
\bigskip

\noindent$\cdot$\emph{MATH $128$: Calculus $2$ for Sciences}

Term(s): Spring $2016$

This is the second course in the calculus sequence at Waterloo for science students, i.e.\ students in physics, chemisty, biology, and economics. This course is typically the last calculus course for many science students and tries to cover a broad range of topics. It teachs the evaluation of integrals, applications to volumes and arc lengths, differential equations, sequences and series, Taylor's Theorem, parametric and vector representations of curves, and polar coordinates in the plane.

\noindent\emph{MATH $136$: Linear Algebra $1$}

Term(s): Fall $2018$.

This is the first linear algebra course offered to Honours Mathematics students at the University of Waterloo. It covers systems of linear equations, matrix algebra, elementary matrices, computational issues. It also covers Real $n$-space, vector spaces and subspaces, basis and dimension, rank of a matrix, linear transformations, and matrix representations, determinants, eigenvalues and diagonalization.

\paragraph{Undergraduate TA}During my undergraduate degree at the University of Windsor I had the opportunity to be a teaching assistant during my final year of study. I was assigned to TA the introductory linear algebra course for first years in the Fall term of $2014$, and the introductory course on mathematical proof in the Winter term of $2015$. Both courses saw me giving a weekly tutorial based on the assigned problem set for that week. I was also responsible during this time for working in the University's tutoring centre. My role there was to provide one-on-one or small group support to any students who came in with questions about math courses at the university. This opportunity was an interesting challenge because it required a broad understanding of the courses offered by the university that particular term, since any student studying math could approach me expecting to receive help with their coursework.

\subsection{Guest Lectures}
\paragraph{Introduction to Game Theory}
As I discussed in Section $2.2$ I have had the opportunity to give guest lectures in CO $456$, Waterloo's mathematically rigorous introduction to game theory. In the Fall $2018$ offering, I gave a single $80$ minute lecture on the topic of correlated equilibria. This was the first time that particular topic was taught in the course. As such, I was responsible for designing not only the lesson plan from scratch, but also preparing suitable exam and assignment problems related to the material. I used a post-test technique in the lecture where I saved the last ten minutes of instruction time for the students to solve a hands-on problem in class of computing a correlated equilibrium for a small strategic game. The head instructor of the course, who was observing my lecture, commented that he found the technique to be a very engaging way for the students to review the material and he would be interested in trying it himself in the future.

I also had an opportunity to give a series of three $50$ minute guest lectures during the Fall $2019$ offering of CO $456$ during a week where the head instructor was travelling for a conference. They had just finished up a section of course material, and I was given free reign to choose the next topic. I chose Nash's Theorem that every finite strategic game has a mixed Nash equilibrium. The first two lectures concentrated on the fundamental tools needed to prove the theorem: Sperner's Lemma and Brouwer's Fixed Point Theorem. The last lecture discussed the combination of these two results to obtain Nash's Theorem. I used a strategy of providing a high-level outline at the start and end of each lecture so that students could maintain a big picture view of the proof of Nash's Theorem as we progressed through the technical details throughout the week.

\paragraph{Microteaching} During my Master's degree I took a professional teaching development program through the University of Waterloo's Centre for Teaching Excellence called the Fundamentals of University Teaching program. As part of this program, participants were required to give a series of three $15$ minute mini-lectures to their peers and the facilitators of the program. The lectures were supposed to highlight three different areas: a subject suitable for first year students of your discipline, a subject related to your area of research, and a specific teaching method. I chose the Gaussian elimination process for solving linear systems from a first year linear algebra course, the prisoner's dilemma from game theory, and the Think-Pair-Share technique respectively for my three lectures. These sessions were a valuable learning experience. The feedback from peers and teaching experts was invaluable, and the tight time constraints really helped me hone my lesson planning and time management skills in the classroom. 
\subsection{Outside the Classroom}
\paragraph{TORCH}Teaching is a major part of my life, and I believe an important part of teaching is inspiring young students to pursue higher education. The Operations Research Challenge, TORCH for short, is a contest given to high school students which focuses on solving algorithmic operations research problems. The contest is hosted simultaneously at the Universities of Waterloo, Toronto, and Concordia. As an outreach program, it aims to promote awareness and interest in operations research among high school students. I have been a volunteer for the annual contest in $2018$ and $2019$. In this role I was responsible for giving contest students real time feedback on their proposed solutions, as well as discussing in between sessions what studying operations research in university is like and encouraging students to pursue studies in STEM.

In $2020$ I am transitioning into an executive role on Waterloo's organizing team. My role will be to lead to the creative design of problem sets for the contest. I will be responsible for creating contest problems which are challenging, engaging, relevant to student interests, and accessible to students of all background levels. In addition to creating problems myself, I will be overseeing other question writers to ensure a high standard of quality in questions provided for the contest.

\paragraph{Brazilian Jiu-Jitsu}
My love of teaching extends outside of my profession and into my hobbies as well. I have been practicing the martial art of Brazilian Jiu-Jitsu for over $10$ years. My experience in the sport found me in a position of seniority among the students who train at my current gym. This seniority earned me a position as an assistant instructor at Alliance Fitness in Waterloo Ontario. Since $2019$ I have been teaching a weekly beginners class at that gym. I have been able to transfer my understanding of pedagogy, and my teaching experience in academia, to this new setting to quickly become one of the top performing instructors in the gym. 

Pedagogy in teaching sports to hobbyists lags far behind the understanding of pedagogy present in teaching research. Even implementing simple things like designing a syllabus for the beginner's course and preparing lessons plans makes for a marked improvement over the status quo in the area. My classes consistently have twice as high of an attendance over comparable beginner class time slots (workday evenings) held by different instructors in the gym. Furthermore, the rate of technique aquisition among my students is expectionally high. I have shortened the expected time for learning certain sport specific skills from many months to a couple of weeks in our beginner class.

\section{Teaching Strategies}
%reread philosophy statement and pick out teaching strategies to describe in detail
%write a stream of consciousness about teaching strategy that you can later organize thoughts from
%consider whether it is better to organize by teaching context or by technique

\section{Evaluation of Teaching}
I find feedback on my teaching to be tremendously valuable. It is the easiest way for me to identify potential avenues for improvement, while at the same time motivating me by highlighting strong points of my teaching style. I have organized my evaluations below into two categories: those from students in courses I have taught, and those from peers and teaching professionals observing my lectures.
\subsection{Student Evaluations}
%Chart out survey responses for CO 327 with survey adjectives for context for scores (some here, rest in appendix)
%Student comment strong points
%Student comment targets for change
\subsection{Teaching Observations}
%highlights from 1st observation
%highlights from 2nd observation with a narrative about improving

\section{Professional Development}
%FUT
%Math Graduate Teaching Seminar
%CUT
\section{Future Goals}
%Write a textbook
%Teach a fourth year course
%Teach a topics course
%Become more invovled in teaching 1st and 2nd years
%Head instructor for a course with multiple sections
%Graduate and URA supervision

\appendix
\section*{Appendices}
%Full evaluations
%Observation Reports
%Sample Lesson Plan from guest lecture
%Sample Assignment from CO 327
%Sample Lesson Plan from CO 327
\section{}


\end{document}