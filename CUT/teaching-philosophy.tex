\documentclass{article}

\usepackage{apacite}
\usepackage[margin=1in, bottom=1in, top=1in]{geometry}
\title{Statement of Teaching Philosophy - Draft}
\author{W. Justin Toth}
\begin{document}
\maketitle
\newpage
\section*{}
\paragraph{}
The most lucid way to explain my beliefs about to teaching is to explain my beliefs about learning. This is because my teaching philosophy is centred around optimally assisting students on their path to learning. To begin, let us consider why a student should study mathematics. People give many different answers to this question. Many mathematicians will say it should be studied for no other reason than its intrinsic beauty. Engineers and Scientists may say that mathematics teaches many useful formulae and algorithms that make excellent tools in the pursuit of science and engineering. Both of these reasons are very true, but there is an oft-overlooked third reason. Mathematics is fundamentally about developing abstract critical thinking, and unparalleled problem solving skills. 

My teaching philosophy is deeply rooted in the three reasons above that a student will be attending my courses. Problem solving skills are central and universally valuable to all students and thus they lie at the core of my approach. From the very first lecture in my classes, students will be solving problems. Students retain material longer, and achieve deeper levels of learning when they play an active role in obtaining information. The fundamental flow of my lectures plays to this observation. They will often begin by introducing some very elementary definitions, and then immediately posing a question to the class. This forces students to engage their critical thinking early in the lecture, and to be active participants throughout the class meeting. It sets a precedent that the experience in my courses will be a dialogue. I will be spending as much time asking students questions, as they will be asking me. With larger classes I will impose more formal time points at which students break into small groups to discuss the problems posed, before bringing their attention back to the front. To support students' growth as problem solvers throughout the course, when it is time for me to give explanations to them I concentrate on the logical structure and manner of reasoning that allows me to reach the solution as much as I do the content of the solution itself.

Mathematics is full of a plethora of beautiful, deep, and mysterious results. I seek to instill an appreciation for this quality of mathematics, and the occumpanying sense of wonder, in my students. This goal is challenging to achieve. My strategy is to communicate intense passion. When describing a result that I think is particularly beautiful I will be unreserved in my appreciation for it. Quotes like ``The fact that is this true is really incredible!'', or  ``When I first learned this I simply couldn't believe it.'', or ``I think the pattern made by the solutions to this equation look mesmerizing.'' are not uncommon in my classroom. I think expressing emotional attachment to the material helps students get excited about the content, and helps to humanize myself as their instructor.

\paragraph{}
Something about how I address students motivated by applications.
\paragraph{}
I believe that one of the ultimate goals of education should be for students to become independent learners. Facilitating the growth of students so that one day they can become my peers is an important part of my approach to teaching. Towards this end, depending on the level of the class, I will structure aspects of the course to encourage greater independence among my students. For example, instead of giving a lecture on how to write some piece of code for a project, I will email students a link to the documentation, encourage them to read it, try to install the software, and come to office hours if they are stuck. Similarly in third year and up courses I will not provide solutions to assignments or exam practice problems. Instead I will use online discussion boards and incentivize students (sometimes with the aid of bonus marks) to post their solutions there after the class. Often this results in lively discussion, where students debate what the correct answers should be, challenging each other, explaining why did things a certain way, and learning all the more in the process. Altogether my teaching philosophy is to train independent problem solvers, adept at critical thinking, with a deep appreciation of the instrinsic beauty of their subject and a toolkit of incredibly useful mathematics.


\end{document}