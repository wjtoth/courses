\documentclass{article}

\usepackage{apacite}
\usepackage[colorlinks = true,
linkcolor = red,
urlcolor  = blue,
citecolor = green,
filecolor = cyan,
anchorcolor = blue]{hyperref}
\usepackage{setspace}
\usepackage{pdfpages}
\usepackage[margin=1in, bottom=1in, top=1in]{geometry}
\title{Teaching Dossier}
\author{W. Justin Toth}
\begin{document}
\maketitle
\newpage
\tableofcontents
\newpage
\section{Statement of Teaching Philosophy}
\paragraph{}
Mathematics is a rich, deep tapestry of fascinating interrelated concepts about shapes and numbers. Many passionate teachers of the subject, myself included, share this sentiment. As an instructor, my goal is to transmit this perspective to my students. I want to stoke their passion, ignite their curiosity, and instill lifelong confidence in their understanding. Towards these lofty goals I have adopted methods of teaching which encourages students to take greater ownership over the material and their own learning process.

The heart of the student-teacher relationship is the lecture, and I believe in transforming my lectures into an active dialogue between student and teacher to encourage greater engagement by students. During my lectures I alternate between explaining concepts and asking questions with a frequency that evenly divides time between me speaking and students speaking. For instance, when presenting a definition I immediately follow up asking students for an example which satisfies the definition, or during a proof I may begin by writing what we are trying to show and ask students what technique we should proceed with. I find this to be one of the most powerful techniques for teaching. It pushes students to actively engage with the material and also allows them to quickly perform self-diagnosis to see if they are absorbing the lecture content. It even helps me better understand my audience. I can quickly determine which students are easily understanding, which are struggling, which ones are not feeling confident about speaking, and which ones are giving their attention. This feedback allows me to adapt during the current and future lectures by changing pace, or stopping to clarify some important background concept I notice students are missing. Within the first couple weeks of any given term this method leads to significant changes in many students. Often in the first lecture students are hesitant to speak, and will only speak when called upon. As they get used to my style of teaching they become more confident in responding, and become more likely to volunteer ideas or ask questions without being prompted.

I believe doing is necessary for learning, and central to the act of doing mathematics is problem solving. I see independent problem solving as a skill which can be taught and I use problem solving as an opportunity to foster student creativity and increase their comfort in engaging with the unknown. To teach problem solving I work proto-typical problems into my lectures. When working these problems, whether I’m presenting the next step or asking the students for it, the thinking which led to the step is front and center. I present problem solving in a four step approach as outlined by the famous mathematician George Polya in his book “How to Solve It”: understand the problem, devise a plan, carry out the plan, and look back. This framework gives students structure and I find it severely cuts back on generic questions about not knowing what to do.  I see students reframing their questions more specifically in terms of where in the problem solving steps they are stuck, which in turn leads to a deeper understanding on their part, and makes it much easier to diagnose their challenges and provide meaningful feedback. As my students grow in confidence, I start using problems to provoke their curiousity. My lectures and assignments are peppered with optional exercises that highlight a subtle corner-case, interesting novelty, or surprising connection with material outside the course. This is an opportunity to broaden student horizons without overwhelming those who are already bogged down in coursework. Students who engage with this material will often come to office hours to discuss what they discovered, which leads to fascinating side discussions that enrich the course and capture student interest.

Ultimately student confidence hinges on their ability to recall material and identify where it is useful. Similarly, a major motivator for student passion is seeing how interconnected the material they are learning is with other material they have learned or will learn. Thus strengthening both passion and confidence can be done through putting material in context and forming a dense web of connections between different concepts. The basic organization of my courses, lectures, notes, and assignments supports this goal. It starts with the course syllabus where I outline the learning objectives for the course alongside a timeline of when these objectives will be achieved. At the start of a lesson I outline learning outcomes for that session, and review them again at the end of the lesson. Lectures themselves start with a review of previous material needed for the lecture, and end by describing what we learned that session and how we will need it in future sessions. Before exams I set aside lecture time to review the skills I expect students to have attained, and will be testing on that particular exam. Altogether this creates a pattern of reinforcement for the important concepts of the course, and supports students in drawing connections between the lectures over time. I dedicate special revision passes on my lecture notes and assignments to identifying connections with not only past and future lectures, but also with other courses the students may have taken in the past or will take in the future. This exercise is immensely valuable, not only for retention of material, but for motivation. I've seen tremendous improvements in student passion for the material when they come to see how the topics are situated within the bigger picture of what they are learning in their program.

I view teaching as a craft, and I am constantly honing my tools and methods to perform better in this craft. I consider myself incredibly lucky to have the opportunity to share my knowledge, understanding, and ultimately passion with my students. I strive to make the most of this opportunity every time I step in front of a class or open my door for office hours.

\section{Teaching Experience}

\subsection{Courses Taught}
\emph{CO $327$: Deterministic OR Models}

 Role: Instructor
 
 Term(s): Spring $2019$

This course is an applications-focused introduction to the field of operations research for non-specialist students, meaning those not enrolled in an Honours Mathematics program, at the University of Waterloo. Its core topics are linear and integer programming models, duality theory, sensitivity analysis, and cutting planes and branch and bound methods for solving integer programs. The instructor also has some discretion in additional topics to cover from operations research after covering the aforementioned areas. I was the sole instructor for this course in Spring $2019$ and was fully responsible for all areas of the course including the syllabus, course materials, office hours, exams, and $80$ minute lectures which occurred twice per week. 

Course notes from previous offerings were available to me, but were considerably out of date. I decided to rewrite the course notes from scratch. I also created new assignments and exams for the course to align with the updated course notes I had written. For the additional topics, traditionally dynamic programming or stochastic optimization is taught. In seeking feedback from my class, I found that they were not interested in these topics and I decided to teach them about stable matching: a Nobel Prize winning method of assigning goods in two-sided markets which had never been taught in the course before but was extremely relevant and the students were enthusiastic about learning.

\subsection{Teaching Assistantships}
\paragraph{Graduate TA}Over the course of my graduate studies at the University of Waterloo I have had the priviledge of being a teaching assistant for many different courses. The standard duties of a TA in our department, Combinatorics and Optimization, involve grading, proctoring exams, and holding office hours to support students seeking help. Below I have described the courses to which I was assigned a teaching assistantship, as well as any additional duties I was responsible for in each course.
\bigskip

\noindent\emph{CO $456$: Introduction to Game Theory}

Term(s): Fall $2019$, Fall $2018$, Fall $2017$, Fall $2016$

This fourth year course offers students a mathematically rigorous introduction to the field of game theory. It covers combinatorial games, strategic games, the existence and computation of Nash Equilibria, fixed point theorems, mechanism design, and cooperative game theory. In addition to the standard TA duties, I was a guest lecturer for this course in $2019$ teaching a series of three lectures on Sperner's Lemma, Brouwer's Fixed Point Theorem, and Nash's Theorem. I was also a guest lecturer in $2018$, giving a lecture on correlated equilibria. 

In the $2017$ and $2016$ offerings of the course there was a programming project component where the students would work in groups, using their knowledge of game theory to code AI competitors for a simulated tournament. During these offerings I was responsible for programming the tournament simulation, and making sure the competitions which were held throughout the term ran smoothly.
\bigskip

\noindent\emph{CO $353$: Computational Discrete Optimization}

Term(s): Winter $2019$, Winter $2018$, Winter $2017$, Winter $2016$

CO $353$ covers the algorithmic aspects of discrete optimization. This includes problem formulations, greedy algorithms, local-search heuristics, approximation algorithms, linear programming duality, cutting planes, branch-and-bound, column generation, dynamic programming, problem reductions and NP-hardness.
\bigskip

\noindent\emph{CO $351$: Network Flow Theory}

Term(s): Spring $2018$

Network Flow Theory covers the fundamental properties and algorithms for flows on directed graphs. It covers shortest path problems, the max-flow min-cut theorem and applications, minimum cost flow problems, network simplex and primal-dual algorithms.
\bigskip

\noindent\emph{CO $250$: Introduction to Optimization}

Term(s): Spring $2017$

This second year course is a typically an undergraduate's first exposure to optimization. The central focus of this course is to teach students linear programming theory and methods for solving linear programs.

\paragraph{}
Early in my graduate career I was assigned teaching assistant positions for two faculty of mathematics courses given to first year students at the University of Wateroo. In addition to the standard TA duties, for theses courses I was responsible for running a weekly tutorial where I would prepare problems for the students to practice in class, demonstrate key techniques, and provide reviews of critical course material.
\bigskip

\noindent$\cdot$\emph{MATH $128$: Calculus $2$ for Sciences}

Term(s): Spring $2016$

This is the second course in the calculus sequence at Waterloo for science students, i.e.\ students in physics, chemisty, biology, and economics. This course is typically the last calculus course for many science students and tries to cover a broad range of topics. It teachs the evaluation of integrals, applications to volumes and arc lengths, differential equations, sequences and series, Taylor's Theorem, parametric and vector representations of curves, and polar coordinates in the plane.
\bigskip

\noindent\emph{MATH $136$: Linear Algebra $1$}

Term(s): Fall $2018$.

This is the first linear algebra course offered to Honours Mathematics students at the University of Waterloo. It covers systems of linear equations, matrix algebra, elementary matrices, computational issues. It also covers Real $n$-space, vector spaces and subspaces, basis and dimension, rank of a matrix, linear transformations, and matrix representations, determinants, eigenvalues and diagonalization.

\paragraph{Undergraduate TA}During my undergraduate degree at the University of Windsor I had the opportunity to be a teaching assistant during my final year of study. I was assigned to TA the introductory linear algebra course for first years in the Fall term of $2014$, and the introductory course on mathematical proof in the Winter term of $2015$. Both courses saw me giving a weekly tutorial based on the assigned problem set for that week. I was also responsible during this time for working in the University's tutoring centre. My role there was to provide one-on-one or small group support to any students who came in with questions about math courses at the university. This opportunity was an interesting challenge because it required a broad understanding of the courses offered by the university that particular term, since any student studying math could approach me expecting to receive help with their coursework.

\subsection{Guest Lectures}
\paragraph{Introduction to Game Theory}
As I discussed in Section $2.2$ I have had the opportunity to give guest lectures in CO $456$, Waterloo's mathematically rigorous introduction to game theory. In the Fall $2018$ offering, I gave a single $80$ minute lecture on the topic of correlated equilibria. This was the first time that particular topic was taught in the course. As such, I was responsible for designing not only the lesson plan from scratch, but also preparing suitable exam and assignment problems related to the material. I used a post-test technique in the lecture where I saved the last ten minutes of instruction time for the students to solve a hands-on problem in class of computing a correlated equilibrium for a small strategic game. The head instructor of the course, who was observing my lecture, commented that he found the technique to be a very engaging way for the students to review the material and he would be interested in trying it himself in the future.

I also had an opportunity to give a series of three $50$ minute guest lectures during the Fall $2019$ offering of CO $456$ during a week where the head instructor was travelling for a conference. They had just finished up a section of course material, and I was given free reign to choose the next topic. I chose Nash's Theorem that every finite strategic game has a mixed Nash equilibrium. The first two lectures concentrated on the fundamental tools needed to prove the theorem: Sperner's Lemma and Brouwer's Fixed Point Theorem. The last lecture discussed the combination of these two results to obtain Nash's Theorem. I used a strategy of providing a high-level outline at the start and end of each lecture so that students could maintain a big picture view of the proof of Nash's Theorem as we progressed through the technical details throughout the week.

\paragraph{Microteaching} During my Master's degree I took a professional teaching development program through the University of Waterloo's Centre for Teaching Excellence called the Fundamentals of University Teaching program. As part of this program, participants were required to give a series of three $15$ minute mini-lectures to their peers and the facilitators of the program. The lectures were supposed to highlight three different areas: a subject suitable for first year students of your discipline, a subject related to your area of research, and a specific teaching method. I chose the Gaussian elimination process for solving linear systems from a first year linear algebra course, the prisoner's dilemma from game theory, and the Think-Pair-Share technique respectively for my three lectures. These sessions were a valuable learning experience. The feedback from peers and teaching experts was invaluable, and the tight time constraints really helped me hone my lesson planning and time management skills in the classroom. 
\subsection{Outside the Classroom}
\paragraph{TORCH}Teaching is a major part of my life, and I believe an important part of teaching is inspiring young students to pursue higher education. The Operations Research Challenge, TORCH for short, is a contest given to high school students which focuses on solving algorithmic operations research problems. The contest is hosted simultaneously at the Universities of Waterloo, Toronto, and Concordia. As an outreach program, it aims to promote awareness and interest in operations research among high school students. I have been a volunteer for the annual contest in $2018$ and $2019$. In this role I was responsible for giving contest students real time feedback on their proposed solutions, as well as discussing in between sessions what studying operations research in university is like and encouraging students to pursue studies in STEM.

In $2020$ I am transitioning into an executive role on Waterloo's organizing team. My role will be to lead to the creative design of problem sets for the contest. I will be responsible for creating contest problems which are challenging, engaging, relevant to student interests, and accessible to students of all background levels. In addition to creating problems myself, I will be overseeing other question writers to ensure a high standard of quality in questions provided for the contest.

\paragraph{Brazilian Jiu-Jitsu}
My love of teaching extends outside of my profession and into my hobbies as well. I have been practicing the martial art of Brazilian Jiu-Jitsu for over $10$ years. My experience in the sport found me in a position of seniority among the students who train at my current gym. This seniority earned me a position as an assistant instructor at Alliance Fitness in Waterloo Ontario. Since $2019$ I have been teaching a weekly beginners class at that gym. I have been able to transfer my understanding of pedagogy, and my teaching experience in academia, to this new setting to quickly become one of the top performing instructors in the gym. 

Pedagogy in teaching sports to hobbyists lags far behind the understanding of pedagogy present in teaching research. Even implementing simple things like designing a syllabus for the beginner's course and preparing lessons plans makes for a marked improvement over the status quo in the area. My classes consistently have twice as high of an attendance over comparable beginner class time slots (workday evenings) held by different instructors in the gym. Furthermore, the rate of technique aquisition among my students is expectionally high. I have shortened the expected time for learning certain sport specific skills from many months to a couple of weeks in our beginner class.

\section{Teaching Strategies}
%reread philosophy statement and pick out teaching strategies to describe in detail:
%       questioning strategies in lecture
%       outlining, learning outcomes, reviews
%       problem solving frameworks
%write a stream of consciousness about teaching strategy that you can later organize thoughts from
%consider whether it is better to organize by teaching context or by technique
\subsection{In Lecture}
\paragraph{}
The lecture is the fundamental unit of instruction in a university setting. As such I place a great emphasis on preparing my lectures in a way that enables my students to extract as much benefit as possible from each lecture. For me, the purpose of a lecture is to give my students learning outcomes, sufficient motivation to attain them, and all the tools necessary to make achieving those learning outcomes as easy as possible.

\paragraph{Learning Outcomes}Everything I do in lecture is centre around learning outcomes. For each lecture, depending on time and where we are in the course, I will identify one to four learning outcomes I want the students to get from the lecture. I will write them down on the leftmost board in the room, leave them up for the entire session, and check them off as we go. My learning outcomes are always specific, descriptive, and measurable. For example, not ``learn Braess's paradox'' but ``Braess's paradox: demonstrate how adding a road to a traffic network can increase total congestion''. Structuring the session around learing outcomes and presenting them at the start serves to prime students' brains for learning, clarify expectations for them, and provide a helpful structure to the lecture allowing students to organize their thoughts and measure progress.
\paragraph{Past, Present, and Future} Each lecture does not exist in a vacuum. Typically their is a lecture which will proceed it, and  a lecture which will follow it. When preparing a lecture, I reflect on which content from previous lectures will be needed to achieve the day's learning outcomes, and where the learning outcomes of the day will be needed in future lectures. At the start of lecture I add bullets to the outline containing the learnign outcomes describing the skills from previous lectures which will be relevant today, and bullets describing how the content of the day will feed into future goals in the course. This serves to put the material in context, which supports student retention. It allows them to form connections between what they learned last class, or last week, or maybe even last month and what they are working on today. Furthermore, the mention of future goals motivates them to study the material of this session and mentally prepares them for what is coming next. 

With the past, present, and future learning outcomes fleshed out, my outline is complete. I will make reference to this outline throuhgout the lecture, as the relevant pieces come up. Further, I save time towards the end of the session to review the entire outline. Allowing me to once again summarize the learning outcomes of the day and how they fit into the bigger picture of the course.

\paragraph{Pre-test}Directly after the outline of learning outcomes, I will usually continue the lecture with a pre-test activity. The goal of a pre-test is to evaluate, with no stakes, the baseline of comprehension for the material that will be covered in the day's lecture. In cases where the lecture exists in a series on related material for which I already have a good feel for the level of understanding my students have I will skip the pre-test. That said, the pre-test is a valuable method to help me calibrate the level of my instruction to the level of the students in the room, and I only omit it as a conscious decision when I feel my teaching level is well calibrated to the students already. The pre-test does not need to take more than a couple minutes. It can be as simple as an informal show-of-hands poll on whether students have heard of the main concept of discussion today and to what extent. That is the method I favour and use most often. Another option is to give a short exercise at the start of class. This method is somewhat more time consuming. I only use it when I need to be absolutely certain the students are at a certain level of learning attainment. Say for example, I am teaching a very technical proof that day and want to be sure that students have grasped the critical lemma from the previous lecture. The pre-test is only as valuable as my willingness to adapt based on its results. This means building in some flexibility into my lecturing style. If I observe the majority of students are lacking in background, then I may have to spend more time on the background during that lecture and not get as far into the content I planned. Conversely, if the students are exceptionally prepared, I need to be willing to skip content they have previously seen and are comfortable with and be prepared to go deeper into the material than initially planned.

\paragraph{Post-test}The mirror image of the pre-test is the post-test. Partially I use the post-test to signal to myself how well the students absorbed the material from this session, but more importantly I use the post-test to signal to students how well they absorbed the material from this session. A post-test is a small exercise performed at the end of a lecture to evaluate attainment of the desired learning outcomes of that lecture. In practice, I want post-tests to be fast, taking around five minutes, and they usually will target the lower levels of Bloom's taxonomy. Usually they will feature a calculation using a technique from the day's lecture, or ask students to give another example of a concept presented today. In more proof-focused courses, they may be asked to observe a simple corollary which directly follows from a lemma presented in class. The post-tests are meant to be done individualy or in small groups. Some groups finish quickly and are free to leave early. Other groups realize they did not understand something from the lecture they initially thought they did, and it triggers questions that I can immediately address for them.

\paragraph{Questioning Strategies}Between the pre-test and the outline review lies the bulk of the lecture time to be used to deliver content. I like to give my lectures using a high density of questions for the students. This is a simple, efficient, and effective way for me create a highly interactive learning environment within the lecture. Sometimes it can be difficult to get students to open up and get used to the amount of spoken input I expect from them during the lecture. I have developed some techniques for increasing my success rate with using frequent student questioning. 

The first is to set the precedent early. I mean this in two senses. I start using questioning from the first day of lecture, and make sure to heavily ask student questions every single lecture. This is much easier than switching to this technique part way through the semester, or not doing for a few lectures, then having an interactive lecture. Once the students adapt to the routine of the interactive lecture they will be more comfortable answering questions, and even come to expect to be given the opportunity for input. I also mean to set a precedent on the lecture scale. Once I start delivering the lecture material, I am looking for the earliest opportunity to throw a question at the students. Usually this is immediately after the first definition or example I present. This gets the students focused on actively engaging with the material and responding early in the lecture, and builds momentum that can be carried through the entire session. As an instructor, when I get in the habit of constantly looking for the nearest opportunity to ask a question, I find my lectures are far more interesting for the students and their depth of understanding increases significantly.

On a question-by-question level it can be difficult to coax responses out of students. It is important that I ensure I wait sufficiently long for students to respond. I use a ten second count in my head, or take a sip of water, to make sure I'm pausing long enough for a student to both work out the answer and work up the confidence to respond. Most of the time no hands go up immediately, but after a waiting period some hands will go up. Although sometimes students are not responding because they genuinely do not know the answer. When I get a sense this is happening, I will break the question down for them into smaller and easier parts. At some level I will find something simple enough that someone in the class will know the answer and be willing to respond. I prefer to not simply give the answer and move on. Although it may feel like the best thing to do to preserve momentum and save time, the fact that students are not responding means we have identified a gap in their understanding and it is important to drill down on this gap and help students bridge it piece by piece.

\subsection{Outside Lecture}
\paragraph{}
While time in lecture is critically important for student development the vast majority of student time is spent outside of lecture. I encourage students to continue learning outside of lecture by providing resources for them to ask questions and receive feedback, and by providing compelling assignment problems which encourage constant thinking and revision of the material. While I believe in using assignments and projects which require students to work outside of lecture time, I believe it is important to recognize that my courses are not the only courses students are taking during the term. Therefore a big component of my approach to teaching and learning outside of the lecture is balancing my desire to keep have students keep the course material front of mind on regular basis with the need to be respectful of my students' time and busy schedules.

\paragraph{} Opt-in engagement is a great way to give students learning opportunities outside the lecture environment without overloading there schedules. The classic office hours are an example of this. The great benefit of office hours is that they allow me to give attending students individualized attention and to target their specific needs. Office hours are great for clearing up student questions, but they also present an even greater opportunity to spur student interest in the subject. After clearing up student questions I try to take the opportunity to engage in conversation with the student about why they are taking the course, what their interests are, etc. This often will lead to interesting discussions that run well outside the scope of a particular course. These discussions are nevertheless incredibly important, since they help me learn about student goals (and give them a chance to articulate their goals) and let me open the student's eyes to the wider possibilities of what the subject can offer.

Office hours are not always convenient for students, and it's rare that timeslots can be found which work for every student. Some students are also anxious about talking face to face with their professor, and might prefer to discuss course material with other students. To faciliate this I use online discussion forums such as Piazza with my students. The teaching assistants and myself will monitor the forum to provide authoritative answers to student questions, but usually this amounts to nothing more than confirming a well-written reply by another student is in fact correct. These online forums help create a constant conversation between students on the course material while not putting too much burden on their schedules. To faciliate and encourage discussions, I will use the forums to have students create assignment solutions. After an assignment has been completed, a topic on the discusion board will appear for each question on the assignment. In that topic, the students are encouraged to discuss and crowdsource together the solution to the assignment problem. For more intricate or subtle questions I have found these topics to lead to a lot of stimulating conversation amongst the students.


\paragraph{}
\subsection{Assessment}
\paragraph{}
My approach to assessment is based not only only my experience as a teacher but on my understanding of the research literature. During the CUT program, I wrote a research paper surveying the academic literature on how exam formats impact student learning outcomes. Some interesting findings were synthesized from this paper which I now incorporate into my approach to student assessment.
\paragraph{}
Written midterms and final exams are fundamental evaluations which occur during courses I teach. They are usually cummulative and are worth a significant portion of a student's final grade. Naturally this means that many students will experience so-called test anxiety while preparing for and taking these exams. The literature show that offering students open book exams, i.e.\ exams where they are allowed to reference textbooks and/or course notes, significantly reduces test anxiety. Unfortunately the literature also shows that open book exams lead to significantly lower long term retention of course material than closed book exams. Interestingly, there is a way to obtain the same test anxiety reducing benefits of an open book exam while maintaining the long term retention of a closed book exam. This way is the cheatsheet exam, which I use in all my courses. For exams which do not have a predetermined format decided by institutional policy I allow students to bring into the exam a small cue card on which they can write whatever they want. The literature shows that this format has comparable reductins in test anxiety among students as an open book exam, while also encouraging postive study habits which lead to long term retention of the material since students are forced to prioritize the information they write on their cue card.
\paragraph{}
Alongside written exams I like to encorporate an oral exam component into my courses. This is implemented in an interview format, where a student is given a topic present on, such as a significant theorem from the course, they then prepare their presentation and schedule with me a time where they give the presentation and I ask questions to probe the depths of their understanding. I like to give students a list of topics from which they can engage in self-directed study for this oral exam. The research literature shows that oral exams not only improve student confidence, encourage effective study habits, and strengthen communication skills, but that performance on oral exams is only weakly correlated with performance on written exams. This suggests that they test a different set of skills. In the interest of celebrating the strengths of a diverse student body I strive to incorporate exam formats like oral exams which give students who may struggle with written exams an opportunity to demonstrate their knowledge and learning attainment.
\paragraph{}
Whatever plans students may have after graduation they will likely need to be able to work as part of a team. Increasingly mathematics research is done collaboratively, and in industry every professional is embedded as part of a collaborative team. Thus I believe it is important for students to have opportunities to work together in their courses. I usually give students this opportunity in the form of a long term open ended group project. For instance, in a course on optimization students may be given a more applied project based on an industry case study. Each group would be tasked with proposing a model for the problem and solving it using techniques from the course in whatever way they see fit. This encourages them to not only develop their ability to work with others, but also to leverage their creativity. Finally, it gives them a chance to see how the mathematics they learn in the classroom can have real world impact.

\section{Evaluation of Teaching}
I find feedback on my teaching to be tremendously valuable. It is the easiest way for me to identify potential avenues for improvement, while at the same time motivating me by highlighting strong points of my teaching style. I have organized my evaluations below into two categories: those from students in courses I have taught, and those from peers and teaching professionals observing my lectures.
\subsection{Student Evaluations}
%Chart out survey responses for CO 327 with survey adjectives for context for scores (some here, rest in appendix)
%Student comment strong points
%Student comment targets for change
\subsection{Teaching Observations}\label{sec:teaching-observations}
One element of the CUT teaching program I took alongside my PhD was having lectures I give upervised by an expert in teaching working with the university's Centre for Teaching Excellence (CTE). I completed two teaching observations through the program. Below I summarize the feedback I received from these observations.
%highlights from 1st observation
\subsubsection*{First Teaching Observation}
\paragraph{}
My first observation was on the guest lecture I gave for Introduction to Game Theory in Fall $2018$ on Correlated Equilibria. The observation was conducted by Shahrukh Athar, a Graduate Instructional Developer working for the CTE. I have included the relevant sections of the full report in Appendix \ref{sec:observation-report-1}. The feedback in the report was divided in \emph{aspects to maintain}, where strengths of teaching were identified, and \emph{targets for change}, where opportunities for improvement were identified.
\paragraph{Aspects to Maintain} The first strength Shahrukh identified was my use of questions and comprehension checks. He appreciated the variety of checks I used, as well as the split between content-specific checks and general purpose open-ended questions. He commented that ``\emph{All of this is
fantastic as it demonstrates a learner-centric approach to teaching rather than a teacher-centric
one. It shows that you want to take your learners along with you as the lesson progresses and
are willing to pause and clarify confusions before moving on.}'' 

Shahrukh also found my lesson structure and transitions to be very strong, saying he will ``\emph{encourage [me] to continue
following such a structured approach in the delivery of [my] future lessons}''. He appreciated my use of an outline at the start of the lesson the students could follow along with, and found my method of using comprehension checks to indicate upcoming transitions to the next component of a lesson to be a highly effective technique.

Lastly, Shahrukh really enjoyed the high degree of interactivity in my lessons. Citing both the instructor-student questions throughout the lesson, and student-student interaction which occurred during the post-test. He particularly liked the latter, ``\emph{ I specifically want to celebrate here is the learner-learner
activity that you had towards the end of the lesson}.'' The activity he refers to is a computation of a correlated equilibrium that the students were asked to perform in pairs during the last $10$ minutes of the lecture.

\paragraph{Targets for Change} Shahrukh identified for me a couple of areas in which I could improve my teaching. The first area he identified involved some practical matters he dubbed ``awareness of learner needs''. He sat at the back of the classroom for the lecture, and found that from his position my handwriting was ``\emph{a touch small}''. The fix for this is easy, and I have since adapted it in all my lessons. Before the lecture starts I write some sentences and pieces of mathematics on the board, then head to the back of the room to verify that the font size is legible from that far away. Shahrukh also observed that there were a couple of noise disruptions caused by students walking in the halls outside the classroom since I left the doors open. I had initially done this to be welcoming to students who arrive late, but have since starting closing doors at the start of the lecture after verifying there are no students rushing through the hall trying to reach the classroom.

Shahrukh noticed a habit I had during the lecture of speaking while writing and facing the board. While there were no issues with my voice projecting, it created a visual disconnect with the students. I have since adopted his strategy of ``Write-Turn-Talk'', where I am silent during writing on the board and turn around to explain what I just wrote to my students. A major strength of this technique is that it does not ask students to process two streams of information at once (my voice and my writing), but rather they can concentrate on copying what I a writing as I write it and then concentrate on the verbal explanation I give after it is written down. In my second teaching observation I specifically asked the observer to verify that I had successfully implemented this technique.

\subsubsection*{Second Teaching Observation}
\paragraph{}My second observation was conducted by Caelan Wang, a Graduate Educational Developer with the CTE. The lecture to be observed was one of my regularly scheduled lectures as the instructor for Deterministic OR Models in Spring $2019$, covering cutting plane methods for solving integer programs. Caelan was especially qualified to observe this lecture since she was a graduate from the same department I am pursuing my PhD in, and knew the topic well.

Caelan's general impression of the lesson can be summarized as ``\emph{You were confident, clear, and
knowledgeable in the classroom. Overall, your lesson was definitely a success}.''. As with the previous report, more detailed feedback was separated into aspects to maintain and targets for change. The relevant sections of the full report can be found in Appendix \ref{sec:observation-report-2}
\paragraph{Aspects to Maintain}
Caelan appreciated the structure of my lesson and my preparation. Specifically she highlighted my use of a lecture outline and my starting the session with a review of the previous lecture's material connecting that content to the upcoming lecture. She also appreciated my confident delivery, saying ``\emph{in fact, you did not stumble even once in your teaching, which was impressive.}''

Caelan was also impressed with my handwriting and board use. She found the presentation very clear and liked the use of emphasis with bullet points and underlining, as well as the strategic positioning of figures so that they can stay on the board as long as possible given the small amount of board space in the particular room the lecture took place in.

Finally, Caelan celebrated my use of examples and questioning strategies. She liked how I presentated examples to motivate upcoming theory and definitions, and would revisit examples throughout the lecture to help reinforce new ideas as the lecture progressed. She said ``\emph{Such well-thought examples and questions really kept your learners actively engaged during your lesson. I would highly encourage you to continue these
great practices in your future teaching!}''.
\paragraph{Targets for Change}
%highlights from 2nd observation with a narrative about improving
One suggestion for improvement Caelan had was to remember to repeat questions or responses given by students in class. Her argument being that even though the classroom was small students are concentrating on the instructor, not necessarily each other, and it is helpful to do this to make sure they hear important information. I agree with her assessment, and this is an easy thing to tweak in my lecturing technique.

I had asked Caelan to pay attention to the Write-Turn-Talk technique I was working on from Shahrukh's feedback on my previous observation. She noted that I had successfully implemented the habit of not talking while writing on the board, but she had a related taret for change. While I do not write and talk, I would sometimes face the board to point out aspects of a definition or elements of a figure I had drawn in relation to a student's question. She suggested an adjustment where I practice facing the audience while pointing to elements of the figure. This can be tricky, but the idea would be to look at the figure, point out what I have in mind, turn and then discuss it. In future lectures I plan to practice this technique and I think it will certainly help strengthen communication with my students.

\section{Professional Development}
\subsection{Fundamentals of University Teaching}
\paragraph{}
During my Masters degree I completed the Fundamentals of University Teaching program from the CTE at the University of Waterloo. The program consisted of six workshops which focused effective lesson plan creation, questioning strategies, designing student assessments, motivating students, managing student atteiont, and giving and receiving feedback. Also as part of the program I gave three micro-teaching sessions to my peers in the program. A microteaching session consists of brief $15$ minute lecture with an emphasis on active learning techniques and student engagement. Each session was evaluated by an affiliate of the CTE from whom I received feedback that identified aspects of my teaching to maintain and targets for change. This program provided an excellent foundation for my approach to university teaching.

\subsection{Certificate in University Teaching}
\paragraph{}During my PhD studies I had an opportunity to expand upon the teaching skills I developed in the Fundamentals of University Teaching via the Certificate in University Teaching program. This program is more intensive and in-depth than the Fundamental program. This program involved four workshops: understanding the learner, interactive teaching strategies, assessing student learning, and course design. Each workshop required me to complete a response paper where I reflected on how the concepts from the workshop could apply in my teaching. The program also involved two in-class teaching observations by a CTE Graduate Instructional Developer, the resulting feedback from which I discuss in Subsection \ref{sec:teaching-observations}.

For the program I wrote a research paper summarizing the current research literature on ``Impact of Exam Format on Student Learning Outcomes''. An extended abstract can be found in the Appendices \textbf{TODO LINK}. 

\subsection{Mathematics Teaching Seminar} 
\paragraph{}
The Faculty of Mathematics at the University of Waterloo offers a $12$-week program for graduate students to develop their teaching skills. Each week involved a $2$ hour seminar on an aspect of teaching skills with a specialized focus on teaching mathematics. I completed this program as part of my PhD.

Some specialized seimar presentations studied topics such as presentations skills with a particular focus on blackboard technique for mathematics,  and designing mathematics problem sets. The program also involved microteaching sessions similar to the Fundamentals of University Teaching Program, but since the peers were all graduate students in mathematics the lectures could be tuned to a more specialized audience to practice more technical skills. It also invovled teaching observations of some of the most highly reviewed lectures at Waterloo, and reflection discussions on what techniques they are using to be effective, how they engage students, and how they deal with setbacks during the lecture. Overall this seminar series was very valuable in supplementing the skills I had developed through the CTE programs with techniques specifically focused on the discipline in which I will be teaching.

\section{Future Goals}

\paragraph{}
My immediate short term goal is to teach Deterministic OR Models again in Spring $2020$. I had a great experience last time I taught it, and the students had very positive feedback, but there some suggestions for improvement. I'm hoping to incorporate the feedback I have received to improve the experience for students during the next offering.

Thinking more long term I would like to become a professor at an institution which values high quality teaching as much as research. I am excited to teach across all levels of undergraduate and graduate students. Teaching students early in their journey is a great opportunity to give them a solid foundation in the fundamentals they will need to matter what direction they choose and to get them excited and passionate about pursuing mathematics further. Teaching more senior students and graduate students is a great opportunity to have more close interaction with each individual student, to share my specialities in mathematics, and to expose students to cutting edge research topics.

I would also like to participate more in outreach to high school students, and possibly even younger students. I believe my areas of focus in algorithms, combinatorics, and optimization have many topics which are highly accessible to students who have not yet seen calculus. These topics are full of many interesting puzzles and are also imminently applicable across a wide range of industries, in other words they are perfect for engaging young people and encouraging them to pursue mathematics.

\appendix
\section*{Appendices}
%Observation Reports
%Extended Abstract for Research paper
%Full evaluations
%Sample Lesson Plan from guest lecture
%Sample Assignment from CO 327
%Sample Lesson Plan from CO 327
\section{Teaching Observation Report 1}\label{sec:observation-report-1}
\begin{tabular}{r l}
\textbf{Event Observed:}&	CO-456: Introduction to Game Theory\\
\textbf{Date Observed:}&	November 23, 2018\\
\textbf{Location:}&	E2-1732\\
\textbf{Time:}&	11:30 am - 12:20 pm\\
\textbf{Number of Students Present:}&	39\\
\textbf{Observer:}&	Shahrukh Athar, Graduate Instructional Developer,\\
     &University of Waterloo Centre for Teaching Excellence\\
\end{tabular}

\rule{\linewidth}{0.25mm}
\textbf{Aspects to Maintain:}

\textbf{1. Use of Questions and Comprehension Checks}

Justin, it was great to see you perform comprehension checks regularly during the lesson. You carried out such checks by asking a variety of questions. You performed content-specific checks by asking open-ended questions, such as, “This polytope is always non-empty, does anyone know why?” or “Why is this condition what we want?”, while you carried out general- purpose checks by asking questions such as “At this point does anyone have any questions?”. You also asked questions that lie between the above two categories by asking questions such as “Does anyone have any questions about what correlated equilibria is?”. All of this is fantastic as it demonstrates a learner-centric approach to teaching rather than a teacher-centric one. It shows that you want to take your learners along with you as the lesson progresses and are willing to pause and clarify confusions before moving on. Since confusions at one point of the lesson may hinder learning later, it is very important to address them, especially in mathematics classes. The open-ended questions that you asked encouraged your students to think critically, which is an excellent tool for enhancing the learning experience. My only
recommendation here would be to ask you to pause a bit more after asking questions, for at least 5~10 seconds. As you pause, students would get the time to develop their answers and respond to you or think if they need to ask a question from you.

\textbf{2. Lesson Structure and Transitions}

I found your lesson’s main body to be incredibly well-structured. You had written the lesson’s outline on the board before the class started and this allowed your students to keep track of the lesson’s progress. Most courses in STEM disciplines have lessons that are composed of a number of components. Clear transitions between these components during the lesson enhances the overall structure of the lesson and brings clarity to the students in following the lesson. It was thus great to see that you clearly stated the different parts of the lesson that were being discussed at any given time. For example, after discussing correlated equilibria, you first performed a comprehension check about it and then said, “Let’s try to understand the relationship between correlated equilibria and the mixed Nash equilibria.”. Here, the comprehension check signified that one component of the lesson (discussion on correlated equilibria) was coming to an end, while the above-mentioned sentence signaled to your students that the next component of the lesson (discussion on the relationship between correlated and mixed Nash equilibria) was about to begin. I will encourage you to continue following such a structured approach in the delivery of your future lessons.

\textbf{3. Interactivity}

It is always beneficial to use a mix of instructor-learner and learner-learner interactions as we teach a lesson. This is because each of them fosters learning in a different manner. Through the use of questions, your lesson had a good amount of instructor-learner interaction throughout its length. However, what I specifically want to celebrate here is the learner-learner activity that you had towards the end of the lesson. You asked your students to solve an example (Prisoner’s dilemma) in pairs. The problem was clearly defined and was well-aligned with your learning outcomes. This activity allowed you to carry out post-assessment and provided an opportunity to your students to discuss any confusions with each other, which is great since an assignment on the topic was going to be handed out soon. I will strongly encourage you to continue using such active learning techniques in the future.\newline
\textbf{Targets for Change and Methods for Improvement:}

\textbf{1. Awareness of Learner Needs}

The lesson was conducted in a large classroom, which was deep and wide. In such classrooms, it can be difficult for students sitting further back, to see what is being written on the board or to hear the instructor. I was sitting in the last row of the classroom, and at times I had to really concentrate to see what was being written on the whiteboard as your writing size was a touch small. The doors of the classroom were left open throughout the lesson, and this was a source of distraction whenever people, who were talking to each other, passed in the adjacent corridor. In such moments, it became a bit difficult to follow what you were saying. I will encourage you to check with your students, especially those sitting at the back, as to whether your writing is clearly visible to them, and if they can hear you. Preferably such checks should be performed at the very beginning of the lesson. It may also be a good idea to use a mic in such large classrooms and to close the doors of the room at the beginning. I will also encourage you to ask your students before rubbing a section of the whiteboard. I noticed that you addressed your students as “guys” a number of times. Since there were women sitting in the classroom, they
might not have identified with the word “guys”. I will encourage you to try using terms that are more inclusive. For example, you could try “you all,” “all of you,” “you folks,” and so on.

\textbf{2. Write-Turn-Talk}

When writing on the whiteboard, it is important not to talk as your back is turned to your audience. This creates a disconnect between the teacher and the learners. For example, a student may have their hand raised to signal to the instructor that they want to ask a question, but if the instructor is not looking towards the students then they can easily miss taking that question. This also leads to the loss of eye contact, which is an important aspect of interpersonal communication as it signals interest in others and helps in classroom management (such as preventing students from getting distracted from the material). I noted that you talked while writing on the board several times during the lesson. For example, when you were discussing different cases, you were almost continuously writing on the board and talking at the same time. In your future teaching opportunities, try to incorporate the Write-Turn-Talk technique, where you should write on the board first without talking, then turn towards the learners and talk while looking at them. Doing so will enable you to better engage the students into the learning process and will help you in keeping a bird’s eye view of the class.\newline
\textbf{Additional Comments:}
Thank you for having me in this class for your lesson, Justin! It was a pleasure to observe your teaching. I wish you nothing but the best in all your future teaching endeavors.
\section{Teaching Observation Report 2}\label{sec:observation-report-2}
\begin{tabular}{r l}
    \textbf{Event Observed:}&	CO327: Deterministic OR Models\\
    \textbf{Date Observed:}&	July 16, 2019\\
    \textbf{Location:}&	RCH 206\\
    \textbf{Time:}&	10:00am – 11:00am\\
    \textbf{Number of Students Present:}&	8\\
    \textbf{Observer:}&	Caelan Wang, Graduate Educational Developer,\\
         &University of Waterloo Centre for Teaching Excellence
\end{tabular}

\rule{\linewidth}{0.25mm}
You were confident, clear, and knowledgeable in the classroom. Overall, your lesson was definitely a success.\newline
\textbf{Aspects to Maintain:}

\textbf{1. Lesson Structure and Preparation}
Justin, your lesson was certainly well-structured so that it was easy to see the main points you wanted to get to. I appreciated that you arrived a few minutes early to put the outline of the lesson on the board. It gave your learners and overview of the material covered in the lesson, which helped them to become more focused while getting prepared. Once you started teaching, you reminded your learners of the material in the previous lesson, giving them a nice warm up.
Throughout the lesson, it was evident to me that you were knowledgeable and well-prepared to teach. You always appeared to be very confident, knowing what to put on the board. If fact, you did not stumble even once in your teaching, which was impressive. Even though I didn’t get to see the end of the lesson, the structure was very clear and I had a good idea of where you wanted to get to. Well done!

\textbf{2. Handwriting and Board Use}

Your handwriting was very clear and your board work was well-organized. For example, when you were discussing the dual simplex method, I found it very helpful that you used bullet points to list out the steps of the methods. Moreover, you underlined some key words to make sure your learners know what the emphasis of the lesson were. If a learner is 10 minutes late to your lesson, they could easily get the idea of what they have missed, thanks to your great board work. With the very limited board space you had, it was particularly difficult to use the board well, but you did a great job!

The board did look a little dark from the very back of the classroom. It might be better to turn on the lights above the board and/or use coloured chalk to highlight the key words, so that they would be easier to see.

\textbf{3. Use of Examples}

I really appreciated that you used a lot of examples throughout your lesson, so that your learners could better understand the mathematical principles through practice. When you were explaining the dual simplex method, you gave an example of the linear program, and then explained the big ideas behind the dual simplex method using this example, giving your learners something concrete to work with while learning how to choose a variable to enter/leave the basis.

Throughout your teaching, you have used examples and guided questions to promote critical thinking amongst your learners. For instance, after explaining the dual simplex method, you asked how dual feasibility would be preserved. By looking at the example, your learners were able to come up with the solution to always pick the non-basic variable with the smallest index to avoid cycling. Such well-thought examples and questions really kept your learners actively engaged during your lesson. I would highly encourage you to continue these great practices in your future teaching!\\
\textbf{Targets for Change and Methods for Improvement:}

\textbf{1. Effective Questioning and Responding Techniques}

Justin, as I mentioned earlier, you have certainly asked a lot of good questions during this observation. Moreover, there was ample comprehension checks for you to make sure your learners were on track. These are all great, and I just want to provide some further tips to enhance your questioning and responding techniques.

One thing I noticed was, even though you were doing a lot of comprehension checks, your
learners weren’t always responding to them. When facing a group of relatively quiet learners, it can be hard to understand how they are doing. When you ask a question, I would recommend you to wait a little longer to give your learners enough time to form a response. I understand it can be hard to stand through the “awkward silence”, so I would recommend to
bring a bottle of water and take a sip of water after you ask a question, so that the time passes more naturally. Alternatively, if you don’t get a response after a comprehension check, sometimes I would do a poll of the class to see how they are doing.

When a learner asks a question or makes a comment, I want you to work on repeating it before further discussions. With such a small classroom, your learners could certainly hear each other. However, they would be paying attention to you instead of each other, so when a learner says something, others could easily miss it. That is why repeating what your learners said would be essential to ensure your entire classroom on the same page. I encourage you to try this out the next time you teach.

\textbf{2. Pacing, Volume, and Eye Contact}

The pace of your speech was quite fast, which might have been okay if you are teaching in a different room. However, this specific classroom did not help – it was small, but it also echoed, making it difficult to hear you clearly. I would encourage you to try to speak just a little slower so it would be easier to hear.

You asked for specific feedback on write-turn-talk, and I would say I did not see you write and talk at the same time a lot. However, when discussing the material written on the board, you had the tendency of facing the board to point at your writing. While pointing is certainly a good technique to keep your learners attention, I want you to try to be more mindful of this and try to face the learners even when you need to point at certain things on the board. When you are discussing the material, it would be particularly helpful to maintain good eye contact with your learners, so that you could make sure they are following your explanations.

\section{}
\end{document}