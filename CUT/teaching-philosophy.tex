\documentclass{article}

\usepackage{apacite}
\usepackage[colorlinks = true,
linkcolor = red,
urlcolor  = blue,
citecolor = green,
filecolor = cyan,
anchorcolor = blue]{hyperref}
\usepackage{setspace}
\usepackage[margin=1in, bottom=1in, top=1in]{geometry}
\title{Statement of Teaching Philosophy - Draft}
\author{W. Justin Toth}
\begin{document}
\maketitle
\tableofcontents
\newpage
\section{Statement of Teaching Philosophy}
\paragraph{}
Mathematics is a rich, deep tapestry of fascinating interrelated concepts about shapes and numbers. Many passionate teachers of the subject, myself included, share this sentiment. As an instructor, my goal is to transmit this perspective to my students. I want to stoke their passion, ignite their curiosity, and instill lifelong confidence in their understanding. Towards these lofty goals I have adopted methods of teaching which encourages students to take greater ownership over the material and their own learning process.

The heart of the student-teacher relationship is the lecture, and I believe in transforming my lectures into an active dialogue between student and teacher to encourage greater engagement by students. During my lectures I alternate between explaining concepts and asking questions with a frequency that evenly divides time between me speaking and students speaking. For instance, when presenting a definition I immediately follow up asking students for an example which satisfies the definition, or during a proof I may begin by writing what we are trying to show and ask students what technique we should proceed with. I find this to be one of the most powerful techniques for teaching. It pushes students to actively engage with the material and also allows them to quickly perform self-diagnosis to see if they are absorbing the lecture content. It even helps me better understand my audience. I can quickly determine which students are easily understanding, which are struggling, which ones are not feeling confident about speaking, and which ones are giving their attention. This feedback allows me to adapt during the current and future lectures by changing pace, or stopping to clarify some important background concept I notice students are missing. Within the first couple weeks of any given term this method leads to significant changes in many students. Often in the first lecture students are hesitant to speak, and will only speak when called upon. As they get used to my style of teaching they become more confident in responding, and become more likely to volunteer ideas or ask questions without being prompted.

I believe doing is necessary for learning, and central to the act of doing mathematics is problem solving. I see independent problem solving as a skill which can be taught and I use problem solving as an opportunity to foster student creativity and increase their comfort in engaging with the unknown. To teach problem solving I work proto-typical problems into my lectures. When working these problems, whether I’m presenting the next step or asking the students for it, the thinking which led to the step is front and center. I present problem solving in a four step approach as outlined by the famous mathematician George Polya in his book “How to Solve It”: understand the problem, devise a plan, carry out the plan, and look back. This framework gives students structure and I find it severely cuts back on generic questions about not knowing what to do.  I see students reframing their questions more specifically in terms of where in the problem solving steps they are stuck, which in turn leads to a deeper understanding on their part, and makes it much easier to diagnose their challenges and provide meaningful feedback. As my students grow in confidence, I start using problems to provoke their curiousity. My lectures and assignments are peppered with optional exercises that highlight a subtle corner-case, interesting novelty, or surprising connection with material outside the course. This is an opportunity to broaden student horizons without overwhelming those who are already bogged down in coursework. Students who engage with this material will often come to office hours to discuss what they discovered, which leads to fascinating side discussions that enrich the course and capture student interest.

Ultimately student confidence hinges on their ability to recall material and identify where it is useful. Similarly, a major motivator for student passion is seeing how interconnected the material they are learning is with other material they have learned or will learn. Thus strengthening both passion and confidence can be done through putting material in context and forming a dense web of connections between different concepts. The basic organization of my courses, lectures, notes, and assignments supports this goal. It starts with the course syllabus where I outline the learning objectives for the course alongside a timeline of when these objectives will be achieved. At the start of a lesson I outline learning outcomes for that session, and review them again at the end of the lesson. Lectures themselves start with a review of previous material needed for the lecture, and end by describing what we learned that session and how we will need it in future sessions. Before exams I set aside lecture time to review the skills I expect students to have attained, and will be testing on that particular exam. Altogether this creates a pattern of reinforcement for the important concepts of the course, and supports students in drawing connections between the lectures over time. I dedicate special revision passes on my lecture notes and assignments to identifying connections with not only past and future lectures, but also with other courses the students may have taken in the past or will take in the future. This exercise is immensely valuable, not only for retention of material, but for motivation. I’ve seen tremendous improvements in student passion for the material when they come to see how the topics are situated within the bigger picture of what they are learning in their program.

I view teaching as a craft, and I am constantly honing my tools and methods to perform better in this craft. I consider myself incredibly lucky to have the opportunity to share my knowledge, understanding, and ultimately passion with my students. I strive to make the most of this opportunity every time I step in front of a class or open my door for office hours.

\section{Teaching Experience}

\section{Teaching Strategies}

\section{Evaluation of Teaching}

\section{Professional Development}

\section{Future Goals}


\appendix
\section*{Appendices}
\section{}


\end{document}