\documentclass[12pt]{article}

\usepackage{apacite}
\usepackage{setspace}
\usepackage{hyperref}

\doublespace
\title{Working Title - CUT Research Paper}
\author{W. Justin Toth}
\begin{document}
\maketitle
\begin{abstract}
Research Question: ``How does assessment format impact student learning outcomes, and instructor measurement thereof''.

New Research Question: ``Can innovation assessment formats improve student learning outcomes, and an instructors capacity to measure their achievement.''

\textit{With a particular focus on mathematics, stem}
\end{abstract}

\section{Introduction}

\paragraph{}
A well-known truism in human social interaction is that how you say something often has as much, if not more, importance that the specific content of your speech. Yet when instructors are designing evalutions for their courses, it is all too common that the focus is squarely on what questions to ask to cover the course content, rather than how the problems should be presented to the students. In this survey we seek to synthesize the academic literature on the ``how'', rather than ``what'', of assessment methods and provide guidance for instructors on the impact that exam format can have on student learning outcomes and their ability to measure the attainment thereof.
\paragraph{}
In Section \ref{sec:written} we study the particulars of written finals. For written finals we will concentrate on two areas of focus in the literature. The first is on the use of multiple choice versus constructed response questions, and the second is on allowing students to use reference material such as the course textbook or handwritten notes during the exam. In Section \ref{sec:communication} we study the use of oral communication during exams. This breaks down in to two subsections. In Subsection \ref{subsec:oral} we survey the literature on implementing interview-style oral exams  between student and instructor in the classroom, and in Subsection \ref{subsec:collab} we discuss student-to-student collaboration in exams with a particular focus on the innovative two-stage exam format. We conclude by with Section \ref{sec:recommendations} where present recommendations for instructors on choosing exam formats for their courses based on the previously surveyed literature.
\section{Individual Written Finals}\label{sec:written}
\subsection{Multiple Choice}
\subsection{Reference Material}
\subsection{Frequency}
%can include takehome

\section{Oral Exams and Collaboration}\label{sec:communication}
\subsection{Oral Exams}\label{subsec:oral}
\paragraph{Implementation}
\paragraph{Results}
\paragraph{Benefits}
\paragraph{Drawbacks}
\paragraph{Data}

\subsection{Collaborative Exams}\label{subsec:collab}
\paragraph{Group}
\paragraph{Two-Stage}


\section{Recommendations}\label{sec:recommendations}
content
\subsection{Designing Exams to Encourage Good Learning Habits}
\subsection{Assessing a Variety of Cognitive Processes}
\subsection{Encouraging Genuine Engagement}
\nocite{*}
\bibliographystyle{apacite}
\bibliography{references.bib}

\end{document}