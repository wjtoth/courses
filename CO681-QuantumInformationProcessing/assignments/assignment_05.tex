\documentclass[letterpaper,12pt,oneside,onecolumn]{article}
\usepackage[margin=1in, bottom=1in, top=1in]{geometry} %1 inch margins
\usepackage{amsmath, amssymb, amstext}
\usepackage{fancyhdr}
\usepackage{mathtools}
\usepackage{algorithm}
\usepackage{algpseudocode}
\usepackage{theorem}
\usepackage{tikz}
\usepackage{tkz-berge}
\usepackage[braket, qm]{qcircuit}

%Macros
\newcommand{\A}{\mathbb{A}} \newcommand{\C}{\mathbb{C}}
\newcommand{\D}{\mathbb{D}} \newcommand{\F}{\mathbb{F}}
\newcommand{\N}{\mathbb{N}} \newcommand{\R}{\mathbb{R}}
\newcommand{\T}{\mathbb{T}} \newcommand{\Z}{\mathbb{Z}}
\newcommand{\Q}{\mathbb{Q}}
 
 
\newcommand{\cA}{\mathcal{A}} \newcommand{\cB}{\mathcal{B}}
\newcommand{\cC}{\mathcal{C}} \newcommand{\cD}{\mathcal{D}}
\newcommand{\cE}{\mathcal{E}} \newcommand{\cF}{\mathcal{F}}
\newcommand{\cG}{\mathcal{G}} \newcommand{\cH}{\mathcal{H}}
\newcommand{\cI}{\mathcal{I}} \newcommand{\cJ}{\mathcal{J}}
\newcommand{\cK}{\mathcal{K}} \newcommand{\cL}{\mathcal{L}}
\newcommand{\cM}{\mathcal{M}} \newcommand{\cN}{\mathcal{N}}
\newcommand{\cO}{\mathcal{O}} \newcommand{\cP}{\mathcal{P}}
\newcommand{\cQ}{\mathcal{Q}} \newcommand{\cR}{\mathcal{R}}
\newcommand{\cS}{\mathcal{S}} \newcommand{\cT}{\mathcal{T}}
\newcommand{\cU}{\mathcal{U}} \newcommand{\cV}{\mathcal{V}}
\newcommand{\cW}{\mathcal{W}} \newcommand{\cX}{\mathcal{X}}
\newcommand{\cY}{\mathcal{Y}} \newcommand{\cZ}{\mathcal{Z}}

\newcommand\numberthis{\addtocounter{equation}{1}\tag{\theequation}}


\newenvironment{proof}{{\bf Proof:  }}{\hfill\rule{2mm}{2mm}}
\newenvironment{proofof}[1]{{\bf Proof of #1:  }}{\hfill\rule{2mm}{2mm}}
\newenvironment{proofofnobox}[1]{{\bf#1:  }}{}\newenvironment{example}{{\bf Example:  }}{\hfill\rule{2mm}{2mm}}

%\renewcommand{\thesection}{\lecnum.\arabic{section}}
%\renewcommand{\theequation}{\thesection.\arabic{equation}}
%\renewcommand{\thefigure}{\thesection.\arabic{figure}}

\newtheorem{fact}{Fact}[section]
\newtheorem{lemma}[fact]{Lemma}
\newtheorem{theorem}[fact]{Theorem}
\newtheorem{definition}[fact]{Definition}
\newtheorem{corollary}[fact]{Corollary}
\newtheorem{proposition}[fact]{Proposition}
\newtheorem{claim}[fact]{Claim}
\newtheorem{exercise}[fact]{Exercise}
\newtheorem{note}[fact]{Note}
\newtheorem{conjecture}[fact]{Conjecture}

\newcommand{\size}[1]{\ensuremath{\left|#1\right|}}
\newcommand{\ceil}[1]{\ensuremath{\left\lceil#1\right\rceil}}
\newcommand{\floor}[1]{\ensuremath{\left\lfloor#1\right\rfloor}}

\DeclarePairedDelimiter\abs{\lvert}{\rvert}%
\DeclarePairedDelimiter\norm{\lVert}{\rVert}%

\DeclareMathOperator{\swap}{SWAP}

%END MACROS
%Page style
\pagestyle{fancy}

\listfiles

\raggedbottom

\lhead{\today}
\rhead{W. Justin Toth CO681-Quantum Information Processing A5} %CHANGE n to ASSIGNMENT NUMBER ijk TO COURSE CODE
\renewcommand{\headrulewidth}{1pt} %heading underlined
%\renewcommand{\baselinestretch}{1.2} % 1.2 line spacing for legibility (optional)

\begin{document}
\section{}
\subsection{a}
\paragraph{}
We have (leaving off normalization)
\begin{align*}
\ket{\phi^+} &= \ket{00} + \ket{11}\\
\ket{\phi^-} &= \ket{00} - \ket{11} \\
\ket{\psi^+} &= \ket{10} + \ket{01} \\
\ket{\psi^-} &= \ket{10} - \ket{01}
\end{align*}
So then
$$\ket{\phi^+}\bra{\phi^+} + \ket{\phi^-}\bra{\phi^-} + \ket{\psi^+}\bra{\psi^-} = \frac{1}{2}\begin{bmatrix}
2 & 0 & 0 &0 \\
0 & 1 & 1 & 0 \\
0 & 1 & 1 & 0 \\
0 &0 &0 &2
\end{bmatrix}$$
and 
$$\ket{\psi^-}\bra{\psi^-} =\frac{1}{2} \begin{bmatrix}
0 & 0 &0 & 0\\
0 & 1 & -1 & 0 \\
0 & -1 & 1 & 0 \\
0 &0 &0 &0
\end{bmatrix}$$
Hence our Werner state $\rho$ as a $4\times 4$ matrix is
$$\rho = \frac{1}{6}\begin{bmatrix}
2(1-p) & 0 & 0 &0 \\
0 & 1+2p & 1-4p & 0 \\
0 & 1-4p & 1+2p & 0 \\
0 & 0 & 0 & 2(1-p)
\end{bmatrix}$$
The partial transpose of $\rho$ is 
$$(I \otimes T)\rho = \frac{1}{6}\begin{bmatrix}
2(1-p) & 0 & 0 &1-4p \\
0 & 1+2p3 & 0 & 0 \\
0 & 0 & 1+2p & 0 \\
1-4p & 0 & 0 & 1-3p
\end{bmatrix}$$
\subsection{b}
The eigenvalues $(I \otimes T)\rho$ are $\frac{1+2p}{6}$, $0$, $0$, and $\frac{3-6p}{6} = \frac{1-2p}{2}$. The least of which is $\frac{1-2p}{2}$. From our slides we know that $\rho$ is separable if and only if $(I \otimes T)\rho$ is positive semidefinite. This happens if and only if $\frac{1-2p}{2} \geq 0$, that is when $p \leq 1/2$.
\newpage
\section{}
\subsection{a}
\paragraph{}
Let $\ket{0}, \ket{1}, \dots, \ket{d-1}$ denote the computational basis of $\C^d$. We use $[d-1]$ as shorthand for $\{0,\dots, d-1\}$. Let $i, j \in [d-1]$. Then we have
$$\swap(\ket{i}\ket{j} + \ket{j}\ket{i}) = \ket{j}\ket{i} + \ket{i}\ket{j} = \ket{i}\ket{j} + \ket{j}\ket{i}$$
Hence $\ket{i}\ket{j} + \ket{j}\ket{i}$ is an eigenvector of $\swap$ with eigenvalue $1$. There are $d^2/2$ such vectors, and it is easy to see they are linearly independent. Let
$$B_+ = \{\ket{i}\ket{j} + \ket{j}\ket{i} : i,j \in [d-1]\}$$
Similarly we can observe that
$$\swap(\ket{i}\ket{j} - \ket{j}\ket{i}) = \ket{j}\ket{i} - \ket{i}\ket{j} = -1(\ket{i}\ket{j} - \ket{j}\ket{i}).$$
Hence $\ket{i}\ket{j} - \ket{j}\ket{i}$ is an eigenvector of $\swap$ with eigenvalue $-1$. Again there are $d^2/2$ such non-zero vectors. Let 
$$B_- = \{\ket{i}\ket{j} - \ket{j}\ket{i} : i,j \in [d-1]\}\backslash \{0\}.$$
One can easily verify that $B:= B_+ \cup B_-$ is a set of linearly independent vectors. Since $|B| = d^2$ this implies that $B$ forms a basis of $\C^d\otimes \C^d$. Thus we have shown that the eigenvalues of $\swap$ are exactly $\pm 1$ and more over if we define $P_+$ to be projector onto the span of $B_+$ and $P_-$ to be the projector onto the span of $B_-$ then 
$$\swap = P_+ - P_-.$$
\subsection{b}
\paragraph{}
We can simplify our description of $P_+$ somewhat. Observe that
$$P_+ = \frac{1}{2}\sum_{i=0}^{d-1} \sum_{j \leq i} (\ket{i}\ket{j} + \ket{j}\ket{i})(\bra{i}\bra{j} + \bra{j}\bra{i}) =\frac{1}{2}\big( \sum_{i,j=0}^{d-1} \ket{i}\ket{j}\bra{i}\bra{j} + \sum_{i,j=0}^{d-1} \ket{i}\ket{j}\bra{j}\bra{i}\big) = \frac{1}{2}\big(I + \swap\big).$$
Let $\ket{\psi}, \ket{\phi} \in \C^d$ be pure states. Then
\begin{align*}
P_+ \ket{\psi}\ket{\phi} &= \frac{1}{2}(\ket{\psi}\ket{\phi} + \ket{\phi}\ket{\psi})
\end{align*}
Hence 
\begin{align*}
Tr(P_+ \ket{\psi}\ket{\phi}\bra{\psi}\bra{\phi}P_+) &= \bra{\psi}\bra{\phi}P_+P_+ \ket{\psi}\ket{\phi}\\ &= \frac{1}{2}\big(\ip{\phi}{\phi}\ip{\psi}{\psi} + \ip{\phi}{\psi}\ip{\psi}{\phi})\\ &= \frac{1 + |\ip{\phi}{\psi}|^2}{2}
\end{align*}
Therefore the probability that the swap test declares $\ket{\psi}$ and $\ket{\phi}$ the same is $\frac{1 + |\ip{\phi}{\psi}|^2}{2}$.
\subsection{c}
\paragraph{}
We have that
\begin{align*}
\rho\otimes\sigma = &\frac{1}{4}(I \otimes I &+ r_x' I \otimes X &+ r_y' I \otimes Y &+ r_z' I \otimes Z \\
&r_x X\otimes I &+ r_xr_x' X\otimes X &+ r_xr_y' X\otimes Y &+ r_xr_z' X \otimes Z \\
&r_y Y \otimes I &+ r_yr_x'Y\otimes X &+ r_yr_y' Y \otimes Y &+ r_yr_z' Y \otimes Z \\
&r_z Z\otimes I &+ r_zr_x' Z\otimes X &+ r_zr_y' Z \otimes Y &+ r_zr_z' Z\otimes Z).
\end{align*}
We check the trace of the action of $P_+$ on the constituent terms of $\rho \otimes \sigma$. By symmetry of $\rho \otimes \sigma$, and the fact that $-iY = XZ$ it will suffice to check only $I\otimes I$, $X\otimes X$, $Z\otimes Z$, $I \otimes X$, $I\otimes Z$, and $X\otimes Z$. Using cyclic nature of trace and that $\swap$ is self-inverse, it suffices to understand happens when $\swap$ is applied to the above matrices. $\swap X\otimes X$ has non-zero diagonal entries $1$ only in second and third columns. Similarly $\swap Z\otimes Z$ has non-zero diagonal entries $1$ only in first and fourth columns. Further $\swap I \otimes X$ and $\swap X\otimes Z$ have no non-zero diagonal entries, and $\swap I \otimes Z$ has non-zero diagonal entries $1$ and $-1$.
\paragraph{}
Finally, using the above discussion, we calculate traces. First we have $Tr(P_+I\otimes I P_+) = Tr(P_+P_+) = \frac{12}{4} = 3$. Further
\begin{align*}
Tr(P_+ X\otimes X P_+) &= \frac{1}{4}Tr(X\otimes X + \swap X\otimes X + X\otimes X \swap + \swap X\otimes X \swap )\\
&= \frac{1}{4}(0 + 2 + 2 + 0)\\
&= 1.
\end{align*}
and similarly
$$Tr(P_+ Z \otimes Z P_+) = \frac{1}{4}(0 + 2 + 2 + 0) = 1.$$
Lastly,
$$Tr(P_+ I \otimes X P_+) = Tr(P_+ I \otimes Z P_+) = Tr(P_+ X\otimes Z P_+) = 0.$$
Therefore we have that the probability the swap test declare $\rho$ and $\sigma$ equal is
$$Tr(P_+ \rho \otimes \sigma P_+) = \frac{3 + r_xr_x' + r_yr_y' + r_zr_z'}{4}.$$
\newpage
\section{}
\paragraph{}
We use throughout that $\ket{A}$ and $\ket{B}$ are orthogonal. We have that
$$-HU_0HU_f = (2\ket{\theta}\bra{\theta} - I_2)(2\ket{B}\bra{B} - I_2) = 4\ip{\theta}{B}\ket{\theta}\bra{B} - 2\ket{B}\bra{B} - 2\ket{\theta}\bra{\theta} + I_2.$$
Now we observe that
$\ip{\theta}{B} = \cos\theta$ and that
$$\ket{\theta}\bra{B} = \sin\theta\ket{A}\bra{B} + \cos\theta\ket{B}\bra{B}.$$
Further that
$$\ket{\theta}\bra{\theta} = \sin^2\theta\ket{A}\bra{A} + \sin\theta\cos\theta (\ket{A}\bra{B} + \ket{B}\bra{A}) + \cos^2\theta\ket{B}\bra{B}.$$
Then we see that
\begin{align*}
-HU_0HU_f &= 4\sin\theta\cos\theta\ket{A}\bra{B} + 4\cos^2\theta\ket{B}\bra{B} -2 \ket{B}\bra{B} \\
&-2\sin^2\theta\ket{A}\bra{A} -2\sin\theta\cos\theta(\ket{A}\bra{B} + \ket{B}\bra{A}) -2\cos^2\theta\ket{B}\bra{B} \\
&+ \ket{A}\bra{A} + \ket{B}\bra{B}\\
&= (1-2\sin^2\theta)\ket{A}\bra{A} +2\sin\theta\cos\theta\ket{A}\bra{B} - 2\sin\theta\cos\theta\ket{B}\bra{A} + (2\cos^2\theta-1)\ket{B}\bra{B}\\
&= \cos2\theta \ket{A}\bra{A} + \sin2\theta\ket{A}\bra{B} -\sin2\theta\ket{B}\bra{A} +\cos2\theta\ket{B}\bra{B}
\end{align*}
as desired.
\newpage
\section{}
\paragraph{}
First observe that
$$Tr(\rho) = \sum_j a_j Tr(E_j) = \frac{1}{d} \sum_j a_j$$
and then that
$$p(i) = Tr(\rho E_i) = ba_i + c \sum_{j\neq i} a_j = ba_i + c(dTr(\rho) - a_i) = ba_i + cd - ca_i.$$
Solving for $a_i$ we see that
$a_i = \frac{p(i) -cd}{b-c}.$
\end{document}
