\documentclass[letterpaper,12pt,oneside,onecolumn]{article}
\usepackage[margin=1in, bottom=1in, top=1in]{geometry} %1 inch margins
\usepackage{amsmath, amssymb, amstext}
\usepackage{fancyhdr}
\usepackage{mathtools}
\usepackage{algorithm}
\usepackage{algpseudocode}
\usepackage{theorem}
\usepackage{tikz}
\usepackage{tkz-berge}
\usepackage[braket, qm]{qcircuit}

%Macros
\newcommand{\A}{\mathbb{A}} \newcommand{\C}{\mathbb{C}}
\newcommand{\D}{\mathbb{D}} \newcommand{\F}{\mathbb{F}}
\newcommand{\N}{\mathbb{N}} \newcommand{\R}{\mathbb{R}}
\newcommand{\T}{\mathbb{T}} \newcommand{\Z}{\mathbb{Z}}
\newcommand{\Q}{\mathbb{Q}}
 
 
\newcommand{\cA}{\mathcal{A}} \newcommand{\cB}{\mathcal{B}}
\newcommand{\cC}{\mathcal{C}} \newcommand{\cD}{\mathcal{D}}
\newcommand{\cE}{\mathcal{E}} \newcommand{\cF}{\mathcal{F}}
\newcommand{\cG}{\mathcal{G}} \newcommand{\cH}{\mathcal{H}}
\newcommand{\cI}{\mathcal{I}} \newcommand{\cJ}{\mathcal{J}}
\newcommand{\cK}{\mathcal{K}} \newcommand{\cL}{\mathcal{L}}
\newcommand{\cM}{\mathcal{M}} \newcommand{\cN}{\mathcal{N}}
\newcommand{\cO}{\mathcal{O}} \newcommand{\cP}{\mathcal{P}}
\newcommand{\cQ}{\mathcal{Q}} \newcommand{\cR}{\mathcal{R}}
\newcommand{\cS}{\mathcal{S}} \newcommand{\cT}{\mathcal{T}}
\newcommand{\cU}{\mathcal{U}} \newcommand{\cV}{\mathcal{V}}
\newcommand{\cW}{\mathcal{W}} \newcommand{\cX}{\mathcal{X}}
\newcommand{\cY}{\mathcal{Y}} \newcommand{\cZ}{\mathcal{Z}}

\newcommand\numberthis{\addtocounter{equation}{1}\tag{\theequation}}


\newenvironment{proof}{{\bf Proof:  }}{\hfill\rule{2mm}{2mm}}
\newenvironment{proofof}[1]{{\bf Proof of #1:  }}{\hfill\rule{2mm}{2mm}}
\newenvironment{proofofnobox}[1]{{\bf#1:  }}{}\newenvironment{example}{{\bf Example:  }}{\hfill\rule{2mm}{2mm}}

%\renewcommand{\thesection}{\lecnum.\arabic{section}}
%\renewcommand{\theequation}{\thesection.\arabic{equation}}
%\renewcommand{\thefigure}{\thesection.\arabic{figure}}

\newtheorem{fact}{Fact}[section]
\newtheorem{lemma}[fact]{Lemma}
\newtheorem{theorem}[fact]{Theorem}
\newtheorem{definition}[fact]{Definition}
\newtheorem{corollary}[fact]{Corollary}
\newtheorem{proposition}[fact]{Proposition}
\newtheorem{claim}[fact]{Claim}
\newtheorem{exercise}[fact]{Exercise}
\newtheorem{note}[fact]{Note}
\newtheorem{conjecture}[fact]{Conjecture}

\newcommand{\size}[1]{\ensuremath{\left|#1\right|}}
\newcommand{\ceil}[1]{\ensuremath{\left\lceil#1\right\rceil}}
\newcommand{\floor}[1]{\ensuremath{\left\lfloor#1\right\rfloor}}

\DeclarePairedDelimiter\abs{\lvert}{\rvert}%
\DeclarePairedDelimiter\norm{\lVert}{\rVert}%
%END MACROS
%Page style
\pagestyle{fancy}

\listfiles

\raggedbottom

\lhead{\today}
\rhead{W. Justin Toth CO681-Quantum Information Processing A2} %CHANGE n to ASSIGNMENT NUMBER ijk TO COURSE CODE
\renewcommand{\headrulewidth}{1pt} %heading underlined
%\renewcommand{\baselinestretch}{1.2} % 1.2 line spacing for legibility (optional)

\begin{document}
\section{}
\subsection{a}
\paragraph{}
$$11+12 \mod 13 = 10$$
\subsection{b}
\paragraph{}
$$\Z^\times_{13} = \{2^0=1, 2^1 = 2, 2^2 = 4, 2^3 = 8, 2^4 = 3, 2^5 = 6, 2^6=12, 2^7=11, 2^8 = 9, 2^9=5, 2^{10}=10, 2^{11} = 7\}$$
\subsection{c}
\paragraph{}
$2^8\equiv 9 \mod 13$ so $\log_2(9)$ is $8$ in $\Z^\times_{13}$.
\subsection{d}
\paragraph{}
\begin{itemize}
\item $3^3 \equiv 1 \mod 13$ so $3$ is not a generator
\item $4^6 \equiv 1 \mod 13$ so $4$ is not a generator
\item $5^4 \equiv 1 \mod 13$ so $5$ is not a generator
\item The order of $6$ is $12$ in $\Z_{13}^\times$ so is $6$ is a generator of $Z_{13}^\times$
\item The order of $7$ is $12$ in $\Z_{13}^\times$ so is $7$ is a generator of $Z_{13}^\times$
\item $8^4 \equiv 1 \mod 13$ so $8$ is not a generator
\item $9^3 \equiv 1 \mod 13$ so $9$ is not a generator
\item $10^6 \equiv 1 \mod 13$ so $10$ is not a generator
\item The order of $11$ is $12$ in $\Z_{13}^\times$ so is $11$ is a generator of $Z_{13}^\times$
\item $12 \equiv 1 \mod 13$ so $12$ is not a generator
\end{itemize}
Hence the generators of $\Z_{13}^\times$ besides $2$ are: $6,7,$ and $11$.
\newpage
\section{}
\subsection{a}
\paragraph{}
We check the action of $H$ on $X$ and $Z$.
$$HXH^{\dagger} = \frac{1}{2} \begin{bmatrix} 1 & 1 \\ 1 &  -1 \end{bmatrix} \begin{bmatrix} 0 & 1 \\ 1 & 0 \end{bmatrix} \begin{bmatrix} 1 & 1 \\ 1 &  -1 \end{bmatrix} = \frac{1}{2} \begin{bmatrix} 1 & 1 \\ 1 &  -1 \end{bmatrix} \begin{bmatrix}1 & -1 \\ 1 & 1 \end{bmatrix} =\frac{1}{2} \begin{bmatrix} 2 & 0 \\ 0 & 2 \end{bmatrix} = I. $$
$$HZH^{\dagger} = \frac{1}{2} \begin{bmatrix} 1 & 1 \\ 1 &  -1 \end{bmatrix} \begin{bmatrix} 1 & 0 \\ 0 & -1 \end{bmatrix} \begin{bmatrix} 1 & 1 \\ 1 &  -1 \end{bmatrix} = \frac{1}{2} \begin{bmatrix} 1 & 1 \\ 1 &  -1 \end{bmatrix} \begin{bmatrix} 1 & 1 \\ -1 & 1\end{bmatrix} = \frac{1}{2} \begin{bmatrix} 0 & 2 \\ 2 & 0\end{bmatrix} = X.$$
Thus $H$ is in the Clifford group on $1$ qubit.
\subsection{b}
\paragraph{}
Similar to the previous subsection we consider the action of $T$ on $X$ and $Z$.
$$TXT^{\dagger} = \begin{bmatrix} 1 & 0 \\ 0& \zeta_8 \end{bmatrix}\begin{bmatrix} 0 & 1 \\ 1 & 0 \end{bmatrix}\begin{bmatrix} 1 & 0 \\ 0& \zeta_8^* \end{bmatrix} = \begin{bmatrix} 1 & 0 \\ 0& \zeta_8 \end{bmatrix}\begin{bmatrix} 0 & \zeta^*_8 \\ 1 & 0 \end{bmatrix} = \begin{bmatrix} 0 & \zeta_8^* \\ \zeta_8 & 0 \end{bmatrix}.$$
Such a matrix cannot be written as  $\alpha V$ for phase $\alpha$ and $V \in \{I,X,Y,Z\}$. Thus $T$ is not in Clifford group on $1$ qubit.
\subsection{c}
\paragraph{}
$$SXS^\dagger = \begin{bmatrix} 1 & 0 \\ 0 & i \end{bmatrix} \begin{bmatrix} 0 & 1 \\ 1 & 0 \end{bmatrix} \begin{bmatrix} 1 & 0\\ 0 & -i \end{bmatrix} = \begin{bmatrix} 1 & 0 \\ 0 & i \end{bmatrix} \begin{bmatrix} 0 & -i\\ 1 & 0 \end{bmatrix} =\begin{bmatrix} 0 & -i \\ i & 0 \end{bmatrix} = Y.$$
$$SZS^\dagger = \begin{bmatrix} 1 & 0 \\ 0 & i \end{bmatrix}\begin{bmatrix} 1 & 0 \\ 0 & -1\end{bmatrix} \begin{bmatrix} 1 & 0 \\0 & -i \end{bmatrix} = \begin{bmatrix} 1 & 0 \\ 0 & i \end{bmatrix}\begin{bmatrix} 1 & 0 \\ 0 & i \end{bmatrix} = Z.$$
Thus $S$ is in the Clifford group on $1$ qubit.
\subsection{d}
\paragraph{}
Let $C$ be the unitary matrix representing the controlled-not gate. We consider the action of $C$ on $I \otimes X$, $X\otimes I$, $I \otimes Z$. and $Z \otimes I$.
$$C (I\otimes X) C^\dagger = \begin{bmatrix} 1 & 0 & 0 & 0 \\ 0 & 1 & 0 & 0 \\ 0 & 0 & 0 &1 \\ 0& 0& 1 & 0 \end{bmatrix}\begin{bmatrix}0 & 1 & 0 & 0 \\ 1 & 0 & 0 &0 \\ 0 & 0 &0 & 1 \\ 0 & 0 & 1 & 0 \end{bmatrix} \begin{bmatrix} 1 & 0 & 0 & 0 \\ 0 & 1 & 0 & 0 \\ 0 & 0 & 0 &1 \\ 0& 0& 1 & 0 \end{bmatrix} = \begin{bmatrix} 0 & 1& 0& 0\\ 1 & 0  & 0& 0\\ 0 & 0 & 1 & 0\\ 0 & 0 & 0 & 1\end{bmatrix} \begin{bmatrix} 1 & 0 & 0 & 0 \\ 0 & 1 & 0 & 0 \\ 0 & 0 & 0 &1 \\ 0& 0& 1 & 0 \end{bmatrix} = I \otimes X.$$
$$ C (X \otimes I) C^\dagger =  \begin{bmatrix} 1 & 0 & 0 & 0 \\ 0 & 1 & 0 & 0 \\ 0 & 0 & 0 &1 \\ 0& 0& 1 & 0 \end{bmatrix} \begin{bmatrix}  0 & 0 & 1 & 0 \\ 0 & 0 & 0 & 1 \\ 1 & 0 & 0 & 0 \\ 0 &1 & 0 &0\end{bmatrix} \begin{bmatrix} 1 & 0 & 0 & 0 \\ 0 & 1 & 0 & 0 \\ 0 & 0 & 0 &1 \\ 0& 0& 1 & 0 \end{bmatrix}  =  \begin{bmatrix} 0 & 0 &1 & 0 \\ 0 & 0 & 0 & 1 \\ 0 & 1 & 0 & 0 \\ 1 & 0 & 0 &0\end{bmatrix} \begin{bmatrix} 1 & 0 & 0 & 0 \\ 0 & 1 & 0 & 0 \\ 0 & 0 & 0 &1 \\ 0& 0& 1 & 0 \end{bmatrix} = X \otimes X$$
\begin{align*}C (I \otimes Z) C^\dagger &=  \begin{bmatrix} 1 & 0 & 0 & 0 \\ 0 & 1 & 0 & 0 \\ 0 & 0 & 0 &1 \\ 0& 0& 1 & 0 \end{bmatrix}\begin{bmatrix}1 &0 &0 &0 \\ 0 & -1 & 0 &0 \\ 0 & 0 &1 & 0\\ 0& 0& 0 & -1 \end{bmatrix} \begin{bmatrix} 1 & 0 & 0 & 0 \\ 0 & 1 & 0 & 0 \\ 0 & 0 & 0 &1 \\ 0& 0& 1 & 0 \end{bmatrix}\\
 &=  \begin{bmatrix} 1 & 0 & 0 &0 \\ 0 & -1 & 0 &0 \\ 0 & 0 & 0 & -1 \\ 0 &0 &1 & 0\end{bmatrix}\begin{bmatrix} 1 & 0 & 0 & 0 \\ 0 & 1 & 0 & 0 \\ 0 & 0 & 0 &1 \\ 0& 0& 1 & 0 \end{bmatrix}\\ &= \begin{bmatrix} 1 & 0 &0 &0 \\ 0 & -1 &0 &0 \\ 0 &0 &-1 & 0\\ 0 &0 &0 &1\end{bmatrix} = Z \otimes Z.\end{align*}
 \begin{align*}
 C(Z\otimes I)C^\dagger &=  \begin{bmatrix} 1 & 0 & 0 & 0 \\ 0 & 1 & 0 & 0 \\ 0 & 0 & 0 &1 \\ 0& 0& 1 & 0 \end{bmatrix} \begin{bmatrix}1 &0 &0 &0 \\0 & 1 & 0 &0 \\ 0 &0 &-1 & 0\\ 0 &0 &0 & -1 \end{bmatrix}  \begin{bmatrix} 1 & 0 & 0 & 0 \\ 0 & 1 & 0 & 0 \\ 0 & 0 & 0 &1 \\ 0& 0& 1 & 0 \end{bmatrix}\\
 &= \begin{bmatrix} 1& 0 & 0 &0 \\0 &1 & 0 &0 \\ 0 & 0 &0 &-1 \\ 0 & 0 & -1 &0\end{bmatrix} \begin{bmatrix} 1 & 0 & 0 & 0 \\ 0 & 1 & 0 & 0 \\ 0 & 0 & 0 &1 \\ 0& 0& 1 & 0 \end{bmatrix}\\
 &= \begin{bmatrix} 1 & 0 &0 &0 \\ 0 &1 & 0 &0\\ 0 & 0 &-1 & 0 \\ 0 & 0 & 0 & -1 \end{bmatrix} = Z\otimes I.
 \end{align*}
 Thus $C$ is in the $2$-qubit Clifford group.
 \subsection{e}
 Let $A$ be the Toffoli gate (CCNOT). For the sake of space we'll abbreviate the calculations. We have that
 $$A(I\otimes X\otimes I) A^\dagger = \begin{bmatrix} X\otimes I & 0 \\ 0 & Z\otimes Z\end{bmatrix}.$$
 This can't be written as a tensor product $\alpha V_1\otimes V_2 \otimes V_3$ with $\alpha$ scalar and $V_1,V_2, V_3 \in \{I,X,Y,Z\}$. To see this observe that if we take $V_1 \in  \{X, Y\}$ then the first $4x4$ diagonal block of $V_1\otimes V_2 \otimes V_3$ will be $0$, contradicting that it should be $X \otimes I$, and if we take $V_1 \in \{I,Z\}$ then the first $4x4$ diagonal block will be $\pm1$ times the second, but $X\otimes I \neq \pm Z\otimes Z$. Therefore no such tensor product expression for $A(I\otimes X \otimes I)A^\dagger$ is possible, and hence the Toffoli gate is not in the $3$-qubit Clifford group. $\blacksquare$
 \newpage\section{}
 \subsection{a}
 \paragraph{}
We use the same notation for gates as in question $2$. $X_4$ is given by the circuit
 \[ \Qcircuit @C=1em @R=1em {
 & \targ &   \qw & \qw\\
 & \ctrl{-1} & \gate{X} & \qw
}\]
$Z_4$ is given by the circuit
\[ \Qcircuit @C=1em @R=1em {
 & \gate{S} &  \gate{S} & \qw\\
 & \gate{S} & \qw & \qw
}\].
\subsection{b}
\paragraph{}
$X_8$ is given by the circuit (using one Toffoli gate)
\[ \Qcircuit @C=1em @R=1em {
 & \targ &  \qw & \qw & \qw & \qw\\
 & \ctrl{-1} & \targ& \qw  & \qw &\qw\\
 & \ctrl{-1} & \ctrl{-1} & \gate{X} &\qw & \qw
}\]
Bonus: Figure $4.9$ of Nielsen and Chuang gives a controlled single-qubit implementation of the Toffoli gate. Since my use of the Toffoli gate has different controls I reflect the picture, giving the following implementation of the above Toffoli gate where the first qubit is controlled by the second and third:

\[ \Qcircuit @C=1em @R=1em {
 & \gate{H} & \targ & \gate{T^\dagger} & \targ & \gate{T} & \targ & \gate{T^\dagger}& \targ & \gate{T} & \gate{H} & \qw & \qw & \qw \\
 & \qw & \ctrl{-1}& \qw  & \qw & \qw & \ctrl{-1} & \qw & \qw & \gate{T^\dagger} & \targ & \gate{T^\dagger} & \targ & \gate{S} &\qw\\
 & \qw & \qw & \qw & \ctrl{-2} & \qw & \qw & \qw & \ctrl{-2} & \qw & \ctrl{-1} & \qw & \ctrl{-1} & \gate{T} & \qw
}\]
$Z_8$ is given by the circuit
\[ \Qcircuit @C=1em @R=1em {
 & \gate{T} &  \gate{T} & \gate{T} & \gate{T} & \qw\\
 & \gate{T} & \gate{T}& \qw  & \qw &\qw\\
 & \gate{T} & \qw & \qw &\qw & \qw
}\]
\newpage
\section{}
\subsection{a}
\paragraph{}
We can express $F_d, F_d^{-1}, X_d$, and $Z_d$ as sums or rank one matrices as follows:
$$F_d = \frac{1}{\sqrt{d}} \sum_{j,k \in \Z_d} \omega^{j k} \ket{j}\bra{k}$$
$$F_d^{-1} = \frac{1}{\sqrt{d}} \sum_{j,k \in \Z_d} \omega^{- j k} \ket{k}\bra{j}$$
$$X_d = \sum_{k \in \Z_d} \ket{k+1 \mod d} \bra{k}$$
$$Z_d = \sum_{k \in \Z_d} \omega^k \ket{k}\bra{k}$$
where $\omega := \exp(2\pi i / d)$.
\paragraph{}
Then we have
\begin{align*}
F_dX_dF_d^{-1} &= \frac{1}{d} \sum_{j,k \in \Z_d} \omega^{j k} \ket{j}\bra{k}\sum_{k \in \Z_d} \ket{k+1 \mod d} \bra{k}\sum_{j,k \in \Z_d} \omega^{- j k} \ket{k}\bra{j}\\
&= \frac{1}{d} \sum_{j,k \in \Z_d} \omega^{j k} \ket{j}\bra{k} \sum_{j,k \in \Z_d} \omega^{- j k} \ket{k+1 \mod d}\bra{j} \\
&= \frac{1}{d} \sum_{j,k,\ell \in \Z_d}\omega^{\ell(k+1\mod d)}\omega^{-jk}\ket{\ell}\bra{j} \\
&=\frac{1}{d} \sum_{j,k,\ell \in \Z_d}\omega^{\ell(k+1)}\omega^{-jk}\ket{\ell}\bra{j}  \quad\text{(we may drop mod since $\omega^{\ell 0 } = \omega^{\ell d} = 1$)} \\
&= \frac{1}{d} \sum_{j,k,\ell \in \Z_d}\omega^{k(\ell-j)}\omega^{\ell}\ket{\ell}\bra{j} \\
&= \sum_{\ell \in \Z_d} \omega^\ell \ket{\ell}\bra{\ell} \quad \text{(applying Kronecker Delta summing over $k$)} \\
&= Z_d.
\end{align*}
So $F_d X_d F_d^{-1} = Z_d$. Thus we have (we will show the desired form of $F_d^2$ is part $b$)
\begin{align*}
F_dZ_d F_d^{-1} &= F_d^2 X_d F_d^{-2} \\
&= \sum_{k \in Z_d} \ket{-k \mod d} \bra{k} \sum_{k \in \Z_d} \ket{k+1 \mod d}\bra{k} \sum_{k \in \Z_d} \ket{k}\bra{-k\mod d} \\
&= \sum_{k \in Z_d} \ket{-k \mod d} \bra{k} \sum_{k \in Z_d}\ket{k+1 \mod d}\bra{-k \mod d} \\
&= \sum_{k \in Z_d} \ket{-(k+1) \mod d} \bra{- k \mod d} \\
&= \sum_{k \in Z_d} \ket{k+1 \mod d} \bra{k \mod d} \\
&= X_d.
\end{align*}
Hence $F_d Z_d F_d^{-1} = X_d$.
\subsection{b}
Let $p \in \Z_d$. Then we have
\begin{align*}
F^2_d\ket{k} &= \frac{1}{d} \sum_{j,k \in \Z_d} \omega^{j k} \ket{j}\bra{k}\sum_{j,k \in \Z_d} \omega^{j k} \ket{j}\bra{k} \ket{p} \\
&= \frac{1}{d} \sum_{j,k \in \Z_d} \omega^{j k} \ket{j}\bra{k}\sum_{j \in \Z_d}\omega^{jp}\ket{j} \\
&=\frac{1}{d} \sum_{j,k \in \Z_d} \omega^{k(j+ p)}\ket{j}.
\end{align*}
For fixed $j$, $$\frac{1}{d} \sum_{k \in \Z_d} \omega^{k(j+p)} = 1$$ if and only if $$j+p = 0 \mod d$$ and otherwise $$\frac{1}{d} \sum_{k \in \Z_d} \omega^{k(j+p)} = 0$$. Hence 
$$F^2_d \ket{k} = \ket{-k \mod d}$$
and so
$$F^2_d = \sum_{k \in \Z_d} \ket{-k \mod d} \bra{k}.$$
\subsection{c}
Since $-0 \mod d \equiv 0$ and $-d/2 \mod d \equiv d/2$ ($d$ is even), we have that
$$F^2_d \ket{0} = \ket{0}$$
and $$F^2_d\ket{d/2} = \ket{d/2}$$
so $1$ is an eigenvalue of $F^2_d$. Further, let $k \in \Z_d \backslash \{0, d/2\}$. Then
$$F^2_d(\ket{k} + \ket{-k \mod d}) = (\ket{k} + \ket{-k \mod d})$$
and
$$F^2_d(\ket{k} - \ket{-k \mod d}) = -\ket{k} + \ket{-k\mod d}$$
so $-1$ is another eigenvalue of of $F^2_d$, and we have found a basis of $\C^d$ consisting of eigenvectors of $F_d^2$ (the two eigenvectors generated for each such $k$ above, and $\ket{0}$ and $\ket{d/2}$). Hence we have found all eigenvalues of $F^2_d$. That is the eigenvalues of $F^2_d$ are precisely $\pm 1$.
\subsection{d}
\paragraph{}
We check the action of $F_4$:
\begin{align*}
F_4(X\otimes I) F_4^{-1} &= \frac{1}{4} \sum_{j,k = 0}^3 \omega^{jk}\ket{j}\bra{k}(\ket{2}\bra{0} + \ket{3}\bra{1} + \ket{0}\bra{2} + \ket{1}\bra{3})\sum_{j,k = 0}^3 \omega^{-jk}\ket{k}\bra{j}\\
&= \frac{1}{4} \sum_{j,k = 0}^3 \omega^{jk}\ket{j}\bra{k}(\sum_{\ell=0}^3 \ket{\ell}\bra{\ell + 2 \mod 4})\sum_{j,k=0}^3\omega^{-jk}\ket{k}\bra{j}\\
&=  \frac{1}{4} \sum_{j,k = 0}^3 \omega^{jk}\ket{j}\bra{k}\sum_{\ell,j=0}^3 \omega^{-j(\ell+2)}\ket{\ell}\bra{j} \\
&=\frac{1}{4} \sum_{j,k,\ell=0}^3 \omega^{jk}\omega^{-\ell(k+2)}\ket{j}\bra{\ell} \\
&=\frac{1}{4} \sum_{j,k,\ell=0}^3\omega^{-2\ell} \omega^{k(j-\ell)} \ket{j}\bra{\ell}\\
&=\sum_{\ell=0}^3\omega^{-2\ell} \ket{\ell}\bra{\ell}\\
&= I \otimes Z.
\end{align*}
\begin{align*}
F_4(I\otimes X) F_4^{-1} &= \frac{1}{4} \sum_{j,k = 0}^3 \omega^{jk}\ket{j}\bra{k}(\ket{1}\bra{0} + \ket{0}\bra{1} + \ket{3}\bra{2} +\ket{2}\bra{3})\sum_{j,k = 0}^3 \omega^{-jk}\ket{k}\bra{j} \\
&= \frac{1}{4}(\sum_{j,k=0}^3(\omega^{j} + \omega^{-k} + \omega^{3j - 2k} + \omega^{2j - 3k}))\ket{j}\bra{k} \\
&= \ket{0}\bra{0} + -i\ket{3}\bra{1} - \ket{2}\bra{2} + i \ket{1}\bra{3}
\end{align*}
which cannot be written as a tensor product of $\{I, X, Y, Z\}$. Hence $F_4$ is not in the Clifford group.
\newpage\section{}
\paragraph{}
Suppose that $m$ is prime. Let $\lambda \in \Z_m$. Let $r \in \Z_m\times \Z_m \backslash \{(0,0)\}$. We may assume without loss of generality that $r_1 \neq 0$. We define $L_\lambda$ and $\ket{L_\lambda}$ as in the question preamble. Then we have
\begin{align*}
(F^{-1}_m \otimes F^{-1}_m) \ket{L_\lambda} &= \frac{1}{\sqrt{m}} \sum_{j,k \in \Z_m} \frac{1}{m} \sum_{(x_1,x_2) \in L_\lambda}\omega^{x_1j + x_2k}\ket{j}\ket{k}.
\end{align*}
For a fixed $j,k \in \Z_m\times \Z_m$ the $\ket{j}\ket{k}$ term in the above vanishes iff $\sum_{(x_1,x_2) \in L_\lambda}\omega^{x_1j + x_2k} = 0$.  From the hint we know that $$\sum_{s \in \Z_m} \omega^s = 0.$$
Hence the $\ket{j}\ket{k}$ term vanishes iff for all $s \in \Z_m$ there exists $(x_1,x_2) \in L_\lambda$ such that $x_1j + x_2k = s$ (this establishes a bijection from $L_\lambda$ to $\Z_m$, proving $\abs{L_\lambda} = m$, as we'll see below such $x$ must be unique).
\paragraph{}
For all $s \in \Z_m$ we want $x \in \Z_m \times \Z_m$ satisfying
$$x_1j + x_2k = s \quad\text{and}\quad x_1r_1 + x_2r_2 = \lambda.$$
This system has no solution provided
$$jr_1 = kr_2 \quad \text{and} \quad s -kr_1^{-1}\lambda\neq 0 \quad(\text{mod } m)$$
The first condition is satisfied iff $(j,k)$ is a scalar multiple of $(r_1,r_2)$. Given that the first condition is satisfied, the second condition is satisfied for all $s \in \Z_m$, save $s = -r^{-1}_1k\lambda$. Thus we have
$$(F^{-1}_m \otimes F^{-1}_m) \ket{L_\lambda} = \frac{1}{\sqrt{m}} \sum_{\alpha \in \Z_m} \omega^{\alpha \lambda}\ket{\alpha r_1}\ket{\alpha r_2}.$$
So measuring this state will yield a scalar multiple of $(r_1,r_2)$ each with equal probability. This does not reveal any information about $\lambda$.
\end{document}
